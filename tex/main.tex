
\chapter{Социальное программирование}

Большинство людей проживают свою жизнь как во сне. Они похожи на лунатиков. Осознание этого факта в конце концов наделит вас фантастическими социальными навыками. Но прежде чем это произойдёт, мы должны проснуться и осознать ту чепуху, которой нас учили. Вы можете остановиться на секунду и подумать откуда в действительности мы получаем информацию касающуюся секса, свиданий и вообще отношений.

Какие общепринятые мысли на этот счёт мы сможем найти в обществе?

--- Если у тебя есть деньги, то у тебя будут и женщины. Если у тебя нет женщины, то пойди и заработай кучу денег. Таким образом, ты сможешь привлечь женщину, рассказав ей о том, как много у тебя денег.

--- Красавчики всегда получают красивых девушек. Внешность это вторая важная вещь после денег.

--- Вам так же предлагают способы, как вы можете получить понравившуюся вам девушку. Она должна увидеть, как вы её любите, на какие жертвы идёте ради неё. Если вы сможете просто доказать ей как вы её любите, то она обязательно ответит вам взаимностью.

--- Женщину можно логически убедить почувствовать влечение к мужчине. Когда у вас первое свидание с девушкой вы должны купить ей выпивку, а так же подарить цветы и тогда вы убедите её полюбить вас.

--- Если вам понравилась девушка, то вы должны всё тщательно продумать, прежде чем подходить к ней. Если вы будете достаточно осторожны, то не сделаете ошибок при подходе, которые могут обидеть её.

--- Если вы кого-то любите, то вы должны следовать зову своего сердца и рассказывать девушке о том, как вы её любите и как страдаете по ней. Если вы будете это делать --- то она непременно ответит вам взаимностью.

--- Любовь это уникальное чувство, вы должны это понимать. На всей земле у вас есть только одна вторая половинка. Любовь не приходит дважды.

--- Привлекательных девчонок никогда не отвергают. Если вы привлекательный парень, вас так же никогда не отвергнут. Если ты не можешь получить девушку, то это лишь потому, что ты не привлекательный внешне.

--- Окружающие всегда внимательно наблюдают за парнями, которые пытаются знакомиться с женщинами, они даже могут посмеяться над ними вместе со своими друзьями. Поэтому что бы не попасть в неловкое положение, необходимо сначала выяснить нравитесь ли вы девушке, прежде чем знакомиться к ней.

--- Если девушка переспала с вами сразу после знакомства, то скорее всего она просто шлюха. Но если она задерживает секс и заставляет вас постараться для того что бы переспать с ней то это «честная» девушка и в прошлом естественно не вела беспорядочной половой жизни. В сущности, девушка которую трудно получить --- это идеальная девушка для построения долговременных отношений.

--- Парни кажутся одержимыми сексом. Парни наслаждаются сексом больше чем девушки. Вот почему парни пытаются обмануть девчонок --- они просто одержимы сексом.

--- Если вы не будете таким как все, то вы не будете нравиться людям. Если вы находитесь в клубе или на дискотеке, то вам абсолютно необходимо пить алкоголь. Вам нужно держать стакан с выпивкой в руке на тот случай, если на вас кто-нибудь смотрит.

--- Все парни, которые хорошо танцуют, нравятся девочкам. Что бы снять девочку в клубе вам необходимо научится танцевать. Тогда вы сможете подойти к любой понравившейся девушке, которая танцует вместе со своими друзьями, вклиниться между ними и поразить её своим танцем. Затем, когда она по настоящему возбудиться от ваших телодвижений вы сможете увести её домой и трахнуть.

--- Если вы видите что девушка чем-то расстроена или с кем-то ссорится, то это хорошая возможность, что бы подойти и успокоить её или вмешаться в ситуацию. Это именно то, чего она хочет в данный момент. Это даст ей возможность увидеть какой вы хороший.

--- Влечение вызывается феромонами, симметрией лица и определённым телосложением. Если у вас этого нет то вам определённо не повезло\ldots

Всё это полнейшая ерунда.

Нельзя сказать, что бы было абсолютно невозможно привлечь женщину на основании этих предположений. Множество парней именно так и поступают (остаётся только удивляться, что земля по-прежнему населена людьми). Но для того что бы понять как на самом деле можно привлечь женщину, необходимо отказаться от этих идей. Вы не должны безоговорочно принимать на веру эти, либо ещё какие то другие идеи. Выйдите на улицу и проверьте их. Вы довольно быстро выясните, как девушки реагируют на определённое поведение.

Так в конце концов, откуда взялись эти предположения? Кто несёт ответственность за их появление?
\DEFINE{«Социальное программирование»}

С самого раннего возраста, нам насаждаются определённые поведенческие стереотипы, шаблоны по которым нам необходимо вести себя. Назовём это социальным программированием.

Мы присваиваем опыт других людей как если бы это был наш собственный опыт поскольку он был выработан на протяжении многих поколений (подавляющее большинство людей просто не могут ошибаться) По этой причине мы полагаем что все те шаблоны и модели поведения, которые предлагает нам общество, являются абсолютно правильными и не требуют дополнительной проверки. Большинство людей никогда не задумывались о том, кто, что и как влияет на их сознание. Даже те люди, кто это делал, вполне возможно до конца не осознают до какой степени их восприятие мира и поведение обусловлено социальным программированием.

Общество не всегда терпимо относится к проявлениям сексуальности. Секс это слишком мощная штука, люди часто просто не знают, что с этим делать. Начало занятий сексом часто рассматривается как некий переход от мальчика к мужчине. Отношение к сексу определяет степень нравственности человека.

Часто сексу даже приписывается некий тайный сакральный смысл. Тема секса для многих является табу, и некоторые люди не могут разговаривать о нём, используя цензурные выражения. В наши дни многие парни находятся в затруднительном положении, поскольку все их знания о сексе являются неверными, так как получены через социальное программирование.

Бессмысленные и часто даже вредные убеждения, которые парни получают в комплекте с социальными установками создают хаос в их умах. Они по-настоящему нуждаются в знаниях, которые необходимы каждому мужчине. Знания о том, как привлечь женщину.

До тех пор пока вы сами не разберётесь в этом вопросе, общество будет назойливо навязывать вам свои представления о том, как вам получить желаемую девушку. Оно будет делать это с помощью разнообразных инструментов. Через фильмы, книги, музыку, друзей и даже седых профессоров, вам будут говорит что: мужчины любят секс больше чем женщины, что для того что бы получить женщину парень должен иметь много денег и хорошую внешность и другую хренотень.

Вам будут говорить обо всём, кроме того, что по-настоящему важно. Вам будут говорить о том как одеваться, как говорить комплименты, как произвести впечатление, как использовать хорошие манеры, как долго необходимо ждать что бы сделать следующий шаг. Всё это на самом деле ерунда, не имеющая к соблазнению никакого отношения. Это даже контрпродуктивно. Это только будет отвлекать вас от того, что по настоящему важно.

Почему это ерунда?

Давайте рассмотрим подробнее некоторые социальные установки:

\begin{enumerate}
\item Если вы её впечатлите, то она будет поражена и немедленно влюбится в вас.
\item Если вы будете милым и будете заботиться о ней, оказывать различные услуги, то она увидит, как вы о ней стараетесь, и немедленно влюбится в вас.
\item Если вы будете соглашаться с ней во всём и вести себя так, как будто у вас много общего, то она решит что вы совершенный мужчина и полюбит вас.
\item Если вы ей расскажете о том как вам плохо без неё, и как вы страдаете, то это тронет её сердце и она немедленно полюбит вас.
\item Если у вас хорошая внешность или много денег, то как только она узнает об этом, то полюбит вас.
\item Если вы станете её лучшим другом, то она со временем заметит все ваши замечательные качества и влюбится в вас..
\end{enumerate}
Если вы клёвый парень то это может сработать, но ни одна из этих социальных установок не работает сама по себе, потому что:
\begin{enumerate}
\item Если вы попытаетесь впечатлить её, то это метасообщает что у вас была причина для того что бы впечатлять её, поскольку она более привлекательна, чем вы.
\item Если вы будете милым парнем, будете заботится о ней, оказывать ей различные услуги, вы будете казаться скучным, потому что вы ставите её на пьедестал как и каждый обычный парень, вы не можете бросить ей вызов.
\item Если вы будете соглашаться с ней во всём и вести себя так, как будто у вас много общего, то у неё возникнет ощущение что вы слишком стараетесь ради неё и у вас небольшой выбор среди женщин.
\item Если вы расскажете ей как вам плохо, как вы страдаете без неё, то этот номер не пройдет. Представьте, что вам сказала подобное девушка, которая вам не нравится. Вы в неё влюбитесь?
\item Если у вас есть деньги или хорошая внешность, то это может помочь вам. Однако парень с привлекательной личностью может увести вашу девушку и вас не спасёт не внешность не деньги.
\item Если вы станете её лучшим другом, то вы будете последним парнем с кем она займётся сексом.
\end{enumerate}

С возрастом, мы перестаём задумываться над всеми этими вещами. Мы просто пользуемся тем, что предлагает нам общество. Когда мы видим парня, который покупает женщинам выпивку, мы решаем что так и нужно делать. Совсем не обязательно всегда будет неправильным покупать девушке выпивку. Однако с другой стороны, совсем не обязательно что это всегда будет правильным. Но покупая женщине выпивку ты просто становишься похожим на других обычных парней. Ты просто становишься одним из толпы парней которые унижаются перед ней, которые не способны вести нормальную беседу с ней, поскольку они нуждаются в ней. Покупка женщине напитка может казаться удобным способом начать диалог даже когда девушка даже не привлечена вами. Но все-таки покупка женщине напитка может запускать её мысли не в нужном направлении. Если вы купили женщине выпивку, и после этого ей понравились, это лишь означает что она увидела что-то привлекательное в вашей личности. Сам факт покупки никак не может повлиять на её отношение к вам. Женщина может быть привлечена только вашей личностью и больше никаким иным способом.

Девочек в первую очередь привлекают парни, которые демонстрируют им свою личность. Если ты попытаешься оказывать девушке какие либо услуги, предлагать помощь, поддержку, до того как ты продемонстрировал ей свою личность, то скорее всего она сочтёт тебя «непривлекательным», ещё до того, как узнает тебя лучше. Поздравляю\ldots ты стал ещё одним дурачком.
\DEFINE{«Упрашивающий»}

Если вы делаете что либо для девушки, что вы обычно не делаете не для кого, что не является нормальным для вас, и своими действиями надеетесь заслужить симпатию со стороны девушки в ответ, --- то это мы называем «упрашиванием». Упрашивание это то, что возникает между двумя людьми, когда один из них обладает более высокой социальной ценностью, чем другой. Упрашивание подразумевает, что девушка обладает более высокой ценностью чем вы, и поэтому вы не можете привлечь её исключительно своей личностью.

Это может включать в себя любые попытки «принести пользу» или «оказать услугу» девушке, с которой у вас ещё не было секса. Например: покупать выпивку, делать комплименты, дарить подарки и другие уже упомянутые выше вещи. Избегать упрашивания не означает, что вы никогда не будете делать этих вещей. В сущности, постоянный сознательный контроль того, что бы не сделать ничего, что можно расценить как упрашивание, то же является формой упрашивания. Правильнее, что бы определить является ли это упрашиванием, будет спросить себя «Зачем я делаю это? Я просто веселюсь или я пытаюсь понравиться? Смогу ли я ей понравится без этих вещей? Если я поступлю подобным образом это нормально для меня, или же пытаюсь делать то, чего обычно не делаю никогда ни для кого?» Если ваш ответ да, то это упрашивание.

Не является упрашиванием, если вы делаете что то, что является частью вас. Для вас должна быть чёткая граница внутри вас, какое поведение вы принимаете, а какое не приемлите ни от других ни от себя. Если вы можете бросить женщине вызов, то это будет отличать вас от других парней. Но вы должны делать это только потому что таков ваш взгляд на мир --- вы не нуждаетесь в одобрении со стороны женщины, вы чувствуете себя уютно в своём мире. Вы не должны этого делать только потому что так делают другие парни, которые считают что это привлечёт женщин.

Большинство парней просто не понимают, что привлекает женщин. И именно потому, что они этого не понимают, на их сознание продолжают оказывать влияние средства массовой информации. Фильмы, журналы, телевидение и др. убеждают их в том что для того что бы иметь много красивых девочек необходима внешность, деньги и прочая ерунда. В сущности, ни одна из этих вещей не является необходимой. Когда мужчина имеет какие то недостатки внешности, денег и т. д. то он использует это как оправдание почему он не имеет успеха у женщин. Например, мужчина думает что для успеха у женщин необходимо иметь красивые черты лица --- несмотря на то что он постоянно видит парней, которые не обладают особой внешней привлекательностью и имеют много красивых женщин. То же самое относится к лысым, толстым, старым, бедным, некрасивым и т. д\ldots Это всего лишь ярлыки которые вы вешаете на себя для того что бы оправдать свои неудачи. В сущности всё это неважно. Для понимания того, как работает влечение вы должны отбросить убеждения, навязанные вам обществом. Страсть, влечение, влюблённость, сексуальное желание --- всё это обусловлено разними причинами. Вы не должны позволять себе делать заключения о том, как устроены отношения между мужчиной и женщиной используя ложный фундамент. Вас не должно удивлять, когда вы видите парней, которые успешны с женщинами, несмотря на свои какие то внешние недостатки. Вы должны отбросить все свои шаблоны и модели поведения навязанные вам обществом, отвергнуть социальное программирование, и копнуть глубже. Если эти представления об сексуальном влечении являются ложными, то что тогда правда?

\RULE  Влечение --- базируется на эмоциях, а не на логике. То, что женщина хочет логически, очень редко совпадает с тем, на что она эмоционально откликается. Для того что бы привлечь женщину вы должны взаимодействовать с ней на эмоциональном уровне, а не на логическом. Это первый сдвиг в мышлении. Первое что мы видим, копнув поглубже, это то --- что как мужчины так и женщины откликаются эмоционально на одну и ту же вещь. Это ценность.
\DEFINE{«Ценность»}

Ценность, социальная ценность, или статус могут иметь много форм. Это универсальные понятия, мы можем найти их в любом обществе. Некоторые виды ценности могут иметь значение только в определённых обществах или даже ситуациях. Ценность это то, что может дать один человек другому для того, что бы увеличить шансы последнего на успешное выживание и размножение. Кроме того, ценность включает так же положительные эмоции, которые один человек может предложить другому. Но в то же самое время наши эмоции могут заставлять нас делать вещи неразумные. У ценности есть свои изъяны, поскольку эмоции которые с ней связаны и поведение которое эмоции запускают, могут быть как полезными, так и вредными для нас, с точки зрения выживания и размножения. Всё это можно объяснить с эволюционной точки зрения. Учёные знают и объясняют это поведение с научной точки зрения, но нам пока не зачем вдаваться в детали. Просто мы должны знать общие положения.

\RULE  Женщины эмоционально реагируют на тех мужчин, от которых она может получить потомство с более высоким уровнем выживания/размножения. Милый парень который упрашивает её, не является таковым. Исключая конечно то, что милый парень с большей степенью вероятности будет находиться рядом и заботится о ней и её потомстве. --- И здесь находится лазейка для женщин. Менструальный цикл женщины позволяет ей скрыть момент, наиболее подходящий для зачатия от своего постоянного мужчины и забеременеть от мужчины с более высокими показателями по шкале выживание/размножение. Таким образом женщина может получить всё самое лучшее: секс и лучшее потомство от привлекательного мужчины и заботу самого нежного партнёра.

\RULE  Женщина подходит к выбору человека который будет длительное время заботится о ней и её потомстве логически и медленно, в то время как случайных сексуальных партнёров она выбирает быстро и эмоционально. Женщина моментально может почувствовать что мужчина привлекает её как сексуальный партнёр, но что бы определить будет ли мужчина хорошо заботится о ней и её потомстве потребуется время. Вот здесь находится лазейка для мужчин. Если мужчина до секса показывает высокую социальную ценность, а после секса ведёт себя как нежный и заботливый опекун (или возможно даже он научился метасообщать оба эти качества, в зависимости от того, что женщина ищет) тогда он сможет получить всё самое лучшее. Быстрый секс со многими женщинами и длительные отношения и связь с одной женщиной. Всё это может казаться разрушающим, но вы и не обязаны использовать все эти возможности. Существуети «счастье» в традиционном понимании этого слова, когда мужчина встречает одну девушку и безумно любит её до конца жизни и не изменяет ей.

Девушка же видит в нём как привлекательного самца так и заботливого опекуна. Все счастливы. Было бы превосходно если бы каждый из нас нашёл себе идеальную пару, где влечение друг к другу продолжалось бы постоянно, мы имели бы много секса и необыкновенную связь друг с другом. Но к сожалению это встречается нечасто. Итак двигаемся вперёд\ldots То что является ценностью для мужчины может и не быть ценностью для женщины. Для мужчин одной из самых основных ценностей женщины является её внешность --- но для женщин внешность мужчины далеко не главный фактор. Не нужно проецировать мужское восприятие ценности партнёра на женщин.

\RULE  В обществе установлены различные стандарты для оценки мужчин и женщин. Мужчин обычно приветствуют за силу характера и лидерские качества, женщин оценивают по их внешности и житейской смекалке. По причине того, что человек это животное стадное, его уровень выживания и размножения зависит от успеха в общественных отношениях. Как мужчины так и женщины склонны развивать в себе те качества, которые увеличивают их социальную ценность. Когда это происходит естественным образом, то они развивают в себе то, что считают ценным, хотя многое из того что является ценным, отфильтровывается их восприятием поскольку они считают это неважным. По этой причине, мужчины предпочитают развиваться там где есть логика и соперничество: спорт, машины, увеличение физической силы и т д. Обучаясь всему этому они могут повысить свои шансы на силу и безопасность в этом мире. Женщины, в большинстве случаев предпочитают развиваться в социальном общении: учатся языку тела, отношениям, романтике они учатся взаимодействию с окружающим миром и получают соответствующий опыт.
\DEFINE{«Метасообщения»}

В силу того, что для них это более важно, женщины обычно лучше, чем мужчины понимают трудноуловимые нюансы поведения (не видимые для многих мужчин). Они могут многое сказать о человеке наблюдая за ним по языку тела, контакту глаз, выражению лица, тону голоса, по тому как он движется, как говорит. Все эти вещи, которые не лежат на поверхности и скрыты от глаз большинства мужчин, но при этом являются важным каналом получения информации --- мы называем метасообщения. В доисторические времена, если мужчина терял статус, (общественное положение) в своей группе, то через подобные метасообщения он демонстрировал женщинам своего племени что его ценность упала. (Изменялось его поведение) Подобным же образом, если он продвигался вперёд в социальном положении внутри племени, то возрастающую социальную ценность так же передавали его метасообщения. Наблюдая эти тонкие изменения в поведении, женщины могли остро чувствовать, какое положение обществе занимает мужчина, какова его социальная ценность. И что здесь наиболее следует отметить, это то что женщины не понимали этого сознательно не делали каких либо рассуждений они просто эмоционально чувствовали ауру мужчины.

\RULE  Ценность мужчины метасообщается через тонкие невербальные сигналы, через модели поведения и определяет насколько привлекательным он будет в глазах женщин и как они будут реагировать на него. Что особенно интересно, это то что ценность мужчины для женщины может очень быстро меняться, а ценность женщины для мужчины изменяется редко и медленно.

\RULE  Женщины постоянно оценивают поведение мужчины на протяжении всего процесса общения, по этой причине, отношение к мужчине может измениться буквально за мгновения. Они оценивают это постоянно, секунда за секундой. Ценность женщины для мужчины не может измениться резко за минуты или даже дни. Но ценность мужчины может измениться в одно мгновение. Для мужчины это возможно стать более клёвым, уверенным, доминирующим --- ваша ценность может вырасти моментально через метасообщения. Но в то же самое время, для него возможен и обратный процесс. За непродолжительное время он может потерять ценность и соответственно влечение со стороны женщины. Если парень встречает красивую девушку, но её личностные качества ему не очень нравятся, он по прежнему будет считать её привлекательной. Но если девушка думает о парне, что он физически привлекательный, и затем парень начинает косячить, то она скажет о нём «не мой тип мужчины» и пойдёт того, чья личность для неё будет более привлекательной.

\RULE  Личность мужчины (доминирование, уверенность в себе, клёвость, ум, юмор и т д) более важны для женщин, потому что двойные социальные стандарты диктуют что это в большей степени показывает его социальную ценность, чем ценность женщины. Для женщин это решающий фактор при выборе партнёра (даже если парень обладает какими-то внешними данными типа денег или физической привлекательности). Эти вещи могут помочь создать первоначальное влечение, но в дальнейшем на одном этом долго не продержаться. Эти объяснения не отражают всей сложности социальных отношений. Это просто чрезмерное обобщение, которое помогает нам понять суть. Они хорошо служат для первоначального понимания социальной динамики, хотя многое ещё остаётся непонятным. Итак, для чего мы так тщательно рассматриваем эволюционные процессы и социальную структуру общества? В конце концов мы все хотим быть принятыми в обществе, но для некоторых из нас так же важно не просто быть принятыми, а занять в обществе высокое положение, что в конечном счёте отразится на ваших возможностях заняться сексом. Но часто, именно под влиянием общества, мы начинаем заблуждаться относительно того, что действительно привлекает женщин. Остановитесь и обдумайте, какие мысли доминируют в вас на протяжении жизни. Возможно, это будут следующие мысли «Заставь людей полюбить себя. Много работай, зарабатывай деньги и тогда ты сможешь купить то, что поможет тебе» Это ни чей то заговор или плохие намерения. Никто конкретно в обществе не несёт за это ответственность. Просто слепцы ведут слепцов. Мы живём во время которое никогда не существовало прежде. Эра мгновенного наслаждения. Как в голливудских фильмах --- плохой парень, хороший парень и в конце фильма --- торжество общественных норм. Люди предпочитают сенсации, слухи, драмы то, что не заставляет их думать. На каждом углу вам предлагают волшебные таблетки, мгновенное лечение от вашей жажды думать. В конце концов люди не просто начинают ожидать волшебных таблеток, они начинают их требовать. Думать критически и выходить из зоны комфорта не обязательно. Вы как хороший солдат, просто исполняете команды. Хороший солдат не обсуждает приказы, он их выполняет.

К лучшему или к худшему, но такова форма взаимодействия, на которое общество вас запрограммировало.

\RULE  Большинство людей проживают свою жизнь словно в трансе, очень редко останавливаясь что бы задать себе вопрос правильно ли они живут. В большинстве случаев, они ориентируются на мнение общества, о том как «правильно» и как «неправильно» жить, очень редко принимая решения на основании собственных размышлений. Общество имеет множество инструментов для поддержания существующего порядка вещей. Общество объясняет отдельным его членам, каким образом они могут добиться желаемого. Мы запрограммированы верить в то, что чтобы добиться желаемой женщины мы должны отвечать поверхностным стандартам общества. И только соблюдая социальные стандарты, мы получаем право вести себя привлекательно.

Мы запрограммированы думать неправильно о том что только мужчины с поверхностными качествами являются привлекательными для женщин, когда на самом деле имеют большее значение личностные качества и метасообщения, на которые и реагируют женщины. Вместо того что бы подходить к женщинам и метасообщать уверенность и индивидуальность, мы подходим к женщине с предположением, что должны упрашивать или впечатлять её.

Или хуже, мы не делаем даже этого, мы просто работаем над улучшением своей внешней привлекательности (деньги, внешность) надеемся что однажды, женщина заметит это и полюбит нас. Многие годы мы следуем этим убеждениям, вместо того что бы развивать собственную личность, мы тратим наше время на то что бы играть в чужие игры, которые не мы придумали и которые мы по-настоящему не понимаем. Те внешние вещи, которые мы привносим в нашу жизнь, никогда не будут частью того, кто мы есть. Вы можете увидеть парней у которых есть красивая внешность или много денег, и которые успешны с женщинами, а затем сделать ложный вывод о том «Вот что я должен иметь, для того что бы иметь красивых женщин». Но на самом деле, не внешность или деньги дают парням девчонок и позволяют вести себя уверенно и привлекательно.

Многие парни хотят научиться вести себя уверенно. Это во многом зависит от того «покупаешься» ли ты на социальные стандарты или нет, что ты их заслуживаешь красивых женщин. Вы должны полностью осознавать, что исключительно ваше поведение делает вас привлекательным, что именно поведение метасообщает женщинам, кем вы являетесь. Часто неопытные парни пытаются логически убедить девушку почувствовать влечение к нему, они идут домой и всю ночь напролёт думают о том как убедить её ещё больше, какие аргументы привести. Парни, которые успешны с женщинами, привлекают женщин быстро и без логического убеждения. Они просто привлекательные. Это не то, что они делают. Это часть того, кто они есть.

\RULE  Быть привлекательным в глазах женщин это не то, что ты делаешь --- это то, кто ты есть. Этого нельзя добиться через шаблоны или тактики пикапа. Надо лишь по настоящему понимать что означает «быть собой» и демонстрировать лучшие качества каждому, кого ты встретишь.

\RULE  Для того что бы привлечь конкретную девушку ты должен быть успешен с женщинами в общем. Если ты хочешь вернуть старую подружку, или получить девушку, которая много значит для тебя в данный момент, то скорее всего у тебя ничего не получится.
\DEFINE{«Реагирующий»}

«Реагирующий» значит давать быструю видимую эмоциональную реакцию и не замечать существование большей проблемы. Быть реагирующим, означает быть менее привлекательным, потому что вы концентрируетесь на том, что бы забирать ценность у девочки вместо того, что бы развивать её в себе. Быть реагирующим на женщину или одержимым женщиной, обычно означает что вы её уже потеряли, даже не осознавая это. Всегда более реалистичным будет зажечь с новой девочкой, чем пытаться исправлять то представление о вас, которое уже существует у старой девочки. Когда у девочки уже сформировалось определённое представление о вас, то очень трудно будет что либо изменить в её восприятии. Но когда Вы встречаете девочку впервые, Вы создаете полностью новый опыт.
\DEFINE{«Проактивный»}

Быть проактивным означает, что вы знаете, с какими трудностями вам придётся столкнуться в будущем, и заранее к ним готовитесь, вырабатывая определённые навыки. Это означает что вы будете расширять свой опыт, встречаясь со многими женщинами и практикуясь, и когда вы встретите нужную вам девушку, то у вас всё получится. По сути, это означает что вы должны временно отказаться от настроя завести подружку (снять определённую девочку). Вы должны настроить себя на то --- что бы стать успешным с женщинами в целом --- и тогда, когда вы встретите девочку которая вам действительно нужна, то вы будете её достойны. Затем, вы подойдёте к точке в которой осознаете свою ценность и как её передать. Вам не нужно стремится к счастью с каким то одним человеком. Вы просто должны быть тем --- кто даёт ценность женщине, потому что вы интересны для них и вам не нужно заполнять внутреннюю пустоту.

И внезапно, вы сделаете скачок от парня который хватается за любую девушку, к парню который по настоящему сможет выбрать ту, которая по настоящему ему подходит. Это является правдой относительно того, как вам найти нужную девушку. Без социального программирования. Без отговорок и оправданий. Это может быть не просто. Но однажды вступив на путь развития, вы достаточно скоро сможете ожидать вознаграждения за свои усилия.
\chapter{Ценность}

Всё что мы воспринимаем в окружающем мире мы интерпретируем (объясняем) субъективно. Это прищуренный взгляд сквозь пелену эмоций. Наш разум существует в состоянии бесконечной войны между логикой и эмоциями, это две постоянно противоборствующие стороны. Наши эмоции постоянно толкают нас к поиску ценности (наши эмоции говорят нам что для нас полезно) Но иногда наши эмоции заставляют нас делать вещи, которые не имеют смысла с логической точки зрения. (или мы просто думаем что не имеют)
\DEFINE{«Обратная рационализация»}

Вследствие того. что наши решения, принятые под влиянием эмоций, могут доставлять нам дискомфорт, мы придумываем различные логические оправдания нашему поведению, уже после совершённого действия. Таким образом, мы можем думать что действовали логически, хотя на самом деле все решения принимались под влиянием эмоций. Этот процесс был назван обратной рационализацией. Обратная рационализация это процесс который происходит практически у всех людей, но с разной степенью выраженности. Это очень важный момент --- на нём основан фундаментальный принцип влечения.

\RULE  Используя процесс рационализации, мы будем объяснять наши представления о других людях (насколько они привлекательны, насколько с ними весело) базируясь на их ценности для нас. Мы склонны сосредотачивать внимание на отдельных моментах личности другого человека. Логически мы всегда можем воспринимать человека достаточно объективно, в же время эмоции будут принуждать нас концентрироваться на отдельных качествах этого человека, что бы включился процесс рационализации и тогда мы сможем «логически объяснить» те чувства, которые мы к нему испытываем. Это означает что всякий раз когда мы взаимодействуем с другими людьми, мы избирательно сосредотачиваемся на отдельных качествах другого человека и используем эти качества как «основания» для «объяснения» наших чувств к нему. Мы не воспринимаем те качества, которые противоречат нашим представлениям об этом человеке. Подумайте над этим\ldots Если вы как большинство парней придаёте слишком большое значение каким то поверхностным деталям то возможно вы упустили нечто главное, то от чего зависит их отношение к вам- ваша ценность для них. Например, вы можете быть парнем с высокой ценностью но плохими манерами и тогда, большинство женщин рационализируя своё отношение к вам, скажут «в нём живёт дух свободы и он подчиняется лишь своим правилам». Или вы можете быть парнем с низкой ценностью и хорошими манерами и тогда большинство женщин рационализируя своё отношение к вам скажут «он не мой тип». Так или иначе, но именно ваша ценность будет определять отношение к вам женщин. Девочка может думать, что её не заботит, имеет ли парень социальную ценность Она так же может думать что ей нужен парень с которым бы она почувствовала особую связь или парень, который бы заставил её смеяться. Но на самом деле, именно ценность парня определяет будут ли его шутки казаться ей смешными или дурацкими. Именно ценность может сделать так, что девочка почувствует «особую связь» с парнем, которого она видит первый раз в жизни. Вспомните школьные годы, «клёвый парень», лидер класса, мог сказать всё что угодно и окружающие думали что это весело. Вы можете так же вспомнить случаи когда девочка чувствовала «особую связь» с клёвым парнем, который даже и не подозревал о её существовании.

Построение «особой связи» с девочкой, а так же «уместность» и «забавность» ваших шуток придёт тогда, когда вы будете обладать достаточной ценностью.
\DEFINE{«Социальный альянс»}

Как существа социальные, мы стремимся создавать «социальные альянсы» особенно с теми людьми, кто увеличивает наши шансы на выживание и размножение, а так же дарит нам положительные эмоции. В нашем социальном окружении всегда есть люди к которым мы тянемся, люди к которым мы безразличны, и люди которые отталкивают нас. Ввиду того, что мы имеем ограниченное время и энергию на создание социальных альянсов, ты мы стремимся образовывать союзы с теми людьми, общение с которыми, как мы думаем, принесёт нам наибольшую выгоду. По отношению к большинству остальных людей мы настроены нейтрально, или даже способны создавать конфликты при определённых обстоятельствах с теми людьми, которых мы воспринимаем как угрозу. Люди создают различные альянсы по различным причинам. Основой отношений с другими может служить их общественное положение, сексуальная привлекательность, наличие денег или просто веселье (всё что может служить нашему выживанию и размножению и дарить нам положительные эмоции). Но отношения приходят и уходят и их продолжительность зависит от ценности тех людей, с которыми мы создаём альянсы и их способности сохранять свою ценность.

\RULE  Всякий раз когда человек чувствует что создание нового альянса принесёт ему больше пользы чем пребывание в старом, то он начинает пытаться рационализировать (искать причины для разрыва старых отношений с целью образования нового альянса) то что он чувствует. Люди могут выбирать, поддаваться своим побуждениям или нет. Часто они этого не делают. Но когда они это делают, то они начинают по-другому смотреть на отношения с человеком, с которым находились в старом альянсе. Они начинают вспоминать старые отношения, и они не кажутся им такими идеальными как раньше. Совершенно неожиданно, те недостатки которые были у человека и раньше не замечались (из за его высокой ценности) вдруг начинают быть видны совершенно отчётливо. Концентрируясь на них, разум находит оправдания для разрыва отношений. Когда люди поддерживают отношения между собой, то они концентрируются на позитивных эмоциях, которые они получают. Таким образом, оправдывается поддержание отношений. Но если создание нового альянса (с человеком более высокой ценности) является более целесообразным, тогда люди перестают вспоминать позитивные моменты отношений для того что бы оправдать разрыв. Если раньше внимание было сконцентрировано на тех положительных моментах, которые имели место в прошлом то после утраты партнёром ценности и желания вступить в новый альянс, человек начинает вспоминать лишь недостатки и разочарования в старом партнёре по альянсу. Это делает более мягким переход в новый союз. Некоторые даже сами начинают провоцировать конфликты для того что бы иметь эмоциональные оправдания для разрыва. Этот процесс имеет место не только как воспоминание о прошлом в негативных тонах но и как изменившиеся восприятие существующих отношений.

\RULE  Наш разум постоянно оценивает людей вокруг нас с точки зрения их ценности для нас, мы стараемся избегать людей с меньшей ценностью и взаимодействовать с теми людьми, которые имеют большую ценность для нас. С этим конечно можно поспорить, тем не менее многие вещи мы делаем без сознательного понимания почему и зачем мы это делаем. Если в вас есть ценность, то вы будете притягивать людей как магнит. Они будут думать что это чертовски классно, проводить своё время вместе с вами. Они будут воспринимать вас как более интересного, весёлого, клёвого человека. Человек с меньшей ценностью может делать те же самые вещи что и вы, но не получать такой же реакции на свои действия. В этом и заключается принципиальная разница между клёвым парнем и не клёвым. Это разница внутренней эмоциональной реакции других людей.

\RULE  Человек, имеющий большую ценность, как магнит притягивает к себе внимание других людей. (или же просто своим поведением метасообщает высокую ценность) Внимание, которое к вам проявляют окружающие люди является надёжным индикатором того, воспринимают ли они вас как человека с высокой ценностью. --- Если во время беседы вы вставляете своё замечание, то посмотрите проявят ли окружающие интерес к вашим словам, проигнорируют, или просто проявят вежливое внимание а затем игнорируя вас, спокойно вернуться к своей беседе. Будет ли каждый из окружающих внимательно слушать то, что вы сказали? Примут ли они во внимание то что вы сказали или просто продолжат беседу. Если вы предложите окружающим что либо изменить (тему беседы/текущее занятие /место пребывания) откликнутся ли они на ваше предложение? Можете ли вы контролировать энергетику в группе? Это относится к внешним индикаторам, которые могут вам сказать об вашей ценности для окружающих. Поразмыслив над этим, возможно вы найдете подтверждение моим словам в собственном прошлом опыте. (даже если тогда вы не понимали как это работает) Когда вы разговариваете с группой людей то в первую очередь вы смотрите как реагирует на ваши слова самый «авторитетный» из них или даже обращаетесь к нему непосредственно. Возможно вы делали это даже не задумываясь об этом. Поскольку люди имеют различную ценность для вас то и обращаться к ним вы то же будете по разному. Возможно у вас была ситуация когда ваше внимание привлекла симпатичная девочка в то время как ваш друг вам рассказывал о своих проблемах на работе. Несмотря на ваше желание не обидеть друга, вы тем не менее практически не слышали его поскольку вы наслаждались созерцанием её красоты и прикидывали варианты как её снять. Может быть даже у вас была ситуация когда вы выходили прогуляться вместе с вашим другом, но он выглядел (был одет) не очень хорошо. Пока вы общались друг с другом всё было великолепно. Но если вы заходили в какое-то место (заведение) где его вид не соответствовал данному заведению, то его ценность для вас могла измениться при изменении ситуации. Вы испытывали по отношению к нему различные чувства в различных ситуациях. В зависимости от той или иной ситуации мы испытываем потребность «настраиваться» на определённых людей в нашем окружении. Это не означает, что вы всегда будете делать это лишь означает что у вас будет возникать такая потребность.
\DEFINE{«Ситуационная ценность»}

Один и тот же человек в различных ситуациях может иметь различную ценность. Мы назвали это ситуационная ценность. Когда профессор читает вдохновенную лекцию в своём университете, то его статус в глазах окружающих изменится в сторону увеличения. То же самое происходит и со многими другими людьми: пианист на концерте, парень, устраивающий вечеринку у себя дома, ди-джей, который ставит музыку, которую все любят, знаменитость на которую каждый хочет хоть мельком взглянуть, бармен на работе. Окружающие всегда будут хотеть получить что то от этих людей. Все эти парни находятся в выгодном положении, поскольку при определённых условиях их ценность в глазах окружающих возрастает. Возрастает что то, что могут почувствовать как они сами так и окружающие. В случае с профессором заметьте как студенты поднимают руки, спрашивают его о чем то или просто советуются их голоса звучат не так громко и властно как голос профессора. При взгляде на профессора они часто отводят глаза в сторону, они не так шутят, их слова не имеют особого значения по сравнению со словами профессора. То же самое касается парня, устраивающего вечеринку. Он подходит к группе людей, представляется и гости с ним просто вежливы. Но когда она обнаружат, что он является хозяином вечеринки, то их отношение к нему изменится. Каждому захочется с ним поближе познакомиться, и они будут подчёркнуто вежливо вести себя по отношению к нему. Их голоса вдруг внезапно зазвучат более тихо и подчинённо по отношению к нему, они повернутся к нему всем телом и будут внимательно слушать что он говорит. Все эти парни обладают высокой ситуативной ценностью. Их ценность усиливается средой (ситуацией) в которой они находятся. Они более уверенны в себе, чувствуют себя комфортно и их не особенно заботит что о них подумают другие. Все признают их ценность. Но что произойдёт, если поместим хозяина вечеринки или профессора в другую ситуацию. Если например мы приведём их на дискотеку. где ситуативной ценностью обладает бармен и владелец заведения. Их ситуативная ценность упадёт и они уже не будут чувствовать того отношения к себе со стороны других как в предыдущих ситуациях. Они уже не будут чувствовать то жё самой уверенности, Именно по этой причине большинство парней которые имеют успех у женщин в одном месте, уже не имеют его в новой социальной среде.
\DEFINE{«Ситуационная уверенность»}

«Ситуативная уверенность» это уверенность которая появляется при ожидании одобрения со стороны окружающих --- когда вы делаете что-то в особенной ситуации и ваша высокая социальная ценность является гарантией того что остальные одобрят ваше поведение. Что бы проиллюстрировать ситуационную уверенность давайте представим себе парня который недоволен своим телом. А теперь давайте представим себе что этот парень пошёл в бассейн полный детей на чьё мнение ему в общем плевать. Будет он беспокоится о том, что подумают о нём дети? А теперь давайте кое-что изменим. Представьте себе, что в бассейне вместо детей будут плавать люди, принятие и одобрение которых для него очень важно. Поскольку парень нуждается в одобрении этих людей, хочет быть принятым ими то в его внутреннем состоянии произойдёт сильный сдвиг. Он хочет понравиться этим людям и он начинает думать о том, как вести себя, что бы хорошо выглядеть в их глазах. Его уверенность в себе исчезнет, и он начнёт проявлять признаки беспокойства. Возвращаясь к нашим парням с сильной ситуационной уверенностью, что у них общего? Все они имеют высокий уровень социального доказательства собственной ценности
\DEFINE{«Социальное доказательство»}

«Социальное доказательство» это внешняя, видимая демонстрация высокой социальной ценности или участия в социальных альянсах. Например, если вы видите парня который окружён людьми и эти люди внимательно слушают всё что он скажет, ваш разум будет воспринимать его как имеющего высокую социальную ценность, поскольку вы видите группу людей которые реагируют на него.
\DEFINE{«Реагирующий»}

Быть «реагирующим» по отношению к кому то, означает что кто либо может повлиять на ваше душевное равновесие, это означает что другой человек может повлиять на то, то что вы будете думать, и что будете чувствовать. Быть реагирующим не означает какого то определённого поведения, это любое поведение, которое происходит от реагирующего состояния ума. Когда люди реагируют на вас, это может происходить совершенно разными способами. Они будут проявлять внимание к вам, подстраиваться под ваше поведение, они могут хотеть одобрения своего поведения и подтверждения собственной ценности с вашей стороны. И они могут стать эмоционально «опущенными» если вы не будете давать им этого. Обычно, они будут говорить более слабым и менее уверенным в себе голосом, чем они говорили до вашего появления. Они будут больше смеяться над вашими шутками, чем над шутками других людей. Они будут беспокоиться о том, что бы не вторгнуться в ваше личное пространство или не отнять у вас много времени. Прежде чем что либо сказать, они будут думать, достаточно ли это важно и интересно, что бы привлечь ваше внимание. Очевидно, что когда кто либо реагирует на вас, он ставит вас в позицию, где вы обладаете более высокой социальной ценностью. Итак. как мы можем не быть реагирующими?
\DEFINE{«Не реагирующий»}

Быть «не реагирующим» означает что где глубоко внутри, вы понимаете что ваши действия не зависят от реакции на вас другого человека. Это не означает, вообще ничего не делать, что можно было бы расценить как реакцию. Потому что если вы стремитесь «вообще никак не реагировать» то это то же может быть «реагирующим» поведением. Потому что, если парень хочет что то сделать. но не делает этого, из за того что он боится осуждения со стороны других --- то это реакция. Быть «не реагирующим» означает думать и действовать, выражать свою личность вне зависимости от того, что другие люди могут подумать о вас. Если даже парень имеет какой либо физический недостаток или внешне непривлекателен люди могут реагировать на него при определённых условиях, сам он при этом может оставаться «не реагирующим». Именно подобное «не реагирующее» поведение привлекает женщин. Разница между ним и богатыми, внешне привлекательными парнями заключается лишь в том, что у последние используют внешность и деньги как основание для внутренней уверенности в себе. Несмотря на внешность и деньги большинство женщин не будут привлекать парни, которые не смогут продемонстрировать перед ними свою личность. По это причине мы постоянно можем наблюдать абсурдные ситуации. Например, управляющий рестораном, который ездит на дешёвом автомобиле и живёт со своими родителями может трахать весь женский персонал ресторана. Несмотря на то, что в целом его социальный статус невысок, в своём ресторане он обладает высоким статусом и как следствие высокой ценностью. На работе, сотрудники ресторана будут реагировать на него. Их эмоциональное состояние будет находиться в зависимости от того, как он оценит их работу. В его присутствии они будут чувствовать и вести себя немного по другому, чем когда его нет. Они будут внимательно слушать то, что он им рассказывает, смеяться над его шутками и выполнять его указания. В своём ресторане, в своей среде он будет иметь «социальное доказательство», его уверенность в себе будет усиливаться от того, что он знает что все его действия будут одобрены со стороны окружающих. Вследствие этого, он будет вести себя более уверенно и женщины начнут «западать» на него.

\RULE  Фундаментальный принцип влечения заключается в том, что в любом социальном взаимодействии всегда есть человек который заставляет реагировать на себя, сам при этом практически не реагируя. Это основной принцип влечения.
\DEFINE{«Влечение»}

Человек, на которого реагирует большинство окружающих, сам при этом не реагируя, имеет самую высокую ценность и притягивает к себе внимание. На бессознательном уровне, женщины чувствуют это и соответственно реагируют на него, момент за моментом. Женщины откликаются на подобное поведение на эмоциональном уровне. Это притягивает их как магнит. Они чувствуют влечение по отношению к этому человеку, вне зависимости от каких то поверхностных вещей. Это именно то, как работает влечение. Всё это мы можем наблюдать постоянно: привлекательные плохие парни, нахальные парни, загадочные парни с особой аурой. Своим поведением они демонстрируют безразличие, и не «парятся» что о них подумают другие. Что является общим у всех этих парней, это то, что они не реагируют на других людей, а наоборот, есть что то в их личностях, что заставляет других людей реагировать на них. Несмотря на социальный статус, они взаимодействуют с женщинами на эмоциональном уровне. Для них не необходимости соответствовать поверхностным социальным стандартам общества, для того что бы чувствовать уверенность в себе. Они делают это базируясь исключительно на собственной личности. Вот что является базовой ценностью.

Что бы лучше понять что является базовой ценностью, вернёмся к нашему парню в бассейне. Представьте на его месте другого парня, который ещё менее привлекателен, чем наш неуверенный в себе друг. В отличие от нашего застенчивого друга, этот новый парень никак не реагирует на смену обстановки в бассейне, а просто купается и шутит с каждым. По нему видно, что он просто весело проводит время. Этот другой парень будет казаться более клёвым, и люди будут относиться к нему соответственно. Что дало ему уверенность вести себя подобным образом? Возможно, он является хозяином вечеринки. Может быть он просто со своими друзьями. Может быть он написал хорошую книгу (бестселлер). Может быть он просто знает что он самый весёлый и лучший собеседник из всех здесь присутствующих. Может быть он хорошо танцует. Может быть он известный модельер. Может быть он прочёл Кама-сутру и выучил все сексуальные техники. Может быть у него 5 подружек и они ревнуют его друг к другу. Может быть он президент студенческого братства. Или\ldots Может быть\ldots Он просто парень и где-то глубоко внутри он чувствует уверенность в себе и ожидает принятия своего поведения со стороны других вне зависимости от ситуации. Он просто клёвый парень. Некоторые назовут его натуралом.
\DEFINE{«Базовая ценность»}

«Базовая ценность» это ценность, которая всегда с вами, потому что она основана на вашей личности. Она происходит от внутреннего чувства уверенности в себе, она не зависит от того, как вас воспринимают окружающие в данный момент, это уверенность происходит от понимания кто вы есть, и не нуждается ни в каком внешнем укреплении. Именно эта уверенность в себе, позволит вам вести себя таким образом, что передать высокую социальную ценность. Вне зависимости от внешних обстоятельств и ситуации у вас всегда будет ценность и люди будут воспринимать вас соответственно и реагировать на вас. Многие парни тратят целую жизнь, для того, что бы приобрести что то внешнее, что даст им уверенность. Они могут много трудится для того что бы приобрести совершенную работу, совершенное тело или совершенных женщин. Но в конечном итоге, они просто строят стены, которые их ограничивают. Они становятся пленниками своих собственных умов. Для того, что бы быть счастливым, вам потребуется более глубокое понимание того, кто вы есть.

\chapter{Любовь}

Представьте себе парня, у которого проблемы с женщинами. Возможно, его проблемы начались, когда он почувствовал, что может понравиться женщине. Он начинает фантазировать о ней, представляя что у него есть с ней определённая связь, он воображает какие то события, которых в реальности никогда не существовало. Когда он слушает романтические песни о любви, он чувствует сексуальное напряжение, и его воображение рисует ему картины, где он вместе с этой девушкой. Но неизбежно, он всё таки осознаёт что это всего лишь фантазии, что ничего этого в реальности нет\ldots Это горькая пилюля, которую ему придётся проглотить. А теперь представьте себе парня, у которого уже есть подружка. Он идеализирует её и их отношения, и когда он обнаруживает что девушка его обманывает, это становится ударом для него. Он замечает лишь положительные качества своей подружки, постоянно думая о том классном времени, которое они проводили вместе. Он думает о том опыте, о тех переживаниях, которые, сделали их отношения такими сильными. Он думает о тех вещах, которые на его взгляд, укрепляли их любовь. Он вспоминает места, где они познакомились, где любили гулять, где первый раз занимались сексом. Ведь у них есть «их вещи», что то, что их объединяет, что их связывает вместе, чего не имеют другие пары. То, что они имеют, они могут получить только друг от друга, никто больше не способен так дополнять друг-друга, они незаменимы друг для друга, они могут наслаждаться спокойствием и своей любовью без страха потерять друг-друга. И всё это делает их любовь, их отношения такими сильными, такими стабильными, и они «НАВСЕГДА» останутся вместе. И что забавно, когда отношениям приходит конец, все эти чувства для кого-то остаются в реальности. Когда девушка уходит от парня, его реальность рушится. Он страдает, он хочет что бы девушка осознала ценность их общей реальности, её уникальность, незаменимость\ldots но девушка уходит. Та девушка, которая была рядом с ним, больше не существует. Она существовала лишь в воображении парня. То лицо, которое она ему показывала, было лишь одним из множества её лиц.

Парень никак не может понять, что не только девушка, но и он сам, имеет множество лиц. Мы все имеем множество лиц, и выбор какое лицо демонстрировать, будет зависеть от той ценности, которую представляет для нас человек с которым мы общаемся. Лицо, которое мы демонстрируем бродяге, который просит у нас мелочь, будет отличаться от лица, с помощью которого мы будем общаться, когда разговариваем со своей матерью. Парень думает «Подождите, она просто запуталась, она не понимает, а как же насчёт наших общих «вещей»? Она ведь не сможет получить от него то, что давал ей я. Никто не способен любить её так как я. Она трахалась с этим парнем в первый же вечер после знакомства. Она всегда говорила что она не шлюха\ldots мне понадобилось 3 месяца ухаживания. И она трахалась с этим парнем на «нашем» диване, это же особенный диван. там мы с ней впервые занимались сексом. Почему же эта шлюха не помнит этого. Как же она может поступать так?» Он начинает рационализировать ход событий, думая что его подружка просто запуталась, что она не понимает что она делает. Но он не слабак, он просто так не сдастся, он не бросит бороться, он обязательно вернёт её назад, когда закончится её помутнение рассудка. Ведь он и она это единое целое\ldots а этот новый её парень он просто посторонний. Конечно, расставание с старым парнем может быть для неё не очень приятным. он умолял её вернутся к нему назад\ldots ей даже пришлось показать ему одно из своих не очень приятных лиц.

Однако, как только за ним закрылась дверь, она надела одно из своих улыбающихся лиц и пошла на свидание со своим новым парнем. Она просто идёт и наслаждается жизнью, ни о чём не задумываясь. Парень же продолжает думать об этой девушке и страдает, что её больше нет в его жизни. И парень никогда не признается сам себе в том, что те неприятные черты характера которые есть у неё, есть и у него. Что если бы обстоятельства складывались по другому, то вероятно он поступил бы точно так же, как и его бывшая подружка. Например, если бы ему самому в определённый момент наскучили их отношения или он бы встретил девушку, которая понравилась бы ему больше. Однако в данный момент он ни чём из этого не задумывается, он просто страдает и чувствует пустоту в своей жизни. Со временем он решает, что ему пришла пора измениться.

Он начинает попытки улучшить себя с помощью каких то поверхностных вещей, таких как: профессиональная деятельность, приобретение собственности, приобретение модной одежды, красивого автомобиля и т. д. Если парень будет достаточно настойчив, то в определённый момент времени у него всё это появится. Однако. он по прежнему одинок. ничего из этих внешних вещей не даст ему подружку. Под влиянием социального программирования часто мы идеализируем сам термин «любовь» вместо того что бы дать чёткое Определение данному явлению. Писатели и философы много спорят о том, что же в действительности означает данный термин и до сих пор они не пришли к единому мнению. В некоторых обществах существуют даже разные термины для объяснения разных видов любви. Многие люди приписывают любви некие сверхъестественные свойства. Они например могут верить в то, что у каждого человека есть только одна вторая половинка, что любовь не приходит дважды, она приходит один раз и на всю жизнь и т. д. Они так же могут верить в то, что любовь сама тебя найдёт в определённое время, надо только ждать и верить и судьба сведёт тебя со второй половинкой. Это «просто произойдёт» когда наступит правильное время. Следуя своей вере во «всё преодолевающую» силу любви люди «накручивают» себя и собственные чувства, они считают что они просто следуют зову своего сердца. Вспомните свою последнюю любовь. Как вы решили, что это любовь? По влечению? По душевной связи? По физическому желанию? По вашему сходству? По взаимной эмоциональной связи? Или это было сочетание всего вместе? Любовь ли связывает пожилую пару, сидящую на крыльце и привыкшую к многолетней рутине? Любовь ли между двумя подростками, закрывшимися на заднем сиденье машины и поспешно достающими презерватив? Любовь ли связывает молодоженов, дающих клятву?

Про любовь часто говорят что это своего рода самогипноз, это такое прекрасное помутнение рассудка, которое заставляет нас поступать как сумасшедший. Любовь не вызывается другим человеком, мы сами создаём её внутри себя. Наши мысли снова и снова возвращают нас к определённому человеку, вследствие этого наше восприятие этого человека меняется. Внезапно, все кажется таким простым. Это любовь. Она охватывает нас, и тело отвечает физиологическими реакциями, усиливая эмоциональное влечение, пока мы полностью не попадаем в его власть. Для некоторых людей любовь --- возможность найти спутника жизни. Это возможность познать и понять другого человека и другой человек сделает то же самое для них в ответ. Здоровые любовные отношения между двумя людьми, способствуют физическому и духовному росту каждого из них. Любовь может быть одним из самых важных и приятных переживаний на протяжении всей жизни человека. Но это зависит от того, готов ли человек испытать любовь. В некоторых случаях любовь может разрушать человека. Для некоторых людей любовь это самообман, это способ не замечать своих собственных недостатков. Люди часто рационализируют свои сильные эмоциональные реакции по отношению к другому человеку, они ищут им оправдания и они их находят --- «это любовь». Они становятся одержимыми человеком, которому они не интересны(неразделённая любовь) и они объясняют сами себе своё поведение тем, что это именно тот человек, которого им не хватает для полного счастья. Они отчаянно нуждаются в другом человеке, в его присутствии, в его одобрении. Часто люди могут сами разрывать отношения со своими «любимыми». С течением времени, слепая страсть которая присутствовала на начальных этапах отношений, непременно уходит, и они думают что «любовь прошла» и сами уходят от партнёра. Позднее, они могут не встретить «нужного» партнёра, и они могут начать сожалеть о своём поступке, думая что в лице прежнего партнёра они потеряли любовь всей своей жизни. Иногда встречаются люди, готовые ответить взаимностью любому, кто первый их полюбит. Они страстно желают «любви», для них это навязчивая идея. Другие же наоборот бегут от любви. Они боятся влюбляться и сами себе запрещают это делать, возможно, они пострадали в прошлом и сейчас просто бояться, что любовь принесёт им страдания. Как мы уже говорили ранее, в определённых ситуациях, у человека может возрастать чувство ожидания одобрения собственного поведения со стороны окружающих. (Ситуативная уверенность) Точно так же, человек может чувствовать что его собственная ценность увеличивается, если он находится в одной группе с определённым человеком с высокой ценностью. Когда ценность человека, его личность, его ожидание принятия и одобрения собственного поведение со стороны окружающих тесно связано с другим человеком, то он будет зависеть от этого другого человека. От этого человека будет зависеть его эмоциональное состояние и хорошее настроение.

И поскольку он зависим, он становится реагирующим в их отношениях. Он боится потерять своего партнёра, боится испытать боль, вместо того что бы получать удовольствие от отношений. Когда происходит такой видимый сдвиг в поведении одного из партнёров, то он становится менее привлекательным для другого партнёра, чувства другого партнёра начинают угасать. Так, может, только тот, кому не нужно общественное одобрение для того, чтобы чувствовать себя отлично, может по настоящему любить. Может ли быть так, что лишь не нуждаясь в любви, мы обретаем ее?
\chapter{Личность}

\textit{(Примечание Sh. Тайлер в своих рассуждениях следует философии Экхарта Толля, в частности разделяется «Я» и «разум». Для лучшего понимания рекомендую ознакомиться с работами Экхарта).}

Если вы едите на лифте, как вы узнаете, на каком этаже вы находитесь? Вы можете определить с точностью свое местоположение только в том случае, если это стеклянный лифт, тогда вы сможете просто посмотреть сквозь стекло. Если же это непрозрачный лифт и индикаторы этажей по какой то причине не работают, то до тех пор, пока двери лифта не откроются, вам трудно будет определить своё местоположение в пространстве. Это образное сравнение подводит нас к представлению о личности.
\DEFINE{«Личность»}

Личность --- это представление о самом себе, которое отличает вас от других. В процессе общения с другими людьми к нам приходит осознание того, что мы от них отличаемся, что мы другие. Мы начинаем осознавать свою уникальность, неповторимость. Через личность мы взаимодействуем с окружающим миром. Совокупность ваших представлений о том, кем вы являетесь и кем не являетесь, чего вы достойны, а чего нет, каково ваше положение в обществе и место в социальной иерархии, на что у вас есть право, а на что нет, в конечном счёте будет определять ваше поведение в социуме. Как существа социальные, все мы имеем способности к доминированию и созданию определённой ауры в нашем социальном окружении. Но как правило, именно наши собственные убеждения, собственные представления о самих себе и ограничивают нас. Они определяют, сколько именно наших способностей нам позволено использовать.

Личность может как давать вам силу, так и ограничивать вас. Если у вас есть уверенность в определённой ситуации и вы знаете, что согласно вашим представлениям о самом себе, вы этого достойны, то вы будете вести себя соответствующе. В других же ситуациях, ваша личность, ваше представление о самом себе может вас удерживать от поступков, которые могут принести вам желаемое, потому что вы думаете, что вы этого не достойны. Часто вы можете увидеть как парень, которому вы предлагаете выйти за рамки собственных ограничений, за рамки представлений о собственной личности, будет отчаянно сопротивляться, будет придумывать десятки оправданий для себя, типа «Я сейчас занят, давай потом», «Нет, это не для меня». Если парень будет думать, что в нём недостаточно ценности для совершения какого то действия или приобретения чего-либо, что он этого не заслуживает, то его разум будет его жёстко блокировать, отвергать и саботировать любую его попытку получить желаемое. Личность, это одно из тех понятий, в которых вам необходимо разобраться самому. Что бы быть успешным с женщинами вам необходимо пойти вглубь себя, на очень глубокий личностный уровень. Разница между парнем, с которым женщина разговаривает до тех пор, пока он её развлекает, и парнем, который может привести женщину в свой мир, продемонстрировав свою личность, заключается в том, есть ли у парня чувство что он имеет право на это. Вот почему женщина не будет заниматься с вами сексом за то, что вы делаете. Она будет заниматься с вами сексом за то, кто вы есть.

Когда женщина впервые видит мужчину, она будет его оценивать и откладывать секс до того момента, пока она не сделает окончательное заключение о его личности. Она будет пытаться понять тот ли он, за кого он пытается себя выдавать, конгруэнтен ли он, есть ли у него ценность, индивидуальность и правда ли он думает, что у него есть право заняться с ней сексом.

Если она будет очарована вашей личностью, и будет уверена в том, что вы действительно тот, за кого себя выдаёте, что вы действуете в соответствии с собственными представлениями о самом себе, то она захочет секса с вами. Однако, если она почувствует малейшую неконгруэнтность в вашем поведении, она моментально потеряет влечение и уйдёт от вас. Итак, если вы действительно хотите привлекать женщин, а не просто их развлекать, то вы должны развивать свои социальные навыки и действовать в соответствии с собственными представлениями о самих себе, вам необходимо быть полностью конгруэнтными.

\RULE  Вы можете присвоить себе любую оценку вашей ценности, от самой высокой до самой низкой. Та оценка, которую вы сами себе поставили, будет определять насколько у вас есть право демонстрировать доминирующее поведение в вашем социальном окружении. Ваша собственная самооценка будет так же определять, как вы будете реагировать, когда получаете ответную реакцию на ваше поведение, как положительную, так и отрицательную.

Осознаёте вы это или нет, но в вашем уме есть представление о том, насколько большого успеха в жизни вы заслуживаете. Это представление, есть ли у вас право на успех у женщин, является частью вашей личности. Иметь право на обладание чем либо (быть достойным чего либо) это многоуровневое понятие, чувство того, что вы имеете право на успех у красивых женщин, тесно связано с вашим ощущением есть ли у вас право на успех в жизни в целом. Многие люди заметили, что их стремление к обладанию красивыми женщинами, изучение пикапа, привело их на путь саморазвития. Они стали лучше, они стали успешнее в жизни в целом. Уверенность, которую они развили с целью трахать красивых женщин, вдруг неожиданно положительным образом сказалась и на других областях их жизни. Парень, который чувствует уверенность в общении с красивыми женщинами, будет думать что у него есть право завязывать разговор со случайными собеседниками, говорить то что он думает, быть центром внимания окружающих, создавать нужную ему ауру в группе людей, устанавливать правила «что является клёвым», и свободно выражать свою личность.

Это не значит, что это обязательно сделает его неприятным собеседником, кто пытается «задавить» разговором окружающих. Скорее, это придаст ему уверенности в себе, как человеку с высокой ценностью, и он будет нравиться людям. Как музыкант, который выступает на сцене и играет свою музыку, так и наш парень будет нравиться каждому, потому что он предлагает ценность.

\RULE  Парень с высокой ценностью и сильным чувством того, кто он есть, будет успешен как в общении с женщинами, так и с людьми в целом. Для большинства людей, будет поразительно видеть, как быстро он добивается успеха в общении с женщинами. Как положительно реагируют на него женщины. Всё это идёт от уверенности парня. Когда женщина встречает мужчину, который полностью уверен в себе при общении с ней, шутит, рассказывает истории, в целом направляет беседу, она будет реагировать на него непроизвольно.

Но если парень чувствует себя некомфортно делая эти вещи, женщина обычно отшивает его. Вот почему, реакция на вас женщин напрямую зависит от вашей ценности, от той оценки, которую вы сами себе поставили. А это в свою очередь напрямую имеет отношение к вашей личности. Итак, почему ваша личность определяет реакцию на вас женщин? Это очень важный момент, ваша личность, ваше представление о самом себе, образуется через ваше взаимодействие с окружающим миром, через реакцию на вас окружающих, через социальную обратную связь.

Предположим, что вы попытаетесь сделать что то, что предполагает более высокую ценность, чем, как вы думаете, у вас есть. Вы делаете эти вещи, но в глубине души вы сомневаетесь в себе (сомневаетесь в том, что у вас есть на это право, что вы этого достойны, что вы сможете это сделать). В конечном счёте ваши сомнения берут верх, и возможно это происходит под влиянием негативной обратной связи от окружающих (куда ты полез нахуй? Сиди, прижми свою жопу), и у вас ничего не получается. Вы думаете «Ну вот, я попробовал и у меня не получилось, не стоит даже и пытаться в будущем делать что то подобное». Ваша личность, ваше представление о собственной ценности, а так же негативная обратная связь со стороны окружающих уничтожили вашу попытку «подняться» Как существа социальные, мы постоянно получаем обратную связь от нашего окружения, в ответ на собственные действия. Общество влияет на нас таким образом, и мы реагируем на это, изменяя собственную личность, собственное представление о самих себе. Иногда в положительную сторону, а иногда в отрицательную.
\DEFINE{«Социальная обратная связь» или «Пинг»}

Наш разум постоянно обменивается информацией с окружающими нас людьми. Это означает что наш разум вовлечен в процесс обмена знаниями с другими, и получение подтверждения правильности наших знаний, путём сравнения с информацией полученной от других.

«Пинг» --- это постоянный и едва заметный процесс, который может проходить как через скрытые метасообщения так и через весьма очевидную вербальную реакцию на вас окружающих людей. С помощью пинга мы постоянно пересматриваем наши представления о самих себе и о мире в целом. Осознаёте вы это или нет, но ваше восприятие окружающего мира очень субъективно и выстроено под влиянием пингов со стороны вашего окружения. Вот почему, когда человек едет в другую страну, то с ним может случится то, что называется «культурный шок». Человек оторван от привычного социального окружения, он не получает пингов. Если человек проживёт в состоянии культурного шока продолжительное время, то это даже может сказаться на его психическом здоровье. Мы можем представить себе пинг (процесс получения обратной социальной связи) как зеркало, в которое мы смотримся для того, что бы получить одобрение со стороны других людей. Социальная обратная связь даёт нам понимание того, что такое нормальное поведение, каков наш статус в обществе, наша ценность и как мы должны вести себя в соответствии с нашим статусом. В процессе нашего взаимодействия с окружающим миром, у нас появляется чёткое понимание того, как поведение человека отражает его статус, его социальные роли, его ценность и положение в группе. Как на сознательном, так и на подсознательном уровне мы прекрасно знаем как выглядит парень с высоким статусом (ценностью) и как с низким. Мы всё время наблюдаем людей с самыми разными статусами. Мы всегда видим разницу в статусе и всегда эмоционально реагируем на неё.

\RULE  В процессе пинга вы постоянно получаете обратную социальную связь которая говорит вам растёт или уменьшается ваш статус. Ваш разум постоянно чувствует социальное давление и необходимость приспосабливать ваше поведение к изменяющейся внешней ситуации, в зависимости от вашей ценности в данной ситуации. Как существа социальные, мы стремимся адаптироваться при изменении внешних обстоятельств. Как только наступает момент, когда наше текущее поведение не соответствует нашему статусу, наш разум распознаёт это несоответствие и даёт нам команду адаптировать наше поведение к нашему статусу. Вот почему популярная девушка из сельской местности, приехав большой город, вдруг полностью меняется. Она здесь никого не знает, вокруг полно красивых женщин и её поведение меняется. Она уже не так уверена в себе. Изменилось её окружение. Однако следует заметить, что разные люди реагируют на социальное давление по разному. Некоторые люди делают это очень явно, другие же не реагируют так сильно. Их реакция зависит от представлении в их умах о том, на что они имеют право.
\DEFINE{«Основание. Почему я имею право на\ldots»}

Для того, что бы вам вести себя как мужчина с высокой ценностью, а значит и привлекательно в глазах женщин (как мы уже разбирали мы все знаем, как это делать) вашему уму необходима какая то зацепка, какое то основание, почему вы отвечаете определённым требованиям. В зависимости от социального программирования это могут быть следующие основания.

\begin{enumerate}
\item Отвечать поверхностным социальным стандартам: установленным через социальное программирование (хорошая внешность, шмотки, деньги и т. д)
\item Альянсы с другими людьми: Если ваши друзья это высокостатусные люди, или у вас подружка-супермодель и т. д.
\item Способности (умения). Когда у вас есть что то, что другие люди хотят получить от вас (доступ к чему то редкому, что трудно достать; какие то знания и умения, которые люди хотят получить от вас; социальные навыки, которые позволяют вам метко шутить и развлекать людей)
\item Социальные роли. Иногда, обстоятельства ставят нас в такое положение, когда мы играем некую социальную роль, которая может временно увеличить наш статус (ценность) (профессор в институтской аудитории, крутой специалист в своей области на работе, бармен в ресторане, быть окружённым людьми с заведомо более низкой ценностью)
\item Личность. Когда в вас есть внутреннее убеждение, что у вас есть право вести себя как мужчина с высокой ценностью, просто потому что вы уникальная личность. Всякий раз, когда ваш разум видит что у вас есть основание, то он говорит вам «Сейчас ты можешь вести себя как мужчина с высокой ценностью, и через тонкие поведенческие моменты доводить это до сознания окружающих» Ваш разум даёт вам доступ к соответствующему эмоциональному состоянию. Это часто называют «быть в ресурсе».
\end{enumerate}

«Быть в ресурсе» это очень сильное переживание. Это просто фантастически улучшит ваш съём. Когда вы в ресурсе, мир будет у ваших ног. Ваш разум затихнет, и у вас всё будет получаться. Ваш юмор будет сражать наповал, истории будут казаться исключительно смешными и интересными, и люди будут следовать за вами, как за лидером. Вы будете как магнит притягивать к себе внимание окружающих.

Существуют несколько различных способов объяснения почему «быть в ресурсе» говорит о высокой ценности. Одним из наиболее логичных выглядит следующее объяснение: состояние это отражение личности, а личность в свою очередь говорит о социальной ценности. Поэтому, тот, кто «в ресурсе», имеет наиболее высокую социальную ценность.

Кроме того, внутренняя ценность может выражаться только аутентично. Ваша личность является отражением вашего жизненного опыта. Вам следует это понять. Например, то как вы шутите, является отражением вашего мировоззрения.

Когда вы в ресурсе, вы свободно выражаете свою личность, без какого либо страха. Вы не пытаетесь кого то впечатлить, убедить или приспособится к кому либо. Вы более реальны. Люди ценят это. Когда вы в ресурсе, вы более аутентичны. Вам безразлично как окружающие реагируют на вас и что они о вас думают. Вы просто выражаете собственную личность и делитесь своей энергетикой с другими. Они чувствуют что вы тот, кто предлагает ценность, потому что вам не нужна их реакция на себя.

Все эти рассуждения ведут нас к одному выводу: Быть в ресурсе означает обладать высокой социальной ценностью.
\DEFINE{«Быть принятым (получать одобрение) и «Состояние»}

Быть принятым или получать одобрение со стороны окружающих (нравится окружающим) это своего рода индикатор, который сообщает вам о том, растёт или снижается ваша ценность в группе. Когда окружающие принимают и одобряют наше поведение, то мы чувствуем удовлетворение (испытываем положительные эмоции), когда окружающие не дают нам одобрения (не реагируют положительно на наше поведение, не принимают его) то мы в ответ чувствуем эмоциональный дискомфорт, который может парализовать наше разумное поведение. В древнем племени, если ваше поведение не нравилось соплеменникам, это могло привести к физической расправе над вами. В современном обществе, чаще всего, убивать вас никто не собирается. Однако, наши инстинкты говорят нам том, что наше выживание зависит от того, нравится ли окружающим наше поведение.

Мы постоянно испытываем потребность нравиться окружающим, быть принятым ими. Вы можете переживать это на эмоциональном уровне. Ваше поведение одобряют --- и вы радуетесь, отвергают --- и вы печалитесь. Мы постоянно испытываем эти перепады в нашем настроении и часто не придаём им большого значения, пока сдвиг в нашем эмоциональном состоянии не будет слишком очевидным для нас самих. Это как шум в ночном клубе, он присутствует постоянно, мы перестаём его слышать и обращать на него внимание. И если только шум вдруг станет значительно сильнее или наоборот станет слишком тихо, вы это немедленно заметите.

Перепады в нашем эмоциональном состоянии зависят от того, есть ли у вас основания для определённого поведения. Мы уже говорили ранее о 5 основаниях, дающих вам уверенность в себе, а значит право вести себя как мужчина с высокой ценностью.

Давайте вернёмся к ним ещё раз.

\subsection{Основание 1. Поверхностные социальные стандарты}

Вспомните тот момент, когда несколько лет назад вы купили себе новую модную вещь или сделали классную стрижку. Вы подумали про себя, «Вот сейчас то, я выгляжу по настоящему клёво». Неожиданно, вы стали замечать, что девушки начали смотреть на вас, и окружающие люди, кажется, начали относиться к вам с большим уважением. Ваша покупка заставили вас пережить приятные эмоции, вы даже стали как то по-другому себя чувствовать.

В то время, вероятно, вы подумали, что это ваша новая куртка или новая стрижка вызывает такую положительную реакцию окружающих. Но давайте рассмотрим это под другим углом. С того момента уже прошло несколько лет, мода изменилась. То что было модно несколько лет назад, сейчас может быть и не модно. Если бы вы надели сейчас ту самую куртку, которую вы носили несколько лет назад, почувствовали бы вы ту же самую уверенность, которую чувствовали раньше? Скорее всего нет.

Окружающие вас люди реагируют не на вашу одежду, а на ваше чувство уверенности в себе, которое даёт вам одежда. Если вы следуете социальному программированию и верите в то, что привлекательная внешность даёт вам основание для уверенного поведения и у вас есть привлекательная внешность, то вы будете вести себя уверенно. Это даёт вам доступ в ресурсное состояние». Вы ведёте себя уверенно и ожидаете положительной реакции на себя окружающих, и это становится само исполняющимся пророчеством. Представьте себе инопланетянина с другой планеты, который прилетел на землю. Это инопланетянин выглядит как обычный человек, только он 1,5 метра ростом, лысый и «на мели». На его планете низкий рост считается преимуществом, поскольку пищи там мало, а за счёт небольшого тела ему как раз необходимо потреблять небольшое количество пищи. Лысая голова это признак зрелости, а зрелость на его планете это привлекательное качество мужчины. Более того, согласно системе ценностей его планеты, путешествовать и набираться опыта гораздо важнее, чем сидеть весь день в офисе, зарабатывая больше денег, чем можно потратить. Фактически, последнее считается врожденным недостатком. По этим причинам (которые будут являться абсурдными для большинства людей на земле) а именно, потому что он маленький, лысый и бедный, он считает что он заслуживает только всего самого лучшего. Прибыв на землю и видя, что его единственные соперники --- высокие, волосатые, хвастающиеся деньгами парни, он немедленно входит в ресурс, думая «Это лучший день моей жизни». Таким образом мы приходим к следующему выводу. «Если вы верите в то, что поверхностные социальные стандарты «работают» и благодаря им вы можете привлечь женщину, то если вы им отвечаете это даёт вам уверенность в себе, и право вести себя как мужчина с высокой ценностью»

\subsection{Основание 2. Альянсы}

Представьте себе парня, у которого никогда не было девушки. Его поведение выглядит неуклюжим в глазах окружающих, он неуверен в себе, он всё время беспокоится о чём то. Но если несмотря на это, какая то девушка найдёт его привлекательным и они начнут встречаться, то поведение нашего парня может измениться. Неожиданно он перестанет быть неуклюжим и неуверенным в себе. Он будет более расслабленным и даже станет веселиться и «отвисать» вместе со всеми. Прежде он ходил вокруг и думал, что он не нравится окружающим, это его сильно беспокоило. Но как только он начал встречаться с девушкой (возник альянс с его участием) он начал чувствовать себя более уверенно. Окружающие девушки заметили произошедшие в нём перемены и начали показывать интерес к нему. Отчего его уверенность стала ещё больше. Его уверенность, которую он получил начав встречаться с девушкой, привлекла других девушек, отчего уверенность выросла ещё больше. Это замкнутый круг. Он нарастает как снежный ком, причём это может происходить в обоих направлениях. Возможно, вы когда-нибудь присутствовали на вечеринке, где вы никого не знали. В начале вечера вы испытывали какое то одиночество, чувство что вы отделены от всего происходящего. Но познакомившись с несколькими людьми, и получив хорошую реакцию на себя с их стороны, вы почувствовали какую то уверенность. Вам вдруг стало легко завязывать контакт и с остальными людьми. В начале вечера вы были напряжены и всё время думали что бы такое сказать, что бы это было «к месту». В конце вечера вы вообще не думали, просто разговаривали, и ваши слова попадали точно в цель. Вы развлекались, шутили, рассказывали весёлые истории, снимали девушек и вынравились окружающим. Ваша уверенность, ваше чувство что у вас есть право на подобное поведение, пришли к тогда, когда вы познакомились с первыми несколькими людьми. Знакомство с ними (вступление в альянс) прибавило вам ценности и как результат, выросла ваша уверенность.

\subsection{Основание 3. Способности (умения)}

Когда вы знаете что другие люди хотят что то получить от вас, вы будете склонны воспринимать себя как человека с более высокой ценностью. Девушки в ночных клубах заводятся от того, что вокруг много парней и им всем что то надо от них. (С возрастом многие женщины начинают переживать о том, что они уже не получают привычного количества мужского внимания) Если у вас есть доступ к чему то редкому, или вы отлично умеете играть на гитаре, или вы отличный рассказчик и у вас всегда есть истории, которые окружающие хотят услышать, то вы будете чувствовать большую уверенность. Представьте себе парня, который ни разу в жизни не знакомился с женщинами. Он нервничает, когда хочет познакомиться с женщиной, потому что его разум говорит ему «Не подходи к ней, не заговаривай с ней, у тебя нет ценности, а значит и права на эту женщину». Для того, что бы помочь ему познакомиться, вы можете предложить ему попробовать свой «лучший шаблон для знакомства». Если он будет считать, что у него есть отличный шаблон для знакомства, это придаст ему уверенности в себе.

Он будет думать «Так, сейчас у меня есть то, с помощью чего другие парни получают женщин». Это зацепка для его ума, основание для уверенности. Если же парень потерпит неудачу, он будет думать что это просто хреновый шаблон, или же он неправильно его подал. У некоторых людей есть уверенность в себе, которая происходит от их умения развлекать окружающих. Умение развлекать, рассказывать весёлые истории, иметь «всегда успешные шаблонные ходы» может дать уверенность в себе. Пока человек развлекает, он удерживает на себе внимание окружающих. Проблема здесь заключается в следующем: Как только все заготовки парня закончатся, парень сразу же замолчит. Его разум будет говорить ему «Не говори больше ничего, у тебя закончился твой материал, ты не сможешь на ходу, экспромтом, сказать ничего, что будет интересно для окружающих». И он столкнётся с тем, что известно как «Мой подготовленный материал закончился и я не знаю что говорить дальше» Конечно же, он уже прожил достаточное время, и в его жизни происходили события о которых можно поговорить. Но поскольку его чувство уверенности в себе и эмоциональное состояние основывается на его подготовленном материале, то когда его материал заканчивается, заканчивается и его уверенность в себе, падает эмоциональное состояние Женщины чувствуют это и утрачивают влечение к нему. Он думает, что он потерял этих женщин поскольку закончился его подготовленный материал, но на самом деле причина в том, что он вышел из ресурса, потерял уверенность. Вот что происходит когда уверенность парня, чувство что он имеет право, основывается на его способностях (умениях). Он чувствует что может что дать другим, что они что то хотят получить от него, и в результате он чувствует уверенность в себе.

\subsection{Основание 4. Социальные роли}

Как существа социальные, мы запрограммированы играть те роли, которые окружающие ожидают от нас, но иногда случается что то из ряда вон выходящее, и мы начинаем играть не свойственную нам роль. В определённых обстоятельствах, например при стихийном бедствии, когда требуется срочная помощь для спасения людей, люди в обычной жизни не похожие на героев, вдруг преображаются. Они бросаются на помощь не только потому, что их внутренняя система ценностей толкает их на это, несмотря на врождённое чувство самосохранения, но и потому что они знают, что в такой ситуации обязательно кто то должен быть героем, и они принимают на себя эту роль.

В любом обществе есть универсальные роли: лидеры, те за кем следуют остальные и последователи, те кто следует. Мы все знаем о существовании этих ролей, и стремимся играть те, которым как мы думаем, мы подходим больше всего. Если вы стремитесь занимать высокое социальное положение, выполнять роль лидера, быть доминирующим, то вам бороться за это право с другими людьми. Это тяжело, многие не выдерживают конкурентной борьбы, социального давления и решают что легче и проще прожить всю жизнь на дне социальной лестницы, играть менее значимую роль, быть ведомым. Несмотря на это, в определённый момент времени обстоятельства могут сложиться таким образом, что им придётся играть роль с более высокой ценностью. Например, если парень получил более высокую должность на работе, выступает в роли учителя, или же он с девушкой, которая ожидает от него доминантного поведения.

Например если вы пришли в ночной клуб со своими друзьями, и они менее опытны с женщинами чем вы, то вероятнее всего вы возьмёте на себя роль их наставника и войдёте в великолепное эмоциональное состояние. Вы будете играть роль наставника, ту роль, которую от вас ожидают ваши друзья и вероятнее всего вы будете более уверены в себе, чем обычно.

С другой стороны, если вы выходите снимать с парнем, который более опытен, чем вы, то скорее всего вы будете наблюдать за тем, что делает он, и ваша собственная игра ухудшится. Он заметит это, и его собственное состояние усилиться, потому что вы отдаёте ему роль лидера, роль наставника. Роли постоянно меняются. Если два парня беседуют с одной девушкой, то один всегда будет играть более доминантную роль, чем другой. Парень, который играет менее доминантную роль, заметит, как падает его эмоциональное состояние. Если раньше он доминировал в общении с женщиной, а затем другой парень перехватил доминирование, то первый парень заметит, как вслед за утратой доминирования падает его эмоциональное состояние и он начинает играть более низко статусную роль. Эмоциональное состояние людей часто зависит от того, какую роль они играют. Им нужна социальная обратная связь от своих друзей или девушек, для того что бы играть ту роль, которую они хотят играть.

Итак, в определённый момент времени окружающая обстановка ставит вас в условия, когда вы играете роль с более высокой ценностью, чем обычно. И как результат, вы становитесь более уверены в себе.

\subsection{Основание 5. Личность}

Итак, последнее и самое важное основание, которое даст вам право вести себя уверенно --- это личность. Если вы ведёте себя уверенно, базируясь на собственной личности, то:

--- Для вас не будет необходимости отвечать поверхностным социальным стандартам.

--- Для вас не будет необходимости вступать в альянсы с другими людьми, которые добавят вам ценности.

--- Вам не нужно будет иметь что то, что другие люди хотели бы получить от вас.

--- Вам не нужно будет находиться в определённой роли Всё вышеперечисленное это формы ситуационной уверенности. Это внешние опоры для вашей уверенности. Ситуационная уверенность может временно привлечь женщин, однако, эти опоры для вашей уверенности зависят от внешних обстоятельств и потому ненадёжны. Женщина не будет заниматься с вами сексом до тех пор, пока она не составит о вас своё окончательное мнение, а для этого требуется время. Может так случиться, что вы неожиданно лишитесь этих внешних опор, которые поддерживали вашу уверенность в себе и ваша настоящая личность станет очевидна для женщины. Если это случится, то она потеряет влечение к вам и уйдёт. Только уверенность, основанная на вашей личности, ваша внутренняя уверенность, не зависящая от внешних обстоятельств будет всегда с вами. Мы называем это «базовая уверенность».
\DEFINE{«Базовая уверенность»}

«Базовая уверенность» это непоколебимая уверенность в том, кто вы такой и чего вы заслуживаете в жизни. Это постоянная уверенность в себе, вне зависимости от внешних обстоятельств, потому что вы знаете, что одобрение или неодобрение вашего поведения окружающими людьми, ваша неудача в какой то отдельной ситуации, никак не могут вам повредить, повлиять на то, кто вы есть. Базовую уверенность в себе трудно найти, это что то, что вы должны развивать в себе на протяжении всей жизни, день за днём. Большинство людей проживают свою жизнь словно во сне, лишь изредка останавливаясь и задумываясь над истинными мотивами своих действий. Они предпочитают не замечать неудобную правду жизни, и их существование и поведение в этом мире происходит как бы на автопилоте.

Вместо того, что бы самим иметь чёткое представление о том, кем они хотят быть, чего добиться в жизни, они позволяют другим людям определять это. Наша жизнь наполнена давлением общества, как себя вести, как выглядеть, как присваивать ценности общества. Всегда есть искушение поддаться этому давлению и поставить собственную самооценку в зависимость от того, что думают о тебе окружающие. Но в конечном счёте, если вы поддаётесь давлению, приспосабливаете своё поведение к требованиям общества тем самым вы отдаёте свою силу, вы предаёте самого себя. Ваша ценность берёт начало в вашей личности. Базовая ценность начинается с осознания собственной уникальности, есть только вы один такой в целом мире. Никто не в состоянии снизить вашу ценность, пока вы сами не решите это сделать. Очень легко принять такое решение, и прожить всю жизнь как серая мышь и конформист.

Ваша личность --- это ваше право иметь, ваша ценность, ваша аутентичность, только вы можете определять это для себя самого. Другие люди всегда будут оказывать давление на вас, они будут толкать вас к исполнению им нужных ролей, они будут хотеть, что бы вы жили по их правилам. Но никто не имеет власти над вами, до тех пор пока вы сами не отдадите им её. Вот почему, когда вы решите что вы выглядите привлекательно, вы станете привлекательным. Если вы решите что ваша история клёвая --- то она станет клёвой. Не определённая внешность или история клёвая сама по себе, а та личность, которая стоит за ними, делает их клёвыми.

\RULE  Вам не нужны шмотки, альянсы, роли, способности для того, что бы иметь право вести себя как мужчина с высокой ценностью. Ваша ценность должна скрываться в вашей уникальной личности, и способности выражать её.

Ваша ценность заключается в вашем отношении к самому себе. Вы должны твёрдо верить в то, что ваша уникальность, которую вы раскрываете через юмор, рассказывание историй, внешность это то, чем вы можете гордиться.

Это зависит от того, сильно ли ваше убеждение в том, что вы этого достойны, что вы имеете на это полное право, право на свободное выражение своей личности, право на доминирование. Это не какой то идеал, оторванный от реальной жизни. Это то, что вы можете наблюдать постоянно. Вот почему так важно иметь чёткое понимание кто ты есть, и устойчивое эмоциональное состояние.
\DEFINE{«Фрейм»}

Фрейм --- это субъективная интерпретация объективной реальности. Объективная реальность одна, а людей много. У каждого свой фрейм (своя реальность, свой взгляд на мир, своя вера в то, чем является мир). В процессе общения, мы постоянно получаем пинг (обратную связь) от других людей. Иногда, когда люди обмениваются информацией, «пингуют» друг-друга, то выясняется что их фреймы, (их взгляды на мир, субъективное понимание окружающей действительности) входят в противоречие. И тогда, выигрывает человек с более сильным фреймом. Человек, обладающий более сильной верой в правильность своего фрейма и показывающий меньшую эмоциональную реакцию. А человек с более слабым фреймом приспосабливается. Он принимает фрейм более сильного. И тогда, фрейм более сильного человека, становится общей реальностью для обоих.

Вы можете представить себе это образно в виде электрической цепи, в которую включены плавкие предохранители. В роли электрического тока у нас будет выступать социальное давление, а роли предохранителей --- фреймы. Когда происходит конфликт фреймов, усиливается напряжение в сети (возрастает социальное давление), и через некоторое время, наиболее слабые предохранители (фреймы) начинают плавиться. Остаются только самые стойкие.

Люди с сильными фреймами обычно обладают наибольшим социальным влиянием. Большинство людей реагирует на них, и принимает их фрейм. Вот почему так важно иметь чувство собственной базовой ценности.

\RULE  Сильный фрейм самодостаточен. Мир будет для вас являться тем, чем вы думаете, он является. Давайте рассмотрим это поподробнее. Если вам сделали комплимент, похвалили, и у вас высокая самооценка, то вы это воспримите абсолютно нормально. (Да, я такой). Если же у вас низкая самооценка, то вы не сможете так прореагировать. Любой комплимент, похвалу в свой адрес вы будете воспринимать как лицемерие, как попытку вами манипулировать. Другими словами именно ваш фрейм будет определять вашу реакцию. Люди будут думать о вас то, что вы сами думаете о себе.

Существуют и еще более удивительные вещи. Скажем, если кто то, с целью манипуляции, скажет вам не искренний комплимент, а вы воспримите его как настоящий, и если ваш фрейм сильнее, то этот комплимент станет настоящим. Человек сказавший вам его, сам поверит в то, что он сказал. Так же, если кто то дразнит вас, и вас это не задевает, то вы воспримите это как шутку и сами и просто ответите шуткой. Однако, с другой стороны, если дразнилка вас зацепит, заставит вас почувствовать себя уязвлённым, то вы дадите соответствующую реакцию вашему обидчику, и он поймёт, что он попал в цель.

До тех пор, пока вы с помощью сильного фрейма будете относиться к дразнилке как к безобидной шутке, и просто шутить в ответ, --- это будет просто шутка. И для вас, и для того, кто дразнит, и для всех окружающих. Ваш взгляд на происходящее, ваша интерпретация происходящего будет само исполняющимся пророчеством. Мир всегда будет для вас тем, чем как вы думаете, он является.

\RULE  Когда у вас сильный фрейм, то вы сами определяете собственную ценность и ваше положение в группе. Вы определяете кто вы есть. Фреймы людей постоянно конфликтуют друг с другом с разной степенью напряжения. Как существа социальные, мы постоянно играем во фреймовые игры (столкновение фреймов) что бы определить кто будет принимать на себя роль человека с более высокой ценностью. Более сильный фрейм всегда побеждает. Например, два человека одновременно подходят к фонтану с водой, что бы напиться. Первым будет пить тот, кто уверен что должен быть первым, второй же будет ждать. Давайте рассмотрим ещё один пример, где борются два фрейма. Вы и девушка:

ОНА: Я супер-красотка, а ты ещё один парень, который ставит меня на пьедестал, и надеется переспать со мной. Ты не сможешь меня получить, однако, ты можешь развлекать меня, если хочешь.

ВЫ: У меня есть выбор среди женщин, Я разговариваю с тобой потому что женщины обычно глупенькие, но привлекательные, я просто сам развлекаюсь. Если ты покажешь мне что ты чуть умнее, и отличаешься от других женщин в лучшую сторону, то, возможно, продолжение. А сейчас, я разговариваю с тобой просто потому, что мне весело. Это один из примеров противоборствующих фреймов, которые не выражены словами. Люди с высокой ценностью и сильным фреймом никак вербально не обсуждают вопросы: Чей фрейм сильнее? Чья ценность выше? Они просто метасообщают это через поведение. Итак, чей фрейм сильнее вы можете определить по следующим признакам:

\begin{enumerate}
\item Кто оценивает, а кто пытается произвести впечатление?
\item Кто эмоционально реагирует на принятие/непринятие собственного поведения окружающими, а кто к безразличен к тому, что о нём подумают другие.
\item Кто отчаянно пытается поддерживать разговор, а кто устанавливает правила общения.
\item Кто теряет при разговоре уверенность в себе, когда сталкивается с другим представлением о том, «что такое клёво», а кто не меняется.
\item Кто оставался бы таким же весёлым, если бы его собеседника не было здесь, а кто чувствовал бы себя херово без собеседника?
\end{enumerate}

Некоторые люди неправильно представляют себе войну фреймов. Они думают, что это своего рода ментальная драка. Но это не тот случай. В сущности, потребность в постоянном контроле фрейма это реагирующее поведение. Иметь сильный фрейм это не значит подавлять других, стремиться к господству. Это скорее полное контроль над собственным поведением, чувствовать себя комфортно с самим и собой и не отдавать собственную силу для того, что бы заслужить одобрение окружающих.

\RULE  Мы эмоционально реагируем на тех людей, которые могут влиять на то, что мы чувствуем (думаем о себе) (часто мы нуждаемся в них больше, чем они нуждаются в нас) Когда вы демонстрируете сильную эмоциональную реакцию на кого-либо, то тем самым вы воспринимаете его как человека с более высокой ценностью, вы отдаёте свою силу.

Как мы уже говорили ранее, мы склонны игнорировать тех людей, которых мы воспринимаем как людей с меньшей ценностью и фокусироваться на тех, кто как мы думаем, обладают большей ценностью.

Когда вы демонстрируете слишком сильную эмоциональную реакцию по отношению к кому-либо, то тем самым вы показываете что воспринимаете этого человека как важную часть своей реальности, что он для вас представляет большую ценность, чем вы представляете для него.

Вот почему так важно жить в собственной реальности, никому не отдавать власть для того, что бы диктовать вам, кто вы есть. У вас всегда будет сильный фрейм. Представление о самом себе, основанное на каких то внешних вещах: кого вы трахали, что вы можете дать другим, как вы можете развлечь, ваши роли, как люди реагируют на вас в определённый день или в определённой ситуации --- это всё ложные представления. Эти вещи не то, кем вы являетесь. Основываясь на этих вещах, вы никогда не сможете понять, кто вы есть. Вы будете тратить свою жизнь на то, что бы в борьбе вернуть обратно свою силу, которую вы сами добровольно отдали. Ваша сила в вашей аутентичности, в глубоком ощущении того, кто вы есть. На вопрос кто вы есть не существует какого то строго логического ответа, это не какие то ваши лучшие качества или умения. Аутентичность, это не то, о чем вы можете размышлять сознательно. Когда вы аутентичны, что то «щелкает» в вашем разуме, и вы понимаете «Да, это оно» В этой точке происходит стабилизация вашего эмоционального состояния, вы можете свободно выражать свою личность, кто вы есть, и люди будут любить вас за это. Люди будут принимать ваш фрейм, потому что вы даёте им то, чего они по настоящему хотят.

Ваша идентичность это то семя, из которого потом вырастут ваши мысли, которые будут определять ваше поведение, которое, в свою очередь, будет определять реакцию на вас окружающего мира. Когда вы будете получать обратную связь от окружающего мира, ваше понимание этой обратной связи, будет подсказывать вам, как можно улучшить свою личность, которая в дальнейшем будет влиять на ваши мысли и поведение. Обратная связь это мост между внешней и внутренней реальностью. Вам необходимо укреплять её.
\chapter{«Клёвость» и конгруэнтность}

Метасообщения человека всегда говорят о том, действует ли он опираясь на собственную личность, в собственной реальности или же он реагирует на социальное давление, и под его воздействием, меняет собственное поведение. Мысли отражаются в его поведении, и вы всегда сможете это почувствовать. Поведение человека, то, как он себя выражает, его эмоциональное состояние или желание убедить себя самого и окружающих людей в том, кем он является --- это признаки, по которым можно судить о том, тот ли он, за кого пытается себя выдавать. Это то, что называется «конгруэнтность».
\DEFINE{«Конгруэнтность»}

«Конгруэнтность» --- это когда ваше поведение является отражением вашей личности (внутренне совпадает с внешним). Это может выражаться как очевидным образом, так и очень тонко. Когда вы конгруэнтны, вы можете позволить себе делать то, чего другие люди не могут. Возможно вы знаете какого-нибудь чудака, который ведёт себя странно, но в то же время он уверен в том, что то, что он делает, это классно. И эту уверенность он метасообщает окружающим «Мне нравится то, что я делаю, мне комфортно с самим собой. Я не ищу реакции на себе, просто то, что я делаю, даёт мне нужный результат». Когда люди действуют конгруэнтно, то это подразумевает что они уверены в том, что то, что они делают, будет нормально воспринято окружающими.

Когда человек ведёт себя конгруэнтно, это говорит о том, что окружающие, в прошлом, уже воспринимали подобное поведение нормально. Тем самым, на вас оказывается социальное давление, что бы вы так же приняли его поведение.

Когда парень разговорчив, легко вступает в беседу с окружающими, заводит множество новых знакомств --- он просто делает то, что думает\ldots или ему нравиться встречаться с множеством женщин --- это просто часть того, кто он есть. Люди будут воспринимать его адекватно, они не осудят его, он не будет сталкиваться с социальным давлением, потому что люди чувствуют, что то, что он делает, это часть его личности. Люди могут чувствовать это по его движениям, жестам, голосу (по всем каналам, передающим метасообщения). Звучание (тон голоса) один из наиболее информативных каналов. Ему комфортно с самим собой, и с тем способом общения с другими, который он выбрал. Он чувствует то лёгкость при общении, он обрёл гармонию. Возможно, кому то покажется, что его поведение не вполне адекватно, и на него попытаются надавить, что бы изменить его поведение. Но это настолько далеко от его реальности, что он этого даже не заметит, и не покажет никакой эмоциональной реакции на «поучения». А поскольку то, что говорят эти люди, смысла. Обратная связь (пинг)просто не доходит до него. И тогда у окружающих не остаётся выбора, они выходят из своей реальности и втягиваются в реальность парня.
\DEFINE{«Не конгруэнтность»}

Не конгруэнтность это понятие противоположное конгруэнтности, когда существует не соответствие между личностью человека и его поведением. Человек, с какой-то целью, ведёт себя определённым образом, но он не уверен в том, получит ли он нужный результат. В таком случае, происходит рассогласование между тем, как человек воспринимает себя и как он хочет, что бы его воспринимали другие. По причине этого несоответствия, человек будет ощущать дискомфорт и демонстрировать этот дискомфорт через невербалику. Он будет вести себя неестественно. Он может выглядеть как «слишком старающийся», или «слишком желающий». Или он может наоборот выглядеть «недостаточно старающимся». Его голос может дрожать, он может пытаться говорить слишком тихо или слишком громко, его глаза могут бегать или наоборот быть слишком неподвижными, на его лице может отражаться напряжение, а в движениях может быть скованность.

Такой человек будет понимать, что он пытается демонстрировать поведение, которое не соответствует его ценности, и его разум будет сопротивляться этому. Он может говорить невпопад и делать вещи, которые будут выдавать окружающим его внутренний дискомфорт. Потому что его поведение будет не соответствовать его собственному представлению о самом себе.

Именно по этой причине вам не нужно стремится впечатлять. Потому что, что бы вы не делали, с целью впечатлить другого человека, это всегда будет не конгруэнтно. Конгруэнтность это то, на что мы реагируем сразу же, как только знакомимся с новым человеком.
\DEFINE{«Клёвый»}

Нельзя дать точное Определение понятию «клёвый». Это часть вашей личности, даже если иногда толкает вас к выходу за пределы того, что считается нормальным поведением. Это то, что позволяет вам привлекать и удерживать внимание людей. Это то, что позволяет вам думать и действовать без оглядки на то, что о вас скажут окружающие. Это не реагирующееповедение. Это качества, которые позволяют вам выделяться, следовать только собственным правилам, никогда и не под кого не «прогибаться». Эти качества позволяют вам быть яркой, творческой личностью, выходить за пределы того, что считается «нормальным».

Конгруэнтность сама по себе не сделает вас клёвым. Если вы слабак, и неинтересная серая мышь, то вы можете быть суперконгруэнтным, демонстрируя кто вы есть, но вы не будете клёвым. И вы не сможете привлекать внимание женщин будучи конгруэнтным, но не клёвым.

\RULE  Когда вы выделяетесь на фоне других парней, женщины будут считать вас привлекательным, потому что вы метасообщаете высокую социальную ценность. Ваши дети так же будут выделяться. Женщин привлекают парни, у которых есть необычные (привлекательные) черты характера или навыки. Всегда лучше, что бы на вас глядели, чем проглядели. Многие люди скажут вам что «быть клёвым» равноценно понятию «быть собой» (быть конгруэнтным). Но это не так. Другие люди думают, что определённые шмотки и чувство стиля выделят их на фоне окружающих, а значит сделают клёвыми. Это тоже ошибка.

\RULE  Если вы конгруэнтный, но не клёвый --- значит вы обладаете низкой ценностью. Если вы клёвый, но не конгруэнтный --- значит вы слишком стараетесь (пытаетесь быть тем, кем в действительности вы не являетесь). Снова мимо. Для того, что бы вас считали привлекательным, вы должны обладать обеими характеристиками. Ваш жизненный опыт, это ваш материал, который вы используете для того, что бы быть клёвым. Вы пробуете какие то новые вещи, и это получается у вас не конгруэнтно. Вы не отступаете, снова и снова их используете, пока они не начнут получаться естественно. Парень, у которого отсутствуют привлекательные качества, должен выйти из того состояния сомнамбулического транса, в котором находится, и начать более осознанную жизнь. Он должен начать действовать проактивно, начать искать и использовать из своего опыта то, что сделает его привлекательным. Развитие своих внешних навыков позволит ему выделяться, быть клёвым. Однако затем, вслед за внешними навыками, он должен изменить собственное мироощущение (стать конгруэнтным своему внешнему поведению). Он должен развивать свой ум, чувство стиля, остроумие, работать над своей внешностью, но в то же время понимать, что именно ценность в первую очередь, делает его клёвым.

\RULE  И достигнув определённой точки в своём развитии, вы поймёте, что вы получили то, к чему стремились. Что то щёлкнет в вашей голове, и исчезнет ваша последняя тень сомнения. Вы станете по-настоящему клёвым, вы будете разительно отличаться от других парней. Ваша ценность всегда будет с вами, потому что она часть вашей реальности. Вы станете притягательным. Когда вы будете подходить к девушке, вы будете чувствовать уверенность в себе, будете излучать ценность. Потому что вы знаете, что вы и ваша реальность это клёво, множество людей может это подтвердить. Ей это так же определённо понравиться. Давайте рассмотрим поподробнее мыслительные и поведенческие шаблоны. В любой ситуации вы можете выбирать что вы будете делать:

1) действовать,

2) реагировать,

3) бездействовать.

1) Если вы действуете --- это проактивная модель поведения. Это сильное поведение, оно увеличивает вашу ценность.

2) Если вы реагируете, то это слабое поведение. Вы теряете ценность.

3) Бездействовать --- это пустое место. Если вы ничего не делаете --- то вы ничего не получите.

Когда парень демонстрирует реагирующее поведение, это означает, что он желает получить одобрение других людей. Если окружающие не одобряют его поведение, это будет влиять на него, заставит его пережить отрицательные эмоции, изменить собственное поведение. Люди по своей природе существа иррациональные. Часто мы больше стремимся к сохранению собственного комфортного эмоционального состояния в данный момент, чем стремимся делать вещи, которые будут полезны нам в долгосрочной перспективе. (особенно, когда мы и не знаем толком, что будет для нас полезно в будущем) Часто парень, который переживает как его воспримут окружающие, будет пытаться впечатлить окружающих людей или логически убедить их полюбить его, вместо того, чтобы вести себя как ему нравиться. Ему необходимо как то оправдывать свое поведение перед самим собой, рационализировать его. Парень ведёт себя как реагирующий, затем рационализирует собственное поведение, что ведёт ещё к более реагирующему поведению. Это замкнутый круг. Действовать проактивно, не быть реагирующим, означает не поддаваться на социальное давление, стремиться утверждать собственную личность, не приспосабливаться.

\RULE  Действовать проактивно не означает, что вы не уважаете окружающих вас людей, относитесь к ним пренебрежительно. Это означает, что вы просто демонстрируете вашу личность, не пытаясь действовать как реагирующий, в попытке получить одобрение окружающих.

Вы должны быть сконцентрированы на том, как то, что вы делаете в настоящем, отразиться на том, как люди будут реагировать на вас в будущем. Развивая себя, стараясь стать человеком который себя любит и уважает, вы закладываете фундамент отношения к вам людей в будущем. Ваша уникальность, ваш опыт, ваша внешность, ваше самоуважение будет тем, за что люди полюбят вас. Особенно самоуважение, люди не уважают тех, кто не уважает себя. Если же вы реагируете на других, хотите одобрения, то вы сам себе злобный Буратино. Ваше стремление быть реагирующим или проактивным определяется вашим мировоззрением. Фреймы людей постоянно борются друг с другом. Проактивные фреймы всегда сильнее. Если вы отделяете себя от ваших эмоций (есть «Я» и есть «эмоция»), управляете ими, то вы будете стремиться развивать проактивные модели поведения. Если же вы рассматриваете ваши эмоции как часть себя, то вы ставите себя в положение человека, который зависит от мнения окружающих, стремиться получить их одобрение. Тогда вы будете склонны развивать реагирующие модели поведения. Где то глубоко внутри на инстинктивном уровне, мы чувствуем, что проактивные модели поведения может использовать только человек с высокой ценностью. Этот человек будет тем, чьего одобрения будут хотеть окружающие люди. Он будет утверждать собственную личность, контролировать фрейм, определять «что сегодня клёво». Люди будут эмоционально реагировать на него. Реагирующие же модели поведения связаны в нашем сознании с низкой ценностью. Такой человек будет позволять окружающим определять его эмоциональное состояние. Он будет страстно хотеть одобрения окружающих, будет казаться «слишком старающимся». Он хочет логически убедить окружающих в своей высокой ценности. Возможно, вы знаете парня, который пытается увеличить собственную ценность, понтуясь, впечатляя других. Наиболее вероятно, что его мировоззрение содержит в себе предположение, что он не сможет понравиться другим, пока не впечатлит их, окажет какую то услуги или похвастается быть вы так же знакомы с парнем, который никому нечего не стремится доказать. Все принимают его статус, его ценность, как само собой разумеющееся. Наиболее вероятно, что его мировоззрение содержит в себе предположение что он будет нравиться людям до тех пор, пока он весел, позитивен и уверен в себе. Это проактивное поведение, это клёвое поведение. Интересно что и реагирующая и проактивная модель поведения направлена на достижение одного и того же результата --- получение позитивного социального отклика. Вся разница в методах и результатах. В то время как проактивный парень становится всё более успешным и его результат очевиден, реагирующий парень раз за разом терпит неудачу и вынужден объяснять самому себе (рационализировать) почему это произошло снова. Следует отметить, что все мы существа социальные, постоянно взаимодействуем друг с другом и в конечном счёте, любое социальное взаимодействие так или иначе есть реакция. Тем не менее, стремиться быть проактивным всегда лучше, чем быть реагирующим. Вашу проактивность всегда заметят и оценят окружающие. Образно, проактивность можно представить в виде моста, который соединяет внешние и внутренние опоры вашей реальности.
\chapter{Зрелость личности}

«Социальное давление» это то, что происходит, когда вы чувствуете что можете потерять одобрение вашего поведения со стороны общества. Когда вы выходите за рамки общепринятых норм, игнорируете статусные различия, то вы можетепочувствовать страх что вас за это накажут. Когда это происходит, вы обнаруживаете что в ействительности, вы обладаете меньшей ценностью, чем пытаетесь продемонстрировать. Существуют и противоположные ситуации. Вы можете испытывать социальное давление, когда по воле обстоятельств вы «играете роль» с высокой ценностью и чувствуете что можете с ней справиться. Социальное давление происходит всякий раз, когда вы демонстрируете поведение, отличающееся от того, поведения, которое давало вам в прошлом одобрение со стороны окружающих. (демонстрируете новые свойства вашей личности) В зрелом возрасте, мы пробуем различные варианты поведения (одеваем разные маски, как бы примеряем на себя разные личности), и затем мы останавливаемся на том варианте, который, как мы считаем, подойдёт нам лучше всего, для самых различных ситуаций. Этот вариант позволяет нам чувствовать себя комфортно, и мы стремимся к тому, что бы этот комфорт продолжался. В действительности ли обладать низким статусом это так плохо, как кажется на первый взгляд? Многие люди сознательно стремятся к низкому статусу. Многие из этих людей скажут вам, что они «хотят» быть теми, кто они есть. У них есть «причины» так себя вести. Они «просто те, кто они есть». Человек с низким статусом (ценностью) на короткий период времени, получает определённую выгоду от своего низкого статуса. Что бы быть личностью с высокой ценностью, необходимо обладать лидерскими качествами, харизмой, способностью увлечь за собой, выдерживать социальное давление. Многие думают, что они на это не способны. Люди ищут лёгких путей, они не хотят меняться. Демонстрировать высокую ценность так же может означать риск «нападок» со стороны окружающих. Многие предпочитают этого избегать. Люди с низким статусом (ценностью) практически не подвергается социальному давлению, они занимают отводимые им роли в группе. Это несомненный плюс для них, это гарантирует им, что окружающие будут одобрять их поведение. Такие люди будут чувствовать комфорт с собственным низко статусным поведением. Легко, например, понять человека, который беспокоится из-за того, что сегодня он оделся менее модно, чем обычно. Но что кажется странным, что многие люди беспокоятся из-за того, что они в определённый момент, оделись более модно, чем обычно. Но такое бывает. Их беспокойство, их страх --- вызван их мировоззрением. Они не воспринимают себя как людей, которые достойны носить модную одежду. Они хотят оставаться в привычных им рамках: носить одежду определённого стиля, пребывать в привычной им группе людей, занимать привычное положение в группе. Делая все эти вещи, они уверены в том, что будут получать одобрение от окружающих, которое получали ранее.

Конечно, иногда их может тревожить их низкий социальный статус и те ограничения, которые он на них накладывает. Но именно он позволяет им в краткосрочной перспективе избегать негативных эмоций. Когда человек сталкивается с какой то новой социальной ситуацией, входит в новую группу людей, то он обнаружит, что сразу же начинается столкновение фреймов, борьба за высокий статус. Кто удержит фрейм? Кто будет реагировать и приспосабливаться.

Поддержание высокой социальной ценности в группе будет постоянно требовать от человека расходовать свою энергию. Он привлекает к себе внимание окружающих и подвергается наибольшему социальному давлению. К нему предъявляются высокие требования, и он должен их оправдывать для того, что бы сохранить свой статус.

Нет ничего странного в том, что людям нравится позитивная обратная связь от других людей. Мы не любим смотреть фотографии, на которых «не получились», мы разглядываем себя в зеркале под разными углами, пытаясь понять под каким углом мы выглядим привлекательнее. Люди бессознательно выбирают те модели поведения и воспринимают себя так, как они хотят что бы их воспринимали другие. Люди могут иметь потребность всегда находится в знакомой компании, где у них есть друзья, люди которые ими восхищаются, где они зависят друг от друга. Люди могут иметь потребность всегда одеваться в определённом стиле. Некоторые не могут поддерживать нормальную беседу при знакомстве, им необходимо сначала выпендриться: показать как много у них нужных социальных связей, какие они крутые, рассказать о своих достоинствах. Они стремятся демонстрировать проактивное поведение, контролировать фрейм, но в реальности они демонстрируют реагирующее поведение, они будут восприниматься как «слишком старающиеся». Многие люди не способны пойти в одиночку в кинотеатр, ресторан или клуб. Если они всё же идут, то они будут суетится, делать всё что бы окружающие не заметили, что они пришли одни. Если такой человек придёт в ресторан один, то он будет демонстрировать окружающим видимость кипучей деятельности, показывать что он чем то занят (например разговаривать по мобильному телефону). Такому человеку будет некомфортно в клубе в одиночку. Он не сможет знакомиться с девушками без своих друзей, которые ему необходимы для «моральной поддержки». Один он будетнервничать, и вероятнее всего пить алкоголь, пытаясь приглушить некомфортное состояние.

Конечно, это отлично, когда люди собираются вместе и весело отвисают, наслаждаясь компанией друг друга. Но в то же время многие бояться выходить отдыхать одни, даже если они этого хотят. Они чувствуют себя неуютно без своих друзей, они озабочены тем, что о них подумают окружающие.

У них есть ситуационная уверенность, но нет базовой уверенности. Легко действовать в социальной ситуации, когда ты нравишься окружающим, окружён друзьями. Когда же ситуация меняется, становится нейтральной (нет одобрения и поддержки, всем в общем на тебя похуй) то многие «сдуваются», их ситуационная уверенность закончилась, а базовую они не развили.

Человек, который развил в себе базовую уверенность, всегда будет уверен в себе, вне зависимости от того, получает ли он поддержку со стороны друзей. Для такого человека нет необходимости принимать чужие фреймы, подлизываться к новым знакомым, для того, что бы получить их одобрение и чувствовать себя комфортно. Основываясь на собственной личности, на уверенности в себе, на собственной высокой ценности, такой человек буден убеждён что он понравиться новым знакомым, и получит их одобрение. Когда у человека существует базовая уверенность, то он легко сможет приобрести ситуационную уверенность. Для этого ему всего лишь необходимо будет освоить некоторые новые формы поведения. Это не будет затруднительно. Если же человек обладает только ситуационной уверенностью, то для того что бы ему приобрести базовую уверенность, необходимо будет полностью поменять своё мировоззрение. Это более долгий и трудный процесс. Парня, который обладает сильной базовой уверенностью, но слабой ситуационной уверенностью, можно характеризовать как человека, который уверенно и комфортно чувствует себя с женщинами в целом, но у него не хватает навыков подойти и познакомится со случайной женщиной на улице. Он просто не знает, как это делается. Его базовая уверенность, которую он приобрёл в жизни, а так же социальные навыки позволяют ему чувствовать что его личность, его самооценка, не находится в зависимости от того, как реагируют на него женщины. Знакомые девушки любят его, он пользуется популярностью, у него великолепные подружки. Для него нет проблем позвонить какой-то знакомой девушке, раскрутить её на свиданку или секс. Но когда он встречает незнакомую девушку на улице, он просто не знает как надо действовать. Есть так же парни, которые обладают сильной ситуационной уверенностью, но слабой базовой уверенностью. Такой парень будет уверенно знакомиться с женщинами, привлекать их внимание, он будет нравиться женщинам. Он легко берёт их номера телефонов. У него есть нужные социальные навыки и ситуационная уверенность, которые позволяют ему легко знакомиться с женщинами. Он знает, как можно заставить женщину смеяться или заинтриговать её. Он чувствует себя комфортно и уверенно до тех пор, пока всё идёт по его плану и он получает результат, который ожидал. Он чувствует что он клёвый и он входит в роль клёвого парня. Но он нуждается в положительной обратной связи. Его уверенное поведение, зависит от той реакции, которую он получает. Одобряется ли его поведение. Если его поведение не одобряется, он перестаёт быть клёвым. Если он подходит к женщине и он ей нравится, (получает положительную обратную связь) он начинает действовать проактивно. Но если он подходит к женщине и вначале она показывает безразличие к нему, то он «сдувается». Он перестаёт быть конгруэнтным и выходит из роли клёвого парня. Однако, если бы он продолжал быть клёвым, то рано или поздно он бы зацепил её внимание. Он отступает от неё и думает «Она такая красивая. С чего это я решил, что смогу привлечь её?»

Он вдруг начинает думать о том, какая она красивая, как много мужчин, наверное, добиваются её. Какой у неё большой выбор из мужчин. И поскольку, для того что бы играть роль клёвого парня, ему постоянно нужна положительная обратная связь, которую он в данный момент не получает, он начинает сомневаться в себе. Теряет уверенность, теряет клёвость.

Возможно даже, что такой парень будет готов, что вначале знакомства девушка может быть равнодушна. Он продолжает свою роль клёвого парня и в определённый момент цепляет её. Но как только он понимает, что он начинает нравиться девушке, это заставляет его беспокоиться. Забавное в этом вот что: Парень смог искусно познакомиться с девушкой, справиться с её равнодушием, привлечь её. Он прошёл самый трудный этап. Когда же он привлёк её, он вдруг начинает волноваться, он не знает что делать дальше. Он чувствует эмоциональный подъём от того, что он смог познакомится с ней, вызвать её позитивную реакцию. Но сейчас вдруг он начинает бояться. Он начинает бояться потерять этот эмоциональный подъём больше, чем потерять девушку. Она хочет продолжить знакомство, и они обмениваются номерами телефонов. Он звонит ей на следующий, но он волнуется и уже не так уверен в себе. Он разговаривает с ней по телефону, но она чувствует что это уже не тот парень, с которым она вчера познакомилась. Девушка, чувствуя это беспокойство парня, реагирует уже не так хорошо, как вчера. Она разговаривает более равнодушно и парень не готов к этому. Он теряется ещё больше, и девушка решает больше с ним не встречаться. Возможно, парень преодолеет и этот рубеж, и его ситуационная уверенность продлиться дольше. Возможно, они решат сходить куда-нибудь вместе. Встретившись с ней в середине дня, парень опять может потерять уверенность в себе. Он может начать нервничать, его сердце будет биться, а ладошки потеть. У него не достаёт базовой уверенности, что бы контролировать ситуацию. Наш парень опять рискует просрать девушку. Парень с сильной базовой уверенностью в себе всегда будет выглядеть более естественным, ему будет легче общаться. Он будет более проактивным, будет просто выражать то, что он думает и окружающие его люди будут реагировать на него. Парень со слабой базовой уверенностью будет всегда выглядеть более неловким при общении с малознакомыми людьми. Он будет более реагирующим, потому что для уверенного поведения ему требуется одобрение окружающих. Это будет очевидно для всех.

В отличие от ситуационной уверенности, базовую уверенность нельзя приобрести как навык. Однако, ситуационная уверенность, может стать первым шагом в вашем развитии. Ситуационная уверенность, на короткий период времени, позволит вам получать позитивную обратную связь со стороны окружающих. Это может стать первым шагом в верном направлении.

У каждого человека есть определённый уровень уверенности, на который он возвращается, как только перестаёт получать одобрение окружающих. Если парень имеет слабую уверенность в себе, постоянно тревожится, то он сформирует личность, которая поддерживает это его состояние. Однако если парень преодолеет свой страх и начнёт пробовать что то новое, получать новый опыт, то он начнёт постепенно меняться. Это будет подталкивать его к поиску базовой ценности. Он будет развиваться. Если парень научится вступать в разговор с незнакомой женщиной и поддерживать беседу сначала 5 минут, затем 10 минут, затем 15 минут, он поймёт что всё в порядке, нечего боятся. Парень осознает что иметь высокий социальный статус имеет свои преимущества. Он захочет развивать новые навыки.

Он начнёт искать в себе и окружающих те черты личности, которые говорят о человеке с высокой ценностью. Он захочет развиваться и присваивать их. Постепенно, поведение, демонстрирующее высокую ценность, перестанет требовать усилий с его стороны, будет естественным. Он больше не будет нуждаться во внешнем одобрении для сохранения уверенности в себе, потому что он стал зрелой личностью. Возможно, вы видели, как это происходило в школе. Иногда дети начинают пробовать какие то новые модели поведения, и если эти модели оказываются успешными, то они закрепляются и происходит изменение личности. Например, ботаник хочет начать дружить с популярными парнями. Он начинает искать их общества: слушать ту музыку, которую слушают эти парни, принимать участие в какой-то совместной деятельности, копировать их манеру одеваться. Через некоторое время он становится абсолютно непохожим на себя прежнего. Вслед за изменением поведения, изменилась его личность. Для того что бы быть успешным с женщинами, вам в сущности, необходимо делать то же самое.

Парням, которые стремятся к успеху у женщин нужно пройти тот же путь. Пока они не знают, что является привлекательным, им необходимо экспериментировать. У них начнёт появляться как позитивный, так и негативный опыт общения с женщинами. Всё это пойдём им на пользу.

Они начнут развиваться. Покупать модную одежду, учиться поддерживать естественный разговор, шутить, рассказывать весёлые истории. Они будут стремиться узнавать клёвые места в городе, куда можно привести женщину. Со временем, их клёвое поведение станет для них естественным.

Вам необходимо прекратить внутреннюю борьбу и начать пробовать новые виды поведения. Вам на практике необходимо проверить те мысли, которые как вы думаете, привлекут к вам женщин. То что работает вы оставите, что не работает --- отбросите. Вы никогда не узнаете правду, пока не попробуете. В процессе того, как вы пытаетесь пробовать что то новое, вы начнёте открывать в себе такие качества, о существовании которых вы даже не подозревали. Вы можете познать ваши слабые стороны, увидеть собственное реагирующее поведение, а так же узнать ваши действительно ценные качества. Развивая ваши привлекательные качества, вы начнёте развивать вашу уверенность в себе и в конечном счёте лучшее понимание того, кто вы есть. Вы можете занимать активную или пассивную жизненную позицию. Вы можете продолжать держаться за то, что тянет вас назад и заставляет реагировать. Или вы можете выбрать хорошее направление собственного развития. Быть успешным с женщинами не означает научиться ловко скрывать собственные недостатки. Вам необходимо развить собственную личность, почувствовать базовую ценность и научится демонстрировать её.

Давайте вспомним нашего бедного парня, которого бросила девушка. Потеряв её, он решил, что заработав много денег, накачавшись и приобретя какие-то другие материальные блага, он станет привлекательным для женщин. Он думал что это сработает, но это не работает. Наш парень начал развивать в себе ситуационную уверенность, основанную на обладании какими то внешними вещами. Он строил свою жизнь на желании получить одобрение окружающих, ища их реакции на себя.

Он добился этих вещей, но он не стал им конгруэнтен. Он хотел стать человеком, который привлекает внимание и вызывает восхищение окружающих, но он им не стал. Он не стал парнем с базовой уверенностью. Все эти внешние атрибуты (красивое тело, наличие денег, крутой тачки) --- это классные вещи, они могут увеличить привлекательность, но только в том случае, если за ними стоит личность с базовой уверенностью. Он хотел восхищения и внимания со стороны окружающих, оставаясь в душе всё тем же неуверенным в себе парнем. За этими вещами он хотел спрятать свою старую личность. Он думал, что обладание чем-то внешним сделает его клёвым. Но никто в это не поверил, даже он сам. Он построил свою реальность на девушке, на внешней опоре и получал через это подтверждение собственной ценности. Она была краеугольным камнем в фундаменте его реальности. Когда она ушла (этот камень вытащили) вся его реальность развалилась как карточный домик. Он испытал эмоциональный удар, однако вновь решил построить свою реальность на внешних опорах. Он думал что это поможет ему.

Он думал что это эмоциональное потрясение, которое он испытал, связано с тем, что он потерял любовь всей своей жизни. Его дальнейшая судьба будет зависеть от того, сможет ли он выстроить более надёжные опоры для своей реальности. Наряду с приобретением каких-то материальных благ и внешних атрибутов, ему необходимы так же новые социальные навыки и новое мировоззрение. Только конструкция, основанная как на внешних, так и на внутренних опорах, может быть надёжной.
\chapter{Абсурдность социальных ценностей}

Секс это не только источник получения удовольствия, но источник подтверждения нашей ценности. Давайте представим себе парня гуляющего в одиночестве. А теперь представьте себе парня, гуляющего со своим клёвым другом. А сейчас, представьте себе парня, прогуливающегося под руки с двумя красивыми девчонками, девчонки хихикают и флиртуют с парнем, видно, что они классно проводят время.

Как вы думаете, кто из этих парней привлечёт к себе больше внимания? Как вы думаете, кто из этих парней является самодостаточным и не нуждающимся в окружающих? В нашем обществе мы воспринимаем мужчину, у которого много женщин, как сильного, успешного мужчину. В нашем обществе сильные и успешные мужчины обычно окружены красивыми женщинами: стареющий олигарх с красивой молодой женой, рок-звезда со своими поклонницами, известный боксер появляющийся в обществе под ручку с ослепительной красоткой. Почему же если женщина узнаёт, что парень пользуется успехом у других женщин, то он становится для неё более привлекательным?

Потому что мужчина, который имеет выбор среди красивых женщин, будет иметь потомство, у которого будет тот же самый выбор. В то время как парень быстро определяет ценность девушки, основываясь на её внешности, девушке для определения социальной ценности парня требуется гораздо больше времени. Она может оценивать парня основываясь на том, как другие женщины воспринимают его, для неё это надёжный индикатор его ценности. Это не означает, что женщине обязательно понравится мужчина, который известен тем, что ведёт беспорядочную половую жизнь, спит с кем попало. Скорее, она будет привлечена мужчиной, который имеет высокую социальную ценность. Потому что если мужчина имеет выбор среди женщин, то он как бы метасосообщает, что он умеет обращаться с противоположным полом, данное качество является привлекательным для женщины. Секс может рассматриваться как некое окончательное признание женщиной ценности мужчины. Когда девушка «даёт» парню, то это можно рассмотреть, как готовность девушки рискнуть собственной ценностью в глазах общества, только для того, чтобы получить ценность от парня. Ценность может иметь различные формы: это и получение сексуального удовольствия, и его признание её ценности, и желание выйти замуж за этого парня. Термин «давать» имеет логичное объяснение в рамках социального программирования. Занимаясь с парнем сексом она как бы отдаёт ценность парню, сама при этом теряя социальную ценность. Его ценность растёт, а её падает.

Представьте себе парня, у которого великолепная подружка. Он её любит и тратит своё время и энергию, вкладывая их в свои отношения с этой девушкой. Но в один прекрасный день он вдруг обнаруживает определённую информацию о её сексуальном прошлом. Как выясняется, однажды она занималась сексом с одним из игроков футбольной команды, в раздевалке, сразу после игры. Это слегка «задело» его и он решил собрать дополнительную информацию о её прошлом. Как выяснилось, это не был какой то единичный случай, она трахалась со всей футбольной командой. Это стало ударом для него и он пытается как-то разобраться в своих чувствах и размышляет «Если она трахалась со всеми этими парнями то как я могу доверять ей?» Конечно же, для него не имеет значения, что на тот момент она не была с ним и не имела никаких обязательств по отношению к нему. На самом деле её прошлое никак не связано с их отношениями в настоящем. Он все время думает об этом и его восприятие этой девочки меняется. Его знания о её сексуальном прошлом как бы обесценивают эту девушку в настоящем, он уже не испытывает при общении с ней те эмоции, которые испытывал ранее. Её признание его ценности не имеет больше для него значения. «Если в прошлом любой парень мог трахнуть её даже не ухаживая за ней, то как она может быть моей подружкой в настоящем?»

Абсурдность социальных ценностей тесно связана с социальным программированием. Если девушка воспринимается как «легкодоступная» то, скорее всего она не будет рассматриваться мужчинами как потенциальная подружка для долговременных отношений. Её социальная ценность будет снижаться, поскольку мужчина будет думать, что любой другой парень может поиметь её. По этой причине женщины учатся вести себя таким образом, что бы не казаться «легкодоступными». Они стараются вести себя так, как будто их трудно получить, тем самым, демонстрируя высокую социальную ценность. Для мужчины нет подобных ограничений. Если мужчина занимается сексом со многими женщинами, это воспринимается обществом нормально. Более того, если мужчина имеет выбор среди женщин, это только подчёркивает его ценность. Часто для мужчины не имеет большого значения, если он обнаруживает, что его подружка не такая и привлекательная физически, как ему казалось сначала. Скорее он может тревожиться о том, сколько мужчин у неё было до него. В то же время, девочка не думает, что она что-то потеряла, переспав со многими мужчинами, просто заботясь о своей высокой социальной ценности, она старается всегда казаться труднодоступной.

Интересно, что человек может иметь как некую внутреннюю, собственную, постоянную ценность так и некую социальную ценность, которая достаточно относительна и которую он может дать другим. В случае с мужчиной, он может создать у женщины ощущение того, что только он один может подарить ей сексуальное наслаждение.

Женщина же всегда хочет создать у мужчины ощущение, что она является труднодоступной, и только она одна может дать ему подтверждение его ценности. Всё это возвращает нас к одной из закономерностей, рассмотренных нами ранее: мы всегда склонны эмоционально реагировать на тех людей,

1) которых мы воспринимаем как имеющих более высокую социальную ценность, чем мы, или на тех,

2) которые способны увеличить или уменьшить нашу собственную ценность.

3) И мужчины и женщины так же находят привлекательными людей, которые способны приподнять их эмоциональное состояние.

Когда женщина занимается сексом со многими парнями она как бы метасообщает, что воспринимает этих парней как имеющих более высокую ценность, чем её собственная. Поступая таким образом, она демонстрирует собственную низкую социальную ценность и секс с ней не может служить подтверждением ценности для мужчины. Когда же парень занимается сексом с большим количеством девушек, то он метасообщает о том, что большое количество девушек воспринимают его как имеющего более высокую ценность, чем их собственная, и поэтому он является источником возможного подтверждения ценности для девушки.

Когда мужчина представляет для женщины источник подтверждения ценности, тогда он становится для неё более привлекательным. Но влечение которое женщина чувствует по отношению к мужчине с высокой ценностью, это не всегда то же самое влечение, которое женщина чувствует по отношению к физически привлекательному парню. Если мужчина привлекателен внешне, то она может почувствовать сексуальное возбуждение, она почувствует своего рода «сексуальную агрессивность». Однако если женщина видит мужчину с высокой ценностью, то скорее станет сексуально-восприимчивой, она примет его фрейм.

Когда мужчина обладает высокой ценностью, женщина начинает реагировать на него, принимая его фрейм. Её внимание будет приковано к этому мужчине, она будет чувствовать сильное эмоциональное возбуждение. Её влечение будет проявляться в следующем: Во-первых, она будет воспринимать его как источник подтверждения собственной ценности. Во-вторых, она захочет находиться рядом с ним. В-третьих, она будет восприимчивой по отношению к нему, если этот мужчина заговорит с ней она с удовольствием откликнется на него. Несмотря на то, что сексуальная агрессивность и сексуальная восприимчивость в общем разные понятия, вместе и по отдельности они ведут к сексу.

Когда мужчина физически привлекателен, он может получить определённое внимание со стороны женщин, благодаря своей внешности. Если его внешность даёт ему уверенность в себе, то через своё поведение он будет демонстрировать высокую ценность и девушки станут более сексуально восприимчивыми по отношению к нему. Однако следует заметить, что привлекателен ли мужчина внешне или нет, в любом случае он может демонстрировать высокую ценность и получать соответствующую эмоциональную реакцию со стороны женщин. Ввиду того, что физическое прикосновение вызывает эмоции, то девушка почувствует возбуждение от его прикосновений и станет более сексуально напористой в ответ.

Логически, женщина социально запрограммирована обществом на то, что бы она верила, что ей должны нравивтся мужчины, которые хотят с ней долговременных серьёзных отношений. Но на эмоциональном уровне женщина догадывается, что мужчина, который слишком хочет её, хочет отношений с ней, может просто не иметь выбора среди женщин (он нуждающийся). Таким образом, женщина не будет чувствовать влечения к мужчине, который пытается логически, в рамках социального программирования, убедить её в том, что он её лучший выбор и они должны быть в отношениях.

Интересно наблюдать, как двое мужчин спорят друг с другом, выясняя, кто из них «круче», а женщина в этот момент стоит рядом, и кажется совершенно не обращает внимания на то, что происходит. В то время как её логика совершенно выключена, её эмоциональное восприятие оценивает происходящее. И как только один из мужчин решительно доказал своё превосходство над другим, тем самым продемонстрировав свою более высокую ценность она мгновенно переключит на него внимание, абсолютно потеряв интерес к мужчине с более низкой ценностью. Если девушка бросает своего парня, она может даже сказать ему, «объясняя» причину разрыва «Тот парень был такой странный, он даже приставал ко мне, я хотела уйти оттуда, а ты почему захотел, что бы я осталась». Девушка всегда выбирает парня с более высокой ценностью, а затем применяет обратную рационализацию, для того что бы «объяснить» своё влечение в терминах социального программирования.

Возможно, если даже парень с высокой ценностью высморкается прямо перед ней в носовой платок, а затем, вместо того, что бы положить его себе в карман, он выбросит его в мусорное ведро, женщина может подумать «Он так уважительно относится к окружающим. Он такой классный. У него определённо есть положительные качества. Возможно, судьба свела нас вместе для того, что бы я помогла ему их развить и открыть для окружающих».

В конечном счёте, многие женщины начинают понимать, каких именно парней они хотят. Они хотят встречаться с парнями которые их возбуждают, которые способны бросить им вызов, которые не похожи на остальных, обычных и милых парней.

Они даже готовы, открыто и прямо обсуждать природу своего влечения с парнями, с которыми они встречаются. Однако это будет происходить только в том случае, когда мужчина уже готов услышать «правду» и понять что женщина имеет ввиду. Если мужчина уже готов это понять и разговаривать об этом, то это классно. Но, большинство мужчин слишком поглощены мыслями о том, как получить от женщины подтверждение собственной ценности. Пока мужчина не разобрался в абсурдности социальных ценностей и не готов увидеть голую, неприкрытую правду, с ним женщина ничего не будет обсуждать.

\chapter{Слепые пятна}

Немногие люди способны увидеть себя такими, какие они есть.

Давайте представим себе парня, который просыпаясь каждое утро, тщательно гладит рубашку, аккуратно отгибает воротничок, и вообще внимательно следит за мельчайшими деталями свой внешности. Он нервничает из-за того, что подумают о нём другие люди, когда его увидят.

Парень никак не хочет осознать, это то в принципе это никого не заботит. Большинству плевать на него, и на эти мелкие детали. Никто их даже не заметит. Но что произойдёт, если парень вдруг осознает, что он тратит своё время и свою энергию на то, что не имеет смысла?

У него есть близкий друг-женщина, которую он в тайне любит. Каждые выходные они проводят вместе, когда они встречаются, он использует дорогую туалетную воду и тщательно прибирается в своей квартире, на всякий случай.

Однажды, он решается на смелый шаг. Она всегда говорила ему, что относится к нему, прежде всего как к другу. Она всегда жаловалась ему на то, что парни, с которыми она встречается, плохо с ней обращаются. Но он не такой. Сегодня вечером он докажет ей, что он тот парень, которого она так долго ждала. Сегодня вечером у них будет горячее свидание. У неё проблемы с парнями, но он тот самый парень, которому она может доверять, и с которым она может поговорить обо всём.

Они идут в ресторан, он покупает вино, они разговаривают. Но вдруг, какой то незнакомый парень, которого они даже не знают, влезает в их беседу. Наш парень пытается что-то сказать, но кажется что девочка его практически не слышит, она флиртует с незнакомцем. Тогда наш парень, покупает этому незнакомцу выпивку и смеётся над его шутками. Незнакомец вознаграждает нашего парня за его поведение, говоря ему «Чувак, ты классный». Несколько минут спустя, нашему парню кажется, что его мир рушиться, он видит, как его девушка весело проводит время с незнакомцем.

Каждые несколько минут эта новая парочка выходит на улицу, спокойно попивая напитки, которые он им купил. Она говорит нашему парню «Мне сегодня нужно быть дома пораньше, Джо меня подбросит. Представляешь. Оказывается мы живём совсем рядом. Я позвоню тебе завтра. Спасибо за всё, целую, пока».

Парень имеет два возможных объяснения произошедшему.

Объяснение номер один. «Этот незнакомец был знаком с ней всего несколько минут, а я ухаживал за ней шесть месяцев. Она показала кто она такая. Я рад, что я всё-таки обнаружил, что она шлюха. Это даже к лучшему. Этот парень просто трахнет её и потом бросит. Он просто обманул её, а она этого даже не заметила».

Объяснение номер два. «Этот парень просто переиграл меня. Я тупил на протяжении 6 месяцев, а он просто подошёл и взял то, что он хотел. Он понравился ей потому что он более интересный, более уверенный в себе парень, чем я. Он был весёлым, он умел флиртовать и поэтому она уехала с ним».

Скорее всего, он выберет первое объяснение своего провала. Возможно, пройдут годы, прежде чем он поймёт, что правильным было второе объяснение. Или он никогда не поймёт этого.

Слепые пятна могут быть очень коварными. Парень поступает не в соответствии с реальностью, а в соответствии со своими убеждениями о том, какова должна быть реальность. Наши представления о реальности базируются на том, чему нас научило общество и на нашем собственном опыте.

Психологическая уверенность, которая базируется на понимании собственной идентичности и определяет то, как мы воспринимаем мир. Без психологической уверенности мы не будем способны принимать даже ежедневные простые решения.

Но иногда возникают вещи, которые несут угрозу для нашей психологической уверенности. Для того, что бы нейтрализовать эту угрозу, мы создаём «слепые пятна», мы просто перестаём видеть вещи, которые противоречат нашим убеждениям.

Слепые пятна это очень сильная вещь. Помните, когда мы обсуждали различные ситуации, мы говорили с вами и о том, что мы, прежде всего, склонны фокусировать своё внимание на тех вещах, которые имеют ценность для нас. Ценность может так же являться и эмоциональной ценностью, люди склонны цепляться за свои старые убеждения, которые дают им положительные эмоции.

Итак, поступая в соответствии с нашей моделью реальности, мы испытываем положительные эмоции, и постепенно накапливаем всё больше новых слепых пятен, которые не согласуются с тем, чем, как мы верим, является мир.

Иногда происходят вещи, которые могут представлять угрозу нашей модели мира, и заставить испытать вас неприятные эмоции. Для того, что бы этого не произошло, для того, что бы предотвратить возникновение неприятных эмоций, мы рационализируем эти угрозы, мы находим им «разумные объяснения», которые позволяют нам сохранять собственные представления о реальности в неприкосновенности.

Неважно, насколько глупым выглядит подобное поведение, оно просто позволяет нам поддерживать собственное комфортное состояние. Если мы натыкаемся на что-то, что не вписывается в нашу реальность, что противоречит нашим убеждениям, то мы просто перестаём это замечать, для того что бы сохранить в неприкосновенности нашу психологическую уверенность.

Иногда, может пройти много лет, прежде чем мы сможем оглянуться назад, и посмотреть на события давно минувших дней совершенно иначе. Нам может показаться смешным то, во что мы раньше верили. С течением времени, мы накапливаем всё больше неопровержимых фактов, не замечать которые становится все сложнее.

Постепенно, эти упрямые факты все-таки доходят до нашего сознания и разрушают наши прошлые взгляды на мир, наши старые убеждения. У человека появляется чувство того, что его убеждения, то во что он верил, всегда расходится с его собственным опытом.

Возможно, он чувствует смятение, чувство неудовлетворения и он начинает подвергать сомнению то, вот что он верил все эти годы, то, чему его учило общество.

Если в этот момент у человека есть внутренние опоры для поддержания своей реальности, то психологическая уверенность останется вместе с ним, и он сможет наблюдать за исчезновением своих слепых пятен с интересом и любопытством. Или же в этот момент его психологическая уверенность может исчезнуть и тогда исчезновение вещей, на которых держались его убеждения, принесёт ему страдания.

В этот момент человек уже готов для того, начинать воспринимать упрямые факты, открыть свои глаза и начать замечать вещи, которые он до этого игнорировал. Но если вы убрали у себя какоелибо слепое пятно, то это не значит, что вы сможете убрать подобное слепое пятно у другого человека. Каждый самостоятельно должен захотеть находить и убирать собственные слепые пятна.

Женщины --- это одно из самых распространённых слепых пятен для мужчины. Для мужчины открытие того, как можно улучшить свои навыки с женщинами, может потрясти до основания его мир.

В определённый момент мужчина может понять, что он не способен привлечь женщину, что социальное программирование, те знания, которые ему дало общество, являются бесполезными. Он может понять, что те поверхностные вещи, те внешние атрибуты, которые он считал важными, на самом деле не имеют значения. И тогда, он должен понять, что, прежде всего, он должен развить свою личность.

Мужчина может быть мужественным и бесстрашным, он может добиться успеха в самых разных областях жизни, и, не смотря на это, он может иметь слепые пятна относительно женщин.

Если женщина отвергает мужчину, то в его понимании это может обесценить тот успех, которого он добился в других областях жизни. Особенно если мужчина «купился» на социальное программирование и принял идею о том, что успешных мужчин не отвергают. Это будет для него жестоким ударом, если он предполагает, что успешность в жизни вообще, гарантирует ему успех с женщинами в частности.

К счастью, у него всегда есть причина, для того, что бы даже и не пытаться подходить к женщинам. «В этом месте нет достаточно красивых для меня женщин. Они одеты вызывающе и ведут себя как шлюхи. Если бы я по настоящему захотел одну из них, то я бы получил её. Я просто не хочу никого» И вот в таком благодушном настроении он сидит и ждёт, пока одна из женщин проявит к нему интерес.

Наконец, одна из женщин начинает делать ему очевидные намёки, что он ей понравился. Справившись со своим волнением, он подходит к этой женщине, и если он не допустит очевидных косяков, то он увезёт её к себе домой. Сам он в это время уверен в том, что это он выбрал её первым, и только его «мастерство» позволило ему соблазнить её. Если вы спросите его о его «съёме» то он будет, гордясь рассказывать, что «Если мне понравилась какая-то девочка, то я её получаю, так всегда».

В общем, существует 2 типа парней, которые успешны с женщинами.

Первый тип --- это парни, которые стремятся избавляться от своих слепых пятен, развивают свои социальные навыки и калибровку. Второй тип парней --- это парни, которые имеют большие слепые пятна, но они используют их для собственной пользы.

Первый тип парней --- это парни, которые знают «ответы на все вопросы», они осознают существование слепых пятен, они видят свои собственные прошлые слепые пятна. Они калибраторы, они видят, как другие люди реагируют на них и развивают сверхпонимание эмоциональной сферы других. И хотя такое сверхпонимание заставляет их немного нервничать, они действуют проактивно, они развивают в себе острое чувство того, что является клёвым, как вести себя естественно, и отличное понимание того, как его действия и поведение отражаются на его ценности в глазах окружающих людей. Он сильный социальный калибратор. Он научился создавать для себя ценность в любой ситуации, он отлично понимает на какое поведение окружающие его люди позитивно отреагируют. Он использует обратную связь со стороны окружающих для того, что бы постоянно совершенствовать своё поведение и получать все более хорошие результаты. Его уверенность в себе основывается на той отличной реакции, которую он получает со стороны окружающих его людей и длится ровно столько, сколько времени он эту реакцию получает. Это дарит ему положительные эмоции.

Второй тип парней --- это парни, у которых есть «бредовая уверенность в себе», они имеют слепые пятна, но эти слепые пятна работают им на пользу. Он сохраняет психологическую уверенность в себе, не обращая внимание на то, как на него реагируют другие, и, в конечном счете, у него всё получается. Он верит в то, что люди которые слишком много думают имеют низкую ценность, они просто лохи. В сущности, если он вдруг начнёт заниматься калибровкой, задумываться о том, как он выглядит в глазах окружающих то некоторые его слепые пятна начнут исчезать, и он разрушит свой фрейм, на котором и основывалась его успешность с женщинами. (Потому что, его чувство уверенности в себе держалось на определённых слепых пятнах). Это так же сделает его более реагирующим. Его успешность заключается не в калибровке, не в получении обратной связи от окружающих, его успешность это его исключительно сильный фрейм.

Он так же научился создавать себе ценность в любой ситуации, просто его поведение, в отличие от парней первого типа, строится на основе настойчивости и чрезмерной уверенности в себе. Вместо того, что бы обнаруживать и устранять слепые пятна, которые вызывают страх подхода к женщинам, он развивает слепые пятна, которые позволяют ему не замечать неудач и негативной реакции со стороны женщин. Его уверенность в себе основывается на тех успехах, которые у него были в прошлом. Он постоянно вспоминает эти успехи и это даёт ему убеждённость, что всё, во что он верит, является верным.

Существует два варианта, как привлечь женщину. Первый вариант это делать действительно клёвые вещи, и второй вариант это безусловная вера в то, что то, что ты делаешь, является клёвым. Обычно, необходимо соблюдать разумный баланс между первым и вторым вариантом.

Давайте представим себе компанию девушек, которые сидят за столиком в ресторане. В течение вечера, к ним подходили познакомиться два парня.

Первый парень, вскоре после знакомства с девушками, почувствовал, что он им не интересен. Проанализировав ситуацию, он попытался изменить своё поведение, но девушки по-прежнему мало интересовались общением с ним. Парень был хороший калибратор, он прекрасно видел равнодушную реакцию девушек на себя, в то же время для того, что бы чувствовать себя уверенно и комфортно при общении, ему необходима позитивная реакция на себя девушек. Не добившись позитивной реакции, поняв, что он не интересен девушкам, парень вежливо с ними попрощался. Второй парень, подойдя к девушкам, даже не смог понять, что он им не интересен. Он всегда предполагал влечение по отношению к себе, он всегда был убежден, что он нравится женщине, даже когда это было не так. Девушкам он был не интересен, но он этого и не замечал, думая что он Бред Пит, и поэтому у него не было причин для того что бы покинуть девушек. В конце концов, эта его непоколебимая уверенность в себе начала нравится девушкам и он получил женщину, которую хотел.

А сейчас, давайте представим себе другую компанию девушек, которые сидят в другом ресторане. К ним так же в течение вечера, подходили знакомиться два парня. Первый парень, сразу после знакомства, обнаружил, что он им не интересен. Он был умный парень, он отлично видел их реакцию, он знал, что необходимо делать для того, что бы разогреть их. У него всё получилось, и он уехал с девушкой, которую он хотел. Позднее, с компанией девушек попытался познакомиться второй парень. В силу своей тупости, он даже не понял, что он им не особо нравится. Он что-то бормотал и вёл себя странно. Сильный фрейм, конечно, может произвести впечатление на некоторых людей, но полный отрыв от окружающей действительности и неумение получать обратную связь это не очень хорошо. Он был полностью убеждён в том, что то что он делает это клёво, он не замечал реакции девушек, и, в конечном счёте, они ушли от него. Очевидно, что сильные и слабые стороны есть у обоих типов парней.

Но существует третий, наиболее редкий тип парней. Они сочетают в себе как умение делать калибровку и вносить поправки в своё поведение, так и сильный фрейм. Эти парни знают о существовании слепых пятен, борются с ними, развивают свои социальные навыки и в то же время они их не приводит в замешательство, та негативная обратная связь, которую они могут получить. Получая негативную обратную связь, они вносят поправки в своё поведение в то же время на эмоциональном уровне они никак на это не реагируют, не теряют своего фрейма. Им не нужна какая-то положительная реакция со стороны окружающих для того, что бы чувствовать уверенность в себе, в то же время это не «бредовая уверенность». Если они получают какую-то отрицательную реакцию на свои действия, они её видят, меняют своё поведение, никак при этом, эмоционально не реагируя, и не прибегая к использованию рационализации. Они одновременно и непоколебимы в своей уверенности в себе и обладают социальными навыками. Немногие достигают подобного уровня развития. Это идеал, к которому нужно стремится.

\chapter{Опоры реальности}

В социальной динамике, способ, с помощью которого мы можем понять будем ли мы чувствовать уверенность или беспокойство в критической ситуации это обратить внимание на то где находится фокус вашего внимания. Где находятся ваши мысли и восприятие? Внутри вас или снаружи? Когда мы чувствуем тревогу, мы склонны концентрироваться на себе, мы пытаемся управлять нашими эмоциями. Мы ставим своё поведение, свои эмоциональные реакции в зависимость от поведения или эмоциональных реакций других людей. Но когда мы чувствуем уверенность в себе, чувствуем собственную ценность, то мы можем концентрироваться на чём-то внешнем. Нам нет необходимости пытаться концентрироваться внутри себя и управлять собой. Как результат, мы кажемся более естественными и искренними, существуя в настоящем моменте.

Это как спортивная команда, которая вначале доминировала на поле, а затем потеряла нить игры. Пытаясь вернуть инициативу, игроки, действуя реактивно, «сбиваются в навал» и как результат, всё становится только хуже. Но если они вновь приобретут хладнокровие и просто начнут играть в собственную игру, тогда они снова вернут себе контроль над игрой. Ирония здесь состоит в том, что когда мы пытаемся повлиять на кого-либо, пытаемся продемонстрировать более высокую ценность, тем самым мы демонстрируем низкую ценность. Когда мы концентрируемся на себе, на наших внутренних переживаниях, пытаемся контролировать себя то окружающие почувствуют это. Они почувствуют нашу нуждаемость и неискренность. Когда человек пытается концентрироваться на себе, пытается управлять своим поведением в данный момент, то произойдёт следующее. Он не будет способен продолжительное время сохранять глазной контакт, он будет моргать, его мысли будут путаться, голос будет дрожать, он не будет излучать энергетику, он будет долго колебаться, прежде чем пойти на физический контакт с кем-либо. Когда же фокус внимания человека направлен на окружающие вещи, то он не будет выглядеть как реагирующий на других людей. Он будет чувствовать уверенность, он может говорить всё что угодно и окружающие воспримут это нормально, даже если это не так, то его это не будет беспокоить. Он может спокойно концентрироваться на внешнем окружении, поскольку у него отсутствует внутренний диалог. У него будет нормальный голос, глазной контакт, и язык тела. Его речь будет звучать естественно, и соответственно окружающей обстановке. У него будет хороший контакт с окружающими его людьми, поскольку ничего его не отвлекает и не рассеивает его внимание.

Если вы, концентрируетесь на том, что вас окружает, то скорее всего вы будете испытывать:

--- Чувство постоянного комфорта.

--- Комфорт вне зависимости от реакции на вас других.

--- Комфорт выслушивать других и быть внимательным по отношению к окружающим.

--- Комфорт при занятии социального пространства, при вашем прикосновении к другим, и при прикосновении к вам других людей.

--- Чувство комфорта, когда вы шутите и дразните окружающих.

--- Чувство комфорта когда вы прямой и искренний с самим собой, о чём бы вы не думали.

--- Комфорт принимать советы от других людей.

--- Комфорт быть самим собой.

--- Комфорт в любом окружении.

--- Комфорт когда вы находитесь в центре внимания окружающих.

--- Комфорт не «ковыряться» в прошлом и «разбираться» с вещами по мере их возникновения в вашей жизни.

--- Комфорт в общении с каждым человеком, как со своим хорошим другом.

--- Комфорт с собственной сексуальностью.

--- Комфорт позволит вам открыться и продемонстрировать свою личность.

--- Комфорт позволит вашим эмоциям управлять вами таким образом, что вы будете, говорит лучшее, на что вы способны.

--- Предположение что то, что вы говорите, будет хорошо воспринято.

--- Предположение что вы клёвый.

--- Предположение что другие люди вами восхищаются и хотят общаться с вами.

--- Предположение что вы самый клёвый чувак в любом месте.

--- Предположение, что вы привлекательны для большинства женщин, и даже если вы обнаружите что это не так, то ничего страшного.

--- Предположение, что множество людей хотят общаться с вами, а женщины хотят встречаться с вами.

--- Безразличие к потере раппорта с другими людьми.

--- Безразличие к тому, одобряют ли другие люди ваше поведение.

Если вы концентрируетесь на себе, то скорее всего вы будете испытывать:

--- Чувство того, что вы должны постоянно контролировать себя, и думать о том, как ваше поведение выглядит в глазах окружающих.

--- Оставаясь сконцентрированными на себе, вы не сможете быть внимательным к поведению окружающих.

--- Чувство того, что вы должны постоянно приспосабливаться к другим.

--- Чувство что вы должны впечатлять и развлекать окружающих.

--- Чувствовать себя обязанным делать что-то, и волноваться из-за этого.

--- Чувствовать что «либо сейчас, либо никогда». Что если вы сейчас не сделаете что-то, то у вас больше не будет другого шанса.

--- Чувство что вы не заслуживаете красивых женщин, что вы не сможете привлечь красивую женщину.

--- Страх того, что во время общения, вы ляпните какую-нибудь глупость.

--- Чувство страха быть осуждаемым со стороны окружающих.

--- Страх показаться скучным.

--- Страх показаться не искренним.

--- Страх продемонстрировать свою страсть, уверенность или сексуальность.

--- Страх потерять самообладание, поскольку другим людям может не понравиться это.

--- Страх того, что вы всегда должны «навязываться» другим людям, иначе вы им не понравитесь.

--- Страх что вы должны развлекать каждого, потому что если вы не будете этого делать, то вы не понравитесь ему.

--- Страх вести себя как человек с более высокой ценностью, чем та, которая как вы думаете, у вас есть.

--- Страх, что окружающие вас осудят за то, что вы знакомитесь с женщиной.

--- Страх разорвать раппорт с другим человеком.

--- Страх быть отверженным. Страх неудачи.

--- Страх того, что если вы примите «обратную связь», советы со стороны окружающих, относительно вашего поведения, то вы уроните свой статус.

--- Страх того, что если другие восхищаются вами, то они делают это неискренне.

--- Страх того, что если вы понравились какой то женщине, то вы должны всё тщательно обдумывать и не совершить ошибку при общении с ней. Потому что, если вы её потеряете, может пройти ещё много времени, прежде чем вы понравитесь другой женщине.

Если парень концентрируется на том, что его окружает это означает что его реальность поддерживается какими то внутренними опорами, поэтому ему нет необходимости концентрироваться на себе. Если же парень концентрируется на себе то скорее всего его реальность поддерживают внешние опоры, он всё время пытается заставить людей полюбить себя.
Итак, что такое внешние и внутренние опоры?

Для того, что бы понять это, мы должны создать определённые контрольные точки, которые будут служить нам как «опоры реальности».

Если чувство реальности парня основывается на том, как другие люди реагируют на него, он всегда беспокоиться что о нём подумают окружающие, ему необходима положительная реакция на себя окружающих, для того что бы чувствовать себя хорошо. Если кто-то скажет такому парню «Ты неудачник», то это окажет влияние на его самооценку, он подумает «Да, я действительно неудачник». Если девушка отвергнет его, он подумает «Я просто недостаточно хорош, чтобы получить такую девушку», т. е. любой отказ, любая негативная реакция на него сказывается на его чувстве уверенности в себе, снижается ожидание того, что на его поведение окружающие отреагируют позитивно.

После подобного эмоционального удара, он уходит в себя, начинает «копаться» в себе. Реальность других людей становится его реальностью, он присваивает себе те качества, которыми наградили его окружающие, становясь всё более неуклюжим и застенчивым. Он не будет способен нормально общаться до тех пор, пока кто-либо не скажет ему комплимент, не подчеркнёт какие-либо его достоинства, это выведет его из подавленного состояния, придаст ему уверенности в себе, но только до тех пор, пока кто-нибудь из окружающих вновь не «заморит» его.

Часто бывает так, что парень, реальность которого построена на одобрении со стороны других имеет больший успех, чем робкий парень, поскольку первый более заинтересован в том, что бы вести себя клёво.

Он знает как вызвать восхищение со стороны окружающих, как вести себя более проактивно и творчески, чем средний парень, но часто, его потребность в одобрении со стороны окружающих, становится для них очевидна, и он терпит неудачу. Он нуждается в одобрении, в социальной энергии, для того, чтобы продолжать вести себя клёво. Если другой парень одет более клёво и привлекает к себе внимание окружающих, он считает что должен высмеять это. Если этот другой парень пришёл со своей подружкой, он начинает «нападать» на эту девушку, чтобы получить её внимание. Если другой парень становится центром внимания для окружающих, он считает что должен переключить внимание на себя, или начать разговаривать с другими людьми или вообще выйти из помещения. Для него не представляют ценности моногамные продолжительные отношения с одной девушкой, поскольку у него существует потребность постоянно получать восхищение со стороны новых женщин.

Парень, чья реальность держится на внешних опорах, должен постоянно создавать слепые пятна, которые помогают ему «объяснять», почему то, что он делает, иногда не работает. С течением времени, он начинает обнаруживать новые факты, которые уничтожают его старые объяснения (рационализации), и тогда он вынужден создавать новые рационализации, для того, что бы защитить свою реальность. Наслаиваясь одна на другую, эти рационализации могут свести его с ума. Ирония здесь состоит в том, что те пути поступления информации из внешнего мира, которые он закрыл, для того, что бы избежать болезненных переживаний, являются теми самыми путями, открыв которые, он мог бы многому научиться.

Если такой парень встречает человека с более развитыми навыками, более успешного, то он может осознать недостаточную собственную компетентность. Вместо того, что бы понять чем именно этот другой парень лучше, попытаться чему-то научится у него, он начинает выискивать его недостатки, концентрирует своё внимание на каких то «объективных», преимуществах, которые как ему кажется, имеет другой парень, он будет стараться «приопустить» более успешного парня. Таким образом, как ему кажется, он сможет почувствовать собственное «превосходство» над другим парнем. Такие парни обычно ненавидят тех, кто лучше и успешнее их. Концентрируясь на отрицательных сторонах других людей, выискивая мнимые «объективные» преимущества, которые они имеют перед ним, он сохраняет в неприкосновенности собственные ограничивающие убеждения. Вместо того, что бы работать над устранением своих недостатков, он занимается рационализацией, чрезмерно упрощая положение вещей и испытывает удовольствие от этого. Это как бизнесмен, у которого нет ясного понимания того, в каком направлении развивать свою компанию, как строить бизнес. У него просто нет нужных черт характера для ведения бизнеса. День за днём. Такой бизнесмен протирает штаны в своём офисе, встречаясь с кучей людей. Если случаются какие-то неприятности, то он становится нервным, эмоционально взрывается, всякий раз когда это происходит. Он постоянно обвиняет других людей в своих неудачах, страдает маниакальной подозрительностью, тратит своё время и энергию на бесполезные споры, оскорбления и обвинения, вместо того, что бы развивать свой бизнес. Он думает что оскорбляя и унижая других, он сможет решить свои проблемы и даже получает определённое удовольствие от маленьких побед в словесных перепалках. Окружающие же его люди, постепенно учатся не обращать внимание на подобные истерики и начинают чувствовать моральное превосходство над ним.

Конечно, временами, подобные вспышки гнева могут иметь под собой основание. Но в конце концов подобное поведение разрушит его жизнь. Это как уверенный в себе парень получает позитивную реакцию от окружающих, просто сам будучи позитивным и ожидая этого от других так и наш бизнесмен своим отношением, ожиданием гадости со стороны других, пробуждает в людях худшие черты их характера и это становится самоисполняющимся пророчеством. Такой человек постоянно видит в окружающих только негатив и сам страдает от этого. Проблема такого человека заключается в том, что его реальность имеет внешние опоры, которые абсолютно ненадёжны. Это как биржевой брокер, который потерял все свои деньги и покончил жизнь самоубийством. В сущности это был неплохой парень, просто в определённый момент удача отвернулась от него, многие люди работали вместе с ним на одной торговой площадке, даже в его худший день. Но его реальность основывалась на его успешности в биржевой игре, когда он проиграл всё, его реальность рухнула, ему незачем было больше жить.

Или как парень, который впал в депрессию, после разрыва со своей девушкой. До того, как он с ней познакомился, он счастливо жил уже много лет на этой земле, и после разрыва отношений, ничто не мешало ему предполагать, что он будет так же счастлив в будущем. Но он построил свою новую реальность на отношениях именно с этой девушкой, когда она ушла, его жизнь показалась ему пустой. Общего в этих парнях то, что они не построили для себя внутренние опоры, смысл их существования заключался во внешних обстоятельствах.

С другой стороны, есть парни, чья реальность поддерживается внутренними опорами, и что бы не происходило в их жизни, во внешнем мире, они чувствуют себя в безопасности. Они доверяют самим себе, и всегда поступают так, как они считают нужным, вне зависимости от того, понравится это другим или нет. Они знают, что от того что они не продемонстрируют окружающим свои лучшие качества, эти качества не перестанут существовать. Они не присваивают слепо ценности общества, а сами определяют, что является важным, а что нет. Они верят в то, что обладают высокой ценностью, вне зависимости от того, признают это другие или нет. Такой человек знает, что все люди, вне зависимости от статуса, могут делать ошибки. Если человек имеет высокий статус, то он уважает его, учится у него, но знает, что никто не застрахован от ошибок, у каждого есть свои «косяки». Неважно насколько клёвый другой человек, насколько хорошо одет или имеет высокий статус, он никогда не думает что он хуже него. Кроме того, он может так же иметь и некоторые внешние опоры для своей реальности, например, любовь близких ему людей, социальные навыки, с помощью которых он может понравиться людям в будущем, определённые поверхностные вещи, такие как внешность и навыки. Всё это он ценит, но он не нуждается в них, для того, чтобы чувствовать себя в порядке. Это дополнительные опоры, которые стабилизируют всю конструкцию, но они не абсолютно необходимы, в крайнем случае ими можно пренебречь.

Обладая внутренними опорами, можно ценить и внешние, без страха потерять их. В то же время, если у парня чувство реальности чересчур поддерживается внутренними опорами, это может принести ему определённые проблемы. Он становится слишком самоуверенным, беспечным, его перестают волновать какие-либо проблемы, он не чувствует желания добиваться чего-либо, поскольку он думает, что это ему не нужно, он и так прекрасно живёт\ldots

Поэтому необходим определённый баланс между внешними и внутренними опорами. Кроме того, у вас должно быть убеждение, что вы обладаете ценностью, вне зависимости от того, признают ли это другие. Большинство людей имеют как внутренние, так и внешние опоры для своей реальности. Так же для большинства людей характерно в разное время больше использовать то внешние опоры, то внутренние.

Вопрос состоит в том, может ли человек найти правильный баланс. Обе эти стороны нашей психологии мы можем использовать успешно, если мы будем поддерживать умеренность и избегать излишне опираться на внутреннее или внешнее. Конечно немногие люди смогут достигнуть правильного баланса, но это идеал к которому надо стремиться.

\chapter{Черты характера и состояние (состояние и восприятие)}

Для того что бы управлять женщиной, вы должны уметь управлять собой. Мы все несём ответственность за состояние нашего разума. Парень, который имеет сильные внутренние черты характера будет способен управлять состоянием своего разума. Парень, со слабыми чертами характера, будет нести ответственность перед самим собой, за все те неприятные вещи, которые происходят в его жизни. Мы проецируем своё внутренне состояние во внешний мир и влияем на людей вокруг нас. Настроения заразительны, если вы хотите что бы люди вокруг вас были в хорошем настроении, то вы сами должны пребывать в таком настроении. Наши состояния основываются на нашем чувстве реальности, мы должны уметь программировать себя так, что бы поддерживать в себе положительное настроение день за днём. Давайте рассмотрим как это происходит.

С течением времени, мы учимся получать доступ к наиболее важным состояниям сознания. Чем более часто мы входим в определённое состояние сознания, тем легче в будущем мы вновь сможем получить доступ к этому состоянию. Со временем, процесс вхождения в определённое состояние станет обычным делом, нам не нужно прилагать какие то усилия. Это могут быть как положительные так и отрицательные состояния, они могут нести нам как хорошие так и плохие эмоции, но всегда это будут привычные нам состояния. Часто наш разум склонен застревать в каких-то мыслительных шаблонах и замкнутых циклах. Например, если мы чем-то огорчены, то мы можем рассуждать следующим образом «Я чувствую себя хуево. А почему? Потому что произошло x, y, z\ldots А если в будущем произойдёт что-то подобное, то я буду чувствовать себя ещё хуже. Как же мне хуёво. Почему? Потому что произойдёт x, y, z\ldots» Например, многие люди бывают одержимы так называемой «дорожной яростью», и зная об этом, они часто сознательно создают конфликтные ситуации, где они бы могли выплеснуть свой гнев на других участников дорожного движения. Логически, они понимают что они не могут контролировать других участников дорожного движения, но эмоционально, озлобленность на дороге является для них привычным состоянием. Когда они стоят в пробке, они начинают орать что бы впереди идущая машина уступила им дорогу, прекрасно при этом понимая, что даже если эта машина уступит дорогу, то за ней будет следующая и пробка от этого никуда не денется. То же самое может происходить с трудоголиками. Они всегда куда то торопятся, у них всегда есть какие то незаконченные дела, которые необходимо срочно доделать. То же самое происходит с парочками, которые любят поскандалить друг с другом. Если они уже продолжительное время живут в мире и спокойствии то они испытывают потребность инициировать ссору что бы поскандалить между собой, потому что без ссор их жизнь кажется им неполноценной.

Такие люди всегда прибегают к обратной рационализации, для того что бы найти объяснение своим чувствам. Раз за разом, в беседе, они будут возвращаться к темам, которые отражают их внутреннее состояние. Они будут рассказывать о своих проблемах, которые как им кажется, создают для них другие люди. Они будут сплетничать об окружающих, о том, сколько беспокойства доставляют им окружающие их люди.

Они путают причину со следствием. Во-первых: они испытывают склонность постоянно возвращаться в привычное для них состояние, а затем, пребывая в этом состоянии, они ищут способ как они могут «объяснить» (рационализировать) то, что они чувствуют. Во-вторых: погружаясь в определённое состояние они будут вспоминать определённые события в прошлом, которые связаны с этим состоянием и те чувства. Которые они тогда испытывали, что ещё больше будет усиливать их текущее состояние.

Они верят в то, что какие то события в окружающем мире заставляют их чувствовать себя подобным образом. Фактически, когда люди начинают вспоминать вещи, которые причиняли им боль в прошлом, то в настоящем они будут испытывать те же самые эмоции. Эта одна из наиболее распространённых причин, почему люди могут впадать в депрессию. Это замкнутый круг, они впадают в привычное состояние, вспоминают события в прошлом и те чувства, которые связаны с этим состоянием, и это даёт им возможность находить «объяснения», почему они чувствуют себя подобным образом, состояние усиливается и т. д.

От ощущения беспомощности или злости многие люди впадают в депрессию и не могут найти выход из неё. Они настолько «накручивают» себя собственными эмоциями, что впадают в состояние беспомощности и оцепенения и чувствуют себя неспособными что-либо изменить. Постоянная зацикленность на проблемах может быть способом снять с себя ответственность за них. Мы сейчас не говорим о каких то психических заболеваниях, просто депрессию многие используют как предлог для оправдания собственного отказа нести ответственность. Жить определённо становится намного легче когда вы убеждены что в том, что с вами происходит, нет вашей вины. Человек с «ментальностью маленького ребёнка» будет думать что с ним поступили несправедливо, всякий раз, когда он не получил того, чего он хотел. Он всегда будет избегать чувствовать собственную ответственность.

Часто в процессе воспитания нам могут прививать ложные убеждения. Многие из нас верят в то, что поступая согласно этим убеждениям, мы всегда будем получать нужный нам результат. Но впоследствии часто мы будем сталкиваться с тем, что убеждения, которые мы пробрели в процессе воспитания не отражают положения вещей в реальном мире. Мы можем испытывать соблазн эмоционально реагировать на складывающиеся обстоятельства, отказываться принимать их, думать о том как же это несправедливо, вместо того, что бы просто действовать отталкиваясь от складывающейся ситуации.

Мы можем попасть в ловушку «ментальности маленького ребёнка», рассуждая о том, что справедливо, а что нет. Мы должны чётко понимать что с одной стороны мы не всегда можем контролировать внешние обстоятельства, а с другой стороны, мы всегда несём личную ответственность за результат нашей деятельности. Вы не можете выбирать свою внешность или семью, в которой вам суждено было родиться. Однако, несмотря на это, всегда есть человек, который имел худшие стартовые условия чем вы, и тем не менее, добился в жизни больших успехов. Если с течением времени человек не хочет изменяться, чувствуя себя комфортно со своими старыми убеждениями, то тем самым он сам готовит себе будущие разочарования и потери. Многие могут думать, что это несправедливо, что они не могут иметь постоянную девушку «прямо сейчас». Они жалуются, что они не могут найти «качественную» женщину, которой бы они могли доверять и которая не обманывала бы их и не бросила бы их. Но мужчина, который ищет в женщине определённые качества, за которые он бы мог её полюбить, отличается от мужчины, который ищет в женщине определённые качества, и которыми он бы мог управлять.

Только слабые мужчины отказываются признавать, что у их подружек имеется гораздо больше возможностей быстро создать новые отношения, если вдруг настоящие будут разрушены. Они отказываются признаться самим себе в том, что если бы у них был выбор среди женщин, то они бы думали по-другому. Единственная причина по которой они так отчаянно цепляются за постоянные отношения это то, что они не чувствует с себе уверенности в том, что они быстро найдут новую подружку, если вдруг их настоящая девушка бросит их. Легко обвинять во всем женщину. Неудачник всегда найдёт возможность обвинить других. Но в итоге, мы всегда несём личную ответственность за наше эмоциональное состояние. Развиваясь, мы формируем качества, которые позволят нам преодолеть любые внешние обстоятельства.

Парень который верит в себя, знает что только он несёт ответственность за всё что происходит в его жизни, он знает каков он в настоящем и как он сможет улучшить себя в будущем. Он верит в то, что он может «прогнуть» под себя окружающий мир, даже если в настоящем это не так, он знает что он изменится и в будущем обязательно сделает это. Парень который верит в себя не реагирует на внешние воздействия, он не ищет «оправданий» для своих проблем в каких-то отрицательных внешних воздействиях и никак на них эмоционально не реагирует. Парень, который ведёт себя проактивно, знает что он имеет власть над состоянием своего разума и несёт за это ответственность. Реагирующий парень возлагает ответственность за состояние своего разума на внешние обстоятельства.

Это два совершенно разных образа мышления. Есть люди, которые делают то, что они хотят, в противоположность людям, которые постоянно колеблются и сомневаются. Успешные люди всегда будут видеть ситуацию, концентрироваться на ней и знать как с этим справиться. Или они могут даже не сосредотачиваться, а просто делать. Нерешительные люди будут долго обдумывать ситуацию, сосредотачиваться на каких-то отрицательных моментах, размышлять о том, что может пойти не так, и как хреново может им быть, если что-то пойдет неверно и в конце концов они чувствуют себя обессиленными собственными мыслями. В то время как один беспокоится о том, что могут подумать о нём окружающие и сидит на месте, другой парень доверяет своим инстинктам и просто весело проводит время, ни о чём не беспокоясь. Если он видит то что он хочет, он немедленно просто подходит и берёт это. Он просто веселится и не выглядит таким странным, как другие парни, которые излишне серьёзны и заставляют женщину чувствовать напряжение, придавая слишком большое значение тому, что она не реагирует на них так, как они этого хотят. Такой парень просто выражает свою личность, он конгруэнтен своему поведению и в любом случае девушка его не осудит. Она будет смеяться, и даже если она отвергнет его, всё равно она будет считать, что этот парень обладает привлекательными качествами.

Нерешительный парень не сможет так себя вести. Он будет воспринимать всё слишком серьёзно, вместо того, что бы просто весело проводить время общаясь с женщинами, демонстрируя свою личность. Он будет размышлять о том, как ему избежать возможных неприятных ситуаций и будет всё время нервничать.

Часть его ментальной энергии сконцентрировано на себе самом, внутри него и это ухудшает его способность адекватно и быстро реагировать на события в окружающем мире. В определённый момент своей жизни реагирующий парень принял решение что он не может контролировать окружающие его обстоятельства и сделает всё, что необходимо, для того, что бы избежать неприятных переживаний. Он даже и нее предполагал, что в этот момент он сделал определенный выбор. Но он его сделал.

Он не пытается постоянно улучшать свои навыки, общаясь с женщинами. То как он себя будет чувствовать он поставил в зависимость от того, как на него реагируют. Он подойдёт к женщине только в том случае, если будет уверен, что все пройдёт хорошо.

\chapter{Слабое поведение}

Образ мышления и поведение тесно связаны друг с другом. Ваши мысли всегда находят отражение в вашем поведении, даже если вы этого не замечаете. Ваша сила и ваша слабость всегда будет проявляться. Женщины очень чувствительны к таким, иногда еле заметным, особенностям поведения. Женщина будет оценивать вашу ценность на основании вашего поведения, на основании того, как другие люди реагируют на вас, и на основании собственных ощущений. Вы можете сознательно контролировать своё поведение, для того что бы демонстрировать ценность. Поступая таким образом, вы будете получать обратную связь, которая позволит вам улучшаться.

Со временем, вы сможете усвоить модели сильного поведения, и это станет вашей реальностью. В социальной динамике, не делать неправильных вещей так же важно, как и делать правильные вещи. Одной из самых худших ошибок, которые может сделать человек и которые снизят его ценность, является «квалифицировать себя». Когда мы говорим «квалифицировать себя», это значит мы стремимся логически убедить других людей увидеть нашу ценность, но это даёт обратный желаемому эффект. Мы вредим себе.

Существуют хорошо опознаваемые образцы поведения, по которым можно судить о том, что люди квалифицируют себя. Изучение подобных образцов поведения поможет нам исследовать собственное поведение на предмет присутствия там попыток «квалифицировать себя». Но часто слишком старательная попытка анализировать своё поведение может принести больше вреда, чем пользы. Потому что, когда вы излишне заняты анализом, то ваше естественное поведение будет страдать.

Следует так же отметить, что поведенческие модели нельзя рассматривать как какую-то точную науку. Если мужчина конгруэнтен, он может иметь множество признаков слабого поведения и тем не менее, это не будет для него проблемой. Основное Правило является следующим: «Любое поведение, которое выглядит нуждающимся --- это не клёво». Однако, если поведение, о котором можно сказать «нуждающееся» происходит с позиции силы, то оно тоже может быть клёвым. Метасообщения, которые следуют за поведением имеют гораздо большее значение, чем поведение само по себе.

Наблюдение за другими людьми в различных ситуациях поможет вам понять эти вещи. Наблюдайте так же за собой. Человек, который просто механически пытается в точности следовать материалу, без сознательного понимания как это работает, будет выглядеть неестественно. Часто, многое зависит от вашего состояния. Если вы в ресурсе, вы можете нарушать правила и тем не менее иметь успех. Просто используйте свой здравый смысл. Есть 4 основных способа продемонстрировать низкую ценность:

--- демонстрация нервозности;

--- демонстрация нуждаемости;

--- усердное старание получить одобрение (быть принятым);

--- усердное старание продемонстрировать высокую ценность.
Демонстрация нервозности

Если вы чувствуете нервозность, то вы можете чувствовать как бешено колотится ваше сердце, потеют ладошки, и пересохло в горле. Вы будете пытаться «разрядиться» выпуская свою нервную энергию в окружающий мир, демонстрируя слабые виды поведения. Например:

--- Говорить слишком быстро. Вам может казаться, что быстро сказав ещё одну, на ваш взгляд клёвую вещь, вы получите признание со стороны окружающих.

--- Излишне жестикулировать, размахивать при разговоре руками. Это может быть истолковано окружающими как ваша отчаянная попытка привлечь к себе внимание. Быть энергичным не плохо, плохо быть «энергичным» когда это происходит от дискомфорта. Если вы обнаружите что размахиваете руками, излучая нервную энергию, то лучше будет вам прекратить это делать и просто опустить руки вдоль тела.

--- Нервно ходить вперёд-назад.

--- Суетится.

--- Говорить слишком много, прыгать с темы на тему, отчаянно пытаясь поддержать разговор.

--- Излишне ограничивать себя в жестах и позе, пытаясь занимать как можно меньше социального пространства. (closed umbrella body language).

--- Стоять с прижатыми друг к другу ногами. Это может быть истолковано окружающими, как будто вы боитесь занимать социальное пространство.

--- Сидеть в неудобной позе.

--- Держать плечи в напряжении.

--- Дрожание в голосе, говорить слабым голосом.

--- Говорить невнятно, бормотать.

--- Замереть, длительное время не менять позу, иногда даже переставая моргая.
Демонстрация нуждаемости

Если вы чувствуете нужду, то вы будете всегда стремиться строить неестественный раппорт с окружающими вас людьми. Вы думаете, что вы ведёте себя дружелюбно, но в сущности, вы слишком дружелюбны. Это можно заметить по многим вещам: как вы в принципе себя ведёте, как вы реагируете на людей которые с вами разговаривают, как вы реагируете, когда девушка уходит от вас. Например:

--- Наклонятся вперёд или «клевать» носом. Выглядеть как будто этот разговор очень важен для вас. Если вы заметили что вы наклонились вперёд, то лучше будет вам немедленно отклониться назад, что бы собеседник наклонился в вашу сторону. Делайте это даже если у вас высокий рост или в месте, где вы находитесь, играет громкая музыка.

--- Слишком быстро проявлять внимание, когда кто-нибудь заговорит. Слишком быстро поворачивать голову или тело по направлению к человеку всякий раз, когда он заговорит. Вы будете выглядеть как слишком нуждающийся в другом и реагирующий. Все в порядке, если вы просто неспешно повернёте вашу голову, но если вы это сделаете резко, то вы будете выглядеть нуждающимся.

--- «Поддакивать» другому, кивая головой и во всём соглашаясь, постоянно восхищаться собеседником, удивлённо приподнимая бровь, что бы он не сказал. Быть дружелюбным не означает, что вы во всём всегда согласны и на всё говорите «Да». Если ваш собеседник говорит что-то действительно стоящее, то вам не обязательно всегда кричать «Да, КЛЁВО, Классно», вместо этого вы можете помолчав сказать «Ну в общем звучит неплохо», или «Возможно, ты прав».

--- Слишком стремиться услышать то, что говорит другой человек. Если вы не можете расслышать что говорит ваш собеседник (например вы находитесь на шумной дискотеке, где играет громкая музыка) и вы постоянно его переспрашиваете «Что?.. Что?», вы демонстрируете нуждаемость и неуклюжую попытку построить раппорт. Вместо того что бы его постоянно переспрашивать, просто смените тему разговора, так вы не будете казаться нуждающимся. Или вы можете просто использовать рефрейминг, и представить ситуацию так, как будто он вас развлекает. Просто скажите уверенным голосом «Чувак, повтори, я очень хочу расслышать что ты говоришь».

--- Страстно желать поддержать разговор, отвечать на глупые вопросы. Например, если кто-либо спрашивает вас, для того, что бы вы квалифицировали себя перед ним, говоря что-то вроде этого «Чувак, почему ты меня спрашиваешь об этом?», вам не обязательно отвечать на этот вопрос логически и прямо. Можете ответить любым бредом, или просто проигнорировать вопрос, если вы не желаете на него отвечать.

--- Запоминать слишком много деталей прошлой беседы. Если вы помните слишком много мелких деталей прошлого разговора, вы метасообщаете другому человеку, что воспринимаете его как имеющего более высокую ценность, чем ваша. Если ваш собеседник в прошлый раз рассказывал действительно какие то поразительные вещи, то тогда всё нормально. Если же при следующей встрече вы помните слишком много деталей обычного разговора, это означает что этот человек имеет для вас высокую ценность, и тот разговор значил для вас гораздо больше, чем для него.

--- Поиск раппорта --- это отстой. Иногда разговаривая с кем-либо, вы сможете заметить, что вы пытаетесь впечатлить собеседника в процессе разговора. И если вы не добиваетесь этого, то вы будете испытывать соблазн говорить всё больше и больше, надеясь что в конце концов вы получите нужный результат и поразите своего собеседника чем-то необычным. Но чем больше вы будете говорить, тем меньше вы его «зацепите». Если вы заметили за собой подобную вещь, то просто остановитесь.

--- Ждать людей, которые ушли от вас. Если девушка говорит вам «Мне нужно пойти в дамскую комнату», часто это означает что она и не планирует к вам возвращаться. Даже если она и вернется, а вы будете сидеть и ждать её как маленькая глупая собачка, то вы потеряете ценность. Лучше всего в этот момент начать разговаривать с кем-либо ещё, что бы в тот момент когда она вернётся, она увидела что вы весело проводите время. В момент её возвращения вы сможете продолжить с ней беседу.

--- Следовать вместо того что бы вести. Если вы всегда следуете за кем-либо, то от вас этого будут ожидать постоянно. Если девочка отходит от вас куда-то, то не следуйте за ней, за исключением случаев, когда вы думаете что она не собирается возвращаться. Если вы думаете что она не собирается к вам возвращаться, то вы можете пойти вместе с ней, вам всё равно нечего терять. Однако если вы видите что девочка заинтересована в вас. То вам нет необходимости идти вместе с ней, просто чувствуйте себя уверенно и через некоторое время она вернётся. Однако, если очевидно что девушка находится в вашем фрейме и просто тащит вас за собой что бы познакомить со своими друзьями или хочет показать что интересное, то в этом нет проблемы.

--- Не понимать язык тела. (Нуждающийся язык тела) Когда при разговоре ваш собеседник с помощью языка тела ещё не продемонстрировал, что он заинтересован в раппорте с вами, а вы с помощью своего языка тела дали понять что он вам интересен, то тем самым вы создали ситуацию при которой его ценность выше, чем ваша. Направление ваших ступней часто показывает где находится человек, имеющий ценность для вас. Если человек при разговоре, с помощью языка тела не показал первый что он заинтересован в раппорте с вами, то и вам не стоит этого делать. Вы даже можете просто уйти. Однако если вы продемонстрируете при разговоре свои ценные качества и заинтересуете собеседника, то вы сможете заметить, что изменится его язык тела, он попытается к вам приспособиться. (Подстроиться).

--- Преследовать людей, когда они уходят. Если человек уходит от вас, и вы начинаете за ним гнаться то тем самым вы заставляете его ещё больше захотеть избавиться от вашего присутствия. Если вы спрашиваете «Ты вернёшься?» или «Куда ты идешь?» то вероятнее всего это их не остановит, поскольку вы делаете то же самое, что будет делать в этой ситуации любой скучный парень. Вместо того. Что гнаться за девушкой просто оставайтесь на месте и начинайте говорить громче, для того что бы привлечь её внимание (обычно используя юмор). Если вы что-то громко выкрикните, то скорее всего женщина остановится и будет смотреть на вас, с целью понять преследуете ли вы её или нет. Если она поймет что вы не преследователь, то скорее всего она развернётся и подойдёт к вам. Вы должны сделать это таким образом, что бы казалось что вы даже и не понимаете что она собирается уходить. Если она не будет рассматривать вас как преследователя, то вы не снизите собственную ценность относительно её ценности. Если вы всётаки решите идти за ней то вам необходимо быстро заставить её смеяться либо каким-то другим способом завладеть её вниманием, тем самым вы сможете компенсировать свою утраченную ценность.

--- Проявлять к кому-то, кого вы не знаете, больше внимания, чем к вашим друзьям. Если вы с кемлибо разговариваете и в этот момент к вам подходит ваш друг, большинство парней просто кивком головы с ним поздоровается и продолжит разговор со своим собеседником. Если вы ведёте себя таким образом, то это сигнализирует что тот человек, с которым вы разговариваете, имеет для вас высокую ценность и вы ищите раппорта с ним. Вам необходимо тепло поприветствовать вашего друга и ввести его в разговор. То же самое применимо и к разговорам по мобильному телефону. Если в момент беседы ваш телефон зазвонит и вы его выключите, не желая отвечать на звонок и тем самым прерывать беседу с человеком, то вы сигнализируете ему о его высокой ценности. Если в этот момент вы заняты чем-то действительно важным, то всё нормально. Если же вы делаете это таким образом, что очевидно что вы боитесь потерять нить разговора с вашим собеседником, то вы демонстрируете нужду. Если во время разговора вы способны отвлекаться на посторонние моменты, то тем самым вы демонстрируете что вы чувствуете себя комфортно.

--- Постоянно пытаться подтверждать, что другие люди помнят то, о чём вы с ними договаривались. Например, если вы снова и спрашиваете девушку «Ты не забыла, мы идём во вторник в кино?», то тем самым вы метасообщаете, что в прошлом вас часто «обламывали» и сейчас вы не уверены в том, что она в последний момент не откажется.

Усердное старание заслужить одобрение (быть принятым). Нормальная беседа имеет определённый ритм. Есть определённое соотношение, сколько и когда каждый участник вкладывает в беседу для того что бы она продолжалась.

Например смех, является не только способом снять напряжение, но так же является индикатором того, что в группе существует определённая аура. Когда один человек прикладывает слишком много усилий для того, что бы поддержать беседу, то произойдут одновременно две вещи.

Во-первых, --- чувствуя, что окружающие перестают принимать его поведение он будет пытаться отчаянно вернуть их одобрение.

Во-вторых, --- пытаясь вернуть одобрение со стороны окружающих он будет говорить всё больше и больше и тем самым разрушит нормальный ритм беседы. Он будет выглядеть неестественно, он будет выглядеть как слишком старающийся.

Примеры:

--- Смеяться над своими собственными шутками. Если вы смеётесь над своими собственными шутками, вы метасообщаете, что вы не уверены в том, что окружающие засмеются. Т. е. разорвётся нормальный ритм беседы (шутка-смех). Это то же самое, как если бы кто-то вам рассказал что-то, что он считает смешным и затем он ожидает от вас реакции (смеха). Даже если вы не засмеётесь, то из уважения к нему, для того что бы поддержать нормальный ритм беседы, вы вербализируете свою реакцию и скажете «Ну да, смешно». Иногда кто-либо, начинает рассказывать весёлую историю или шутку и окружающие его люди начинают смеяться прежде, чем он закончит и станет понятно в чём «соль» шутки. Человек может спросить «Что вы смеётесь? Ведь я ещё не закончил» и это необъяснимо, но окружающие могут заржать ещё сильнее. Когда руководитель шутит, то подчинённые всегда смеются. Всякий раз когда вы шутите и никто кроме вас не смеётся, вы снижаете свою ценность. Вы испытываете потребность заделать ту брешь в нормальном ритме беседы (шутка-смех). Если вы так делаете, то немедленно прекратите. Когда действительно наступает время смеяться, то нет ничего страшного что вы будете смеяться вместе со всеми. Если вы засмеётесь над собственной шуткой слишком рано, то это можно трактовать как вашу попытку снять напряжение (страх что никто не засмеётся). Часто после рассказа шутки следует немного подождать, (10–15 сек), что бы до других «дошло». Если все же никто не начал смеяться, то просто не придавайте этому значения, обычно люди даже и не замечают, что вы пытались их рассмешить.

--- Говорить «Правильно?» или «Так ведь?» после каждого предложения. Говоря подобные фразы, вы тем самым выглядите как человек, нуждающийся в одобрении. Вы не должны позволять другому человеку считать, что вы нуждаетесь в его одобрении.

--- Пытаться получить вербальное одобрение. Часто людям необходимо получить вербальное одобрение со стороны других для того, что бы чувствовать себя комфортно. Они могут говорить такие фразы как «Это ведь было смешно, да?» или «Это ведь было интересно, правда?», вместо того что бы позволить собеседнику свободно выразить своё мнение. Не повторяйтесь и постоянно говорите, до тех пор, пока вы их не «зацепите». Просто продолжайте двигаться вперёд, пока чтонибудь не сработает.

--- Если вы постоянно используете один и тот же тип юмора, то он перестаёт работать. Например, парень может рассказывать весёлые истории, дразнить девушку, щекотать её. Но как только девушка осознает что он не веселиться, не просто весело проводит время, а сознательно пытается получить её реакцию с помощью заготовок, это будет хуёво. Как только девушка почувствует, что вы пытаетесь получить её реакцию, она перестанет смеяться. Будьте предсказуемы в своей непредсказуемости.
Старательная попытка продемонстрировать высокую ценность

Старательная попытка логически убедить кого-либо в том, что вы имеете высокую ценность, наоборот снижает вашу ценность. Это одна из самых худших ошибок которую вы можете сделать. Для того, что бы продемонстрировать высокую ценность вам необходимо, используя свои социальные навыки, просто создать отличную атмосферу для общения. Тогда ваш собеседник сам начнёт задавать вам вопросы и пытаться построить раппорт с вами. Вам необходимо так построить общение, что бы произошла смена ролей, что бы те, с кем вы общаетесь, захотели узнать о вас больше. Если вы попытаетесь впечатлить кого-либо, то вас отвергнут.

--- Всякий раз, когда вы будете настойчиво рассказывать людям о себе (verbal resume) вы будете терять свой статус и ценность.

--- Говорить слишком много для того, что бы выразить мысль, которая может быть выражена меньшим количеством слов. Если вы страдаете «словесным поносом» это демонстрирует вашу неспособность к точному мышлению. Часто если после того, что вы сказали вы сделаете паузу, что бы дать собеседнику возможность обдумать ваше предложение, то ваша идея может показаться более привлекательной.

--- Пытаться снова начинать говорить на тему, которая однажды была прервана. Это типичное поведение человека который надеялся впечатлить собеседника с помощью развития конкретной темы, и был огорчён тем что она прервалась. Если тема была достаточно интересная то ваш собеседник сам попытается к ней вернуться, сказав «Ну, дык, что ты там говорил про\ldots?» Если тема была прервана по независящим от вас причинам, то просто подождите, пока ваш собеседник сам не попытается вернуться к ней, или просто предложите другую тему для беседы. Позднее вы решите, хотите или нет, вновь возвращаться к этой теме. Или вы можете подойти к этой теме с «другого конца», для того что бы окружающим не было очевидно что вам очень хочется обсуждать эту тему. Но лучше всего просто перестать беспокоится, и забыть эту тему. Ведь если ваш собеседник сам не вернулся к ней, то возможно она была не так и хороша.

--- Компенсация неуверенности в себе. Часто, когда люди подвергаются социальному давлению, их неуверенность в себе становится очевидной для окружающих. Например высокая девушка, которой понравился невысокий парень, и которая боится что разница в росте может быть препятствием для их возможных отношений, может говорить ему что она «вообще-то ниже ростом, просто сегодня она надела туфли на высоком каблуке». Или человек который имеет не престижную и низкооплачиваемую работу, может пуститься в рассуждения о том, что деньги это зло, что люди одержимы деньгами.

Важно быть счастливым. Если другой человек интересуется вашей работой следует просто ответить чем вы занимаетесь, не пускаясь при этом в философские рассуждения. Множеству людей нравится их работа. А вам нравится ваша? Это как лысый парень, который шутит о том, что «Лысые мужчины сексуальны», обнаруживая при этом собственную неуверенность. В сущности, не было никакой необходимости для них поднимать эти темы. Если бы они имели сильный фрейм, то всё это было бы не существенно. Если проблема не существует для вас, то она не существует и для других людей.

--- Неуверенность в себе может так же проявляться в беспокойстве о том, как другие люди воспринимают вас, страх сделать что-то недостаточно хорошо, или выглядеть недостаточно привлекательно и быть судимым за это. Многие пытаются заранее оправдаться перед окружающими за то, что ещё не сделали или за то, за что они вообще не обязаны ни перед кем оправдываться. Это может относиться к таким фразам как «Я устал и поэтому\ldots» или «Если бы я захотел то я бы обязательно это имел, просто я не хочу». Если вы плохо одеты то не говорите «Вообще то у меня есть дома одежда и получше» или если вы вдруг идёте в грязной одежде, то не говорите «Я это\ldots после работы, и машина меня обрызгала, просто не было пока времени почиститься». Если вы говорите подобные вещи как бы оправдываясь за что-то, то собеседник почувствует вашу неуверенность в себе.

--- Стараться выглядеть как человек, который некогда не впечатляется. Когда парень чувствует собственные недостатки из-за того что другой парень явно лучше его, то он часто старается не впечатляться, как бы не замечать явных преимуществ другого. Он может говорить о том, как легко он может сделать то, что делает другой парень, или какой другой парень непривлекательный, или что он может что-то сделать гораздо лучше. И озвучивает он свои мысли тогда, когда чувствует собственную неуверенность. Он преуменьшает заслуги другого человека для того, чтобы чувствовать себя комфортно.

--- Отвечать на критику в свой адрес длинными чрезмерными оправданиями. Некоторые парни плохо переносят критику в свой адрес, это заставляет их нервничать и беспокоится, они начинают пытаться что-то кому-то объяснять. Если кто-то негативно оценивает вас, то вы можете сделать паузу, кивнуть головой и просто сказать «Ну да» или «Может быть». Или вы можете просто подразнить их, шутливо извратив их мысли типа «Спасибо, классно, говори ещё, я хочу знать о себе всю правду». Такой способ ответа на критику позволяет вам сохранять фрейм и с юмором реагировать на критику.

--- Некоторые люди считают, что окружающие ими постоянно манипулируют и пытаются снизить их ценность. Они ведут себя как детективы-параноики и постоянно пытаются «раскрывать» заговоры против себя. Они постоянно озвучивают окружающим свои подозрения, что только снижает их ценность. Квалифицируя себя таким образом, они пытаются показать как они умны. Например, такая девушка может сказать вам «Это шаблон для знакомства? Не смог запомнить ничего получше?» Уверенная с себе девушка не будет так отвечать, скорее она поверит в то, что понравилась ему и ей будет любопытно поговорить с ним.
\chapter{Социальная аура и калибровка}

Вместо того, чтобы квалифицировать себя, вам необходимо создать вокруг себя хорошую ауру (атмосферу), для того что бы окружающие захотели узнать вас лучше. Ваша аура начинает действовать на людей вокруг вас.

Людям в целом и женщинам в частности нравятся те, кто излучает позитив. Та аура, которую вы создаёте, является следствием вашего образа мышления. Ваши мысли проходят через эмоциональные фильтры, и с помощью своего поведения вы создаёте ауру вокруг себя.

Поведение клеевого парня, уверенность, веселье поможет вам создать отличную ауру, впрочем вы так же можете создать и отрицательную ауру. Та аура, которую вы создаёте, будет вызывать в людях определённые эмоции, на основании которых окружающие будут решать хотят они или нет находится рядом с вами. Социальная аура проявляется при взаимодействии людей друг с другом. Для этого нет какой то «специальной» причины, люди просто наслаждаются компанией друг друга и взаимоуважением.

Общаясь таким образом, мы тем самым как бы «заряжаем наши батареи». Мы создаём подобную ауру с помощью шуток, рассказывания историй, обсуждения интересных тем, мы можем даже в игривой манере разрывать раппорт друг с другом, демонстрируя при этом нашу личность и укрепляя наши отношения. Когда ваши друзья просто говорят вам «Пойдём, выпьем по 5 капель», обычно это не означает, что они хотят обсудить с вами какую проблему. Скорее они хотят получить удовольствие от общения, побыть в хорошей компании и получить отличные эмоции.

Подумайте о том, что является сущностью «ботаника»? Как должен тратить своё время человек, что бы мы подумали что он «ботаник»? Вероятнее всего мы будем так думать: если человек сутками висит в интернет-чатах, постоянно играет в онлайновые компьютерные игры, чрезмерно «заморачивается» на изучении научных дисциплин. (Живёт в виртуальном мире). Почему мы считаем подобные занятия обычными для ботаника? Люди, которые не обладают социальными навыками, чаще всего будут пытаться как-то компенсировать недостаток нормального общения. Что является общим для всех ботаников, так это то, что они используют социальные костыли, которые позволяют им создавать вокруг себя свою ауру, свой особый мир.

Это как первое неуклюжее свидание в девятом классе, когда девушка и юноша на свидании делают вид, что каждый читает книгу. Время от времени они «отрываются» от чтения для того что бы что-то сказать друг --- другу, когда сказать больше нечего они вновь с усердием погружаются в чтение, уходят в свою зону комфорта, для того что бы придумать что ещё можно сказать.

Почему создание отличной ауры метасообщает высокую социальную ценность? Когда парень сосредоточен на себе, сосредоточен на управлении своим поведением он не способен излучать нужные «вибрации» в окружающий мир. Когда же парень уверен в себе, то для него нет необходимости концентрироваться на себе, он концентрируется на окружении, на той ауре которую он создаёт, являясь сам неотъемлемой частью этой ауры.

Например, при важном собеседовании, немногие парни решатся рассмешить своего нового работодателя. Большинство парней подумают также, что не нужно шутить с офицером полиции, который их арестовал. Когда парень создаёт вокруг себя привлекательную ауру, даже находясь под социальным давлением, то это метасообщает его высокую социальную ценность, поскольку он может чувствовать себя комфортно даже в трудной ситуации.

До тех пор пока большинство людей хотят получать положительные эмоции, парень который может дать им их, может создать привлекательную ауру, будет обладать ценностью практически в любой ситуации. Когда люди смеются и демонстрируют свой интерес, то тем самым они показывают что одобряют подобное поведение, принимают его и дают парню социальное доказательство его ценности.

Именно по этой причине социальные навыки так привлекательны, они могут дать человеку внешнюю ценность, которая полностью основана на их личности. Внешность и деньги не могут дать подобную ценность, если они не являются частью того, кто вы есть.

Итак, почему же для большинства парней очень трудно создать вокруг себя привлекательную ауру?

Причина не в том что им просто нечего сказать, ведь все мы обладаем неисчерпаемыми разговорными ресурсами. Скорее дело в том, что их неуверенность в себе заставляет их почувствовать, что они ничего не должны говорить, поскольку они не смогут ничего ценного добавить к общению.

Если парень воспринимает себя как человека имеющего более высокую социальную ценность по сравнению с окружающими, тогда он будет вести себя естественно и говорить всё, что бы не пришло к нему в голову. Если же он воспринимает людей вокруг себя как имеющих более высокую ценность, чем его ценность, то тогда он может чувствовать что ему просто нечего сказать. Потому что он не будет уверенным в том, что то, что он может сказать будет достаточно клёвым для того, что бы привлечь их внимание.

Социальное давление приводит к тому, что люди начинают чувствовать неуверенность. Без давление приводит к параличу социальной разумности, и человек начинает концентрироваться на себе.

Некоторые люди, реагируя на подобное чувство неуверенности в себе, и основываясь на текущей теме разговора, начинают раздавать непрошенные советы всем вокруг. Другие же, о чём бы беседа не шла, используют любую возможность, что бы показать как то, что обсуждается, связано с их жизнью. Они используют эти «окна» в беседе для того, чтобы квалифицировать себя.

Часто для того, что бы создать определённую ауру общения, один парень может задать другому невинный квалифицирующий вопрос. Например, «Смотри, я недавно купил куртку, нормальная да?» Если его собеседник обычный нормальный парень, то он просто может ответить «Да, классная куртка. Отлично на тебе сидит». В таком ответе нет ничего выдающегося. Это просто хороший ответ. Когда парни изучают социальную динамику, то они не всегда осознают что для того что бы быть клёвым нет необходимости всегда контролировать фрейм. Быть клёвым это тонкие, едва заметные вещи.

С другой стороны, если собеседник оказался «гнилым», он может начать раздувать чувство неуверенности спрашивающего: «А сколько ты за неё заплатил?\ldots Сколько? Я видел точно такую же куртку гораздо дешевле. Готов поспорить, что ты лоханулся. Ты знаешь, как её правильно стирать? Давай я тебе расскажу».

Или же, если собеседник --- дурак, то он может использовать заданный вопрос, чтобы начать «выйбываться»: «Я видел такую куртку, когда был в Италии. Это было прошлой зимой, я заработал там кучу денег. Я там вообще классно провёл время\ldots Угадай, за сколько я там купил такую же куртку?»

Нормальный парень, просто может сказать «Ну в общем, классно. Выглядит дорого» Он понимает, что можно просто подыграть в этой ситуации и похвалить покупку. Дать одобрение, развеять неуверенность. Поскольку это риторический вопрос и они вдвоём просто создают ауру беседы. Он понимает, что в хорошем общении часто не бывает иной цели кроме хорошего общения. Идиот же может начать использовать любую возможность для того, чтобы доказать насколько он крут и умён, типа: «А сколько стоит?\ldots Ну ты ведь знаешь, что качественная вещь стоит гораздо дороже, чем ты заплатил за эту куртку. Скорее всего, ты купил китайскую херню. Я то всегда вижу качественную вещь, меня не проведёшь».

«Социальная калибровка» --- знание того, как протекает социальная энергия, понимание того, как то, что вы делаете, влияет на социальную энергию, и способность изменять собственное поведение для того, чтобы получать желаемый результат.

Парни с развитой социальной калибровкой отчётливо понимают как то, что они говорят или делают, влияет на то, как их воспринимают окружающие. Они понимают, как правильно донести свои мысли до других людей. У них есть особое внутренне чувство, с помощью которого они могут определить:

--- Ценность, которую они уже создали в этой группе.

--- То, что окружающие уже знают или предполагают о них.

--- Как другие люди воспринимают мир и социальную среду.

--- Текущее эмоциональное состояние и мыслительные циклы других людей.

--- Эмоциональное воздействие, которое они могут произвести «прямо сейчас».

--- И вдобавок, у них всегда есть «готовые ответы» в любой ситуации.

Парни, у которых отсутствует калибровка, склонны постоянно делать ошибки, которые происходят от их непонимания всего вышеперечисленного. Например, парень предполагает что в разговоре с девушкой, с которой он только что познакомился, он может разговаривать обо всём, включая даже негативные темы.

С его точки зрения это означает «быть самим собой». Он думает что может говорить о своих проблемах, об ошибках и т. п., что когда он действует подобным образом, то это лишь подчёркивает его уверенность в себе, так как он ничего не скрывает, он такой, какой он есть. Он пытается подражать уверенным в себе парням, и он предполагает, что он делает то же самое, что они. Но он не понимает, что уверенные в себе парни уже продемонстрировали окружающим другие части своей личности, прежде чем так себя вести.

Та же самая проблема может возникать, когда он пытается флиртовать с женщиной. Он может знать, что дразнить и трогать девушку это часть флирта и так нужно делать. Но он не понимает что прежде чем вести себя подобным образом, с её стороны должно быть хотя бы минимальное влечение к нему. Если же предварительно он не создал собственную ценность в её глазах, то она «отморозится» на подобное поведение. Она будет считать, что он идиот и придурок раз дразнит и трогает её. Вместо того, что бы разжечь её влечение по отношению к себе, он делает вещи которые заставляют её испытывать дискомфорт и «закрываться».

--- Можете ли вы «видеть» энергетику?

--- Можете ли вы почувствовать разницу между энергией нужды и энергией комфорта?

--- Можете ли вы почувствовать разницу между мужской и женской энергией?

--- Сможете ли вы почувствовать как изменится энергетика, если вдруг в комнату полную парней зайдёт девушка и наоборот?

--- Сможете ли вы сказать, кто управляет энергетикой в группе?

--- Можете ли вы видеть, как по разному люди реагируют на разную энергетику?

--- Можете ли вы чувствовать как что-то добавляет энергии, или наоборот забирает её?

--- Сможете ли вы посмотреть на группу людей и сказать насколько клёвая или не клёвая энергетика в группе?

--- Можете ли чувствовать, что вообще является клёвым или не клёвым?

--- Сможете ли вы почувствовать что прошло достаточно времени для того, что бы энергетика была готова измениться?

Калибровка включает в себя все эти вещи и калибровка включает в себя так же понимание когда можно расслабиться и перестать заниматься калибровкой. Понимание того, что у людей существуют «ответы на автопилоте».

Вам нужно научиться не говорить таких вещей, которые запускают нежелательные вам ответы на автопилоте. Когда вы пообщаетесь с достаточно большим количеством людей, вы будете видеть все эти мыслительные шаблоны. Большинство людей говорят одинаковые фразы в одинаковых ситуациях. Со временем вы сможете предсказать, что думает каждый, с кем вы общаетесь.
\chapter{Логическое и эмоциональное состояние}

Логика и эмоции это два противоположных полюса нашего разума. Логическое состояние ума подавляет эмоциональную сторону человека. Точно так же, находясь под влиянием эмоций человек перестаёт мыслить логически. Когда нас захлёстывают эмоции, то логика выключается.

Когда же мы погружены в логические рассуждения, то мы становимся менее эмоциональными. Когда парень целый рабочий день погружён в логическое состояние ума, то к концу рабочего дня это его состояние достигнет своего пика. Решая логические задачи, такой парень 8 часов подряд концентрировался внутри себя, стараясь не замечать внешние раздражители и это стало его привычным состоянием.

Некоторые парни скажут вам что «делу время --- потехе час», что на работе нужно заниматься работой, а после работы весело отдыхать. Но не многим из них удалось так развить свою личность, чтобы чувствовать себя комфортно всегда и везде.

Находясь в социальном окружении они часто остаются погружёнными в собственные мысли и тем самым отделяют себя от того, что происходит в окружающем мире. Они чувствуют себя странно, находясь в местах где «хлещет» социальная энергия, где играет громкая музыка, все о чём-то беседуют и веселятся.

Они оказываются не способны отдаться потоку льющейся энергетики окружающих людей, они как бы находятся вне зоны общения, потому что они сосредоточены на собственном мыслительном процессе и всё время пытаются анализировать поступающую извне информацию.

Они чувствуют себя неуютно если вдруг кто-то из окружающих обратит на них внимание и заговорит с ними. Их шутки кажутся неестественными и неуклюжими, часто они не знают о чём вообще можно говорить, потому что для них это всё непривычно. Они не знают как войти в нужное состояние для того, чтобы отдаться потоку энергетики.

Данное поведение демонстрирует их непривлекательные качества. Кроме того, логическое состояние ума подавляет сексуальные инстинкты. По этой причине, люди которые находятся в логическом состоянии ума воспринимаются окружающими как не сексуальные. С логической точки зрения занятия сексом вообще можно рассматривать как дурацкое и весьма рискованное поведение.

Но с эмоциональной точки зрения, секс естественная и невероятно привлекательная вещь. Наша сексуальность основана на наших эмоциях. Люди, которые чаще доверяют своей эмоциональной стороне, имеют гораздо больше секса, чем их погружённые в логические размышления противоположности.

Для парней, которые добились успеха в жизни через логический подход, попытка применить его в съёме не сработает.

Их успешные с женщинами друзья, могут говорить такому парню: «Чувак расслабься. Будь клёвым. Будь уверенным в себе. Просто будь самим собой». Часто это бывает им неверно истолковано. Для него слова «будь уверенным», «будь самим собой» звучат глупо. Когда парень уже добился успеха в определённых областях жизни, то он уже обладает уверенностью в себе, он уже «тот кто он есть»\ldots Он может пытаться следовать советам своих друзей, пытаться расслабится, стать ещё более уверенным, но это не сработает для него.

Когда друзья говорят ему «будь клёвым», «будь самим собой» они пытаются сказать ему что бы он прекратил логические размышления, бросил попытки управлять собой и беспокоится о том, как другие люди воспринимают его. Что бы он просто наслаждался общением ни о чём не заботясь, перестал анализировать.

Часто парни, для которых логическое состояние ума является привычным, используют его для подавления нестабильных, по своей сути, эмоций. Их образ мышления, их логическое состояние ума позволяет им отфильтровывать негативные эмоции ещё на подходе. Тем самым они избегают каких-либо сильных переживаний и стрессов.

Они не позволяют бушевать эмоциям в своём разуме. Они привыкли к логическому состоянию своего ума, оно позволяет им пребывать в своей зоне комфорта. Как «проблемный» парень всё время пытается перевести разговор с окружающими на обсуждение своих проблем, так же и парень в логическом состоянии разума пытается перевести беседу к обсуждению каких-то логических тем. Это позволяет ему оставаться в своей зоне комфорта.

Поведение является отражением внутреннего состояния. Всегда и везде. Когда вы первый раз знакомитесь с кем-то, то неизбежно вы начинаете оценивать друг --- друга, осознаёте вы это или нет. До определённого момента вы не знаете, захотите ли дальше продолжать общаться с этим человеком.

Если вы начинаете диалог с безобидных шуток, то тем самым вы демонстрируете, что вы общительный, интересный собеседник и вы никого не «напрягаете». Вы становитесь тем, о ком хочется узнать больше.

Людям свойственно стремится к положительным эмоциям. Все мы хотим общаться с весёлыми и клёвыми людьми.

Женщина оценивает мужчину прежде всего на основании эмоций, которые она получает находясьрядом с ним. То, каким образом вы будете строить ваше общение с девушкой в момент знакомства и те эмоции которые она почувствует в вашем присутствии, будет в дальнейшем определять захочет ли она дальше общаться с вами.

Она может захотеть провести с вами одну ночь или всю жизнь или просто отшить вас, если вы не сможете «законтачить» с ней.

Первые минуты знакомства это не время для того, чтобы анализировать ситуацию или пытаться впечатлить кого-либо. Если вы сможете создать подходящую ауру общения, то ваши собеседники полюбят вас и девушки будут заинтересованы вами.

Большинство парней пытаются привлечь женщин, общаясь с ними на довольно скучные темы, типа расспросов в стиле интервью, и они не могут зацепить внимание женщин. Они делают то же самое, что делают и другие скучные парни.

Вместо того, чтобы создавать ауру общения они начинают доставлять неудобство девушке, задавая вопросы личного характера. Тем самым они начинают демонстрировать излишний интерес к женщине, о которой они ничего не знают. Они пытаются использовать свой интерес к женщине как оправдание того, что они стараются узнать её получше.

Подобное поведение говорит о низкой ценности таких парней, оно говорит о том, что секс для них значит больше, чем для неё.

Парень более успешный с женщинами, который вызовет больше влечения, прежде всего будет стремиться создать ауру общения.

Людям он будет нравиться, и они будут давать ему подтверждение его ценности. Когда он разговаривает с женщинами, то он всегда знает как общаться с женщиной на эмоциональном уровне, его интерес к женщине, в противоположность обычным парням, наоборот говорит о его высокой ценности. Он оценивает (screening) женщину и заставляет её пытаться узнать его получше.

Его поведение выглядит естественным, ему «всегда есть что сказать» и это не потому, что он тщательно подготовился к беседе, а потому что он просто весело проводит время, действует без страха за результат.

Всё это делает девушку более восприимчивой по отношению к нему, что бы он не сказал или сделал. Будучи сосредоточен на окружающих его вещах он всегда находится в «настоящем моменте», и это даёт доступ к эмоциональной стороне своего разума, это то что делает его успешным с женщинами.

Состояние веселья и игривости --- это эмоции. Логики здесь нет. Вам просто нужно разрешить себе быть весёлым и непосредственным. Вы должны быть непредсказуемым. Вы должны «развлекать» окружающих так, как никто этого не ожидает. Когда вы шутите, а окружающие смеются, они принимают ваше поведение и тем самым вы строите ауру общения с ними.

Однако следует заметить, что сама манера «подачи» юмора более важна чем «содержание» шутки. Два парня могут использовать одинаковые шутки и получать от окружающих совершенно разную реакцию.

Ваш юмор исходит от вашей уверенности в себе, вы просто передаёте ваше внутреннее состояние. Вы используете образное мышление для того, чтобы в игривой манере взаимодействовать с окружающими вас людьми, вступать с ними в связь.

Для того, чтобы игривое поведение сработало вы должны сами искренне верить в то, что вы делаете. Вы должны сами в этот момент чувствовать тот мощный энергетический импульс, который вы хотите передать другим. Если вы хотя бы на секунду засомневаетесь, вернётесь к внутреннему диалогу, для того что бы что-то обдумать и подкорректировать своё поведение, это не сработает. Это ключевой момент.

Если вы сами верите в то, что вы делаете, сами чувствуете те эмоции, которые хотите передать другим, вы всегда будете попадать точно в цель.

Что означает быть игривым? (playful). Это довольно широкое понятие. Оно включает в себя и воображение, и эмоции и много других вещей. Прикольные фантазии, извращённое толкование её слов, глупые сравнения её с кем-то или шутливые обвинения, посылание смешанных сигналов (ближе --- дальше), использование языка тела --- всё это является игривым (забавным).

Например говорить женщине «Я тебя ненавижу», улыбаясь при этом. Или говорить «Ты такая сильная. Будешь моим телохранителем?» Вы можете дразнить женщину, иногда позволить ей одержать над вами маленькую победу, огорчить её, а потом рассмешить, уйти от неё, а потом вернуться, когда она этого не ожидает.

Вы как маленький ребёнок, вы просто играете. Вы можете покрутить её вокруг своей оси, посадить на стул за своей спиной и начать разговаривать с её друзьями. Когда она соскочит со стула и раздражённая появится перед вами снова, то вы можете улыбнуться и спросить «Слушай, а куда ты уходила?»

Вы можете сказать «Ты настолько удивительная девушка, что я даже не знаю о чём можно с тобой разговаривать», а затем повернуться спиной. Скорее всего, она развернёт вас обратно, что бы выяснить почему. Вы можете не отвечать на её вопросы, или давать глупые ответы, что сделает её более заинтересованной в вас.

Вы можете шутить с серьёзным выражением лица, а когда она засмеётся --- то начать смеяться вместе с ней. Или вы можете вообще молчать, но при этом тыкать в неё пальчиком и улыбаться.

Любое поведение о котором можно сказать «игривое», это правильное поведение. Вы весёлый и клёвый вне зависимости от того, как она на вас реагирует. Вы можете дисквалифицировать себя, сказав ей «Я плохой мальчик. Мы не подружимся» а затем рассказать ей о вещах, которые являются неправильными и непозволительными с точки зрения общества, но при этом ужасно притягательными.

Или вы можете шутливо извратить то, что она сказала. Например, она может спросить вас «А где ты живёшь?», а вы можете ответить «Ты конечно классная девчонка и всё такоё\ldots но пока я не готов с тобой переспать». (Вы намекаете, что она напрашивается к вам в гости что бы переспать с вами).

Безусловно, всё это требует калибровки. Для всего есть своё место и время. Однако, прежде всего, доверяйте вашим эмоциям, пусть они руководят вами. Тем не менее, вам нужно учитывать несколько важных моментов.

Некоторые парни своей игривостью пытаются компенсировать неумение «вести» в разговоре, отсутствие лидерских качеств, неумение продемонстрировать свою личность. Многие парни слишком долго пребывают в игривом состоянии, излишне затягивают подобное поведение, используют одни и те же шутки, в конце концов девушке это надоедает и она их отшивает. Не будьте таким.

Если парень провёл с девушкой уже несколько часов, то она может просто хотеть знать где он живёт. Если на её нормальный вопрос, он начинает продолжать глумится в стиле «Ты конечно классная девчонка и всё такоё\ldots но пока я не готов с тобой переспать», то он будет выглядеть глупо, это не нормальное поведение.

Это как прелюдия перед сексом и сам секс. Заканчивается одна стадия общения и начинается следующая. Если слишком долго от прелюдии не переходить к сексу, то в конце концов девушка просто потеряет влечение.

Иногда приходит время побыть серьёзным, просто ответив на вопрос.

Быть ребёнком, беззаботно дурачиться и играться, не беспокоясь ни о чём, все это должно быть частью вашей личности. Но одновременно, быть игривым, может так же говорить о вас то, что вы взрослый зрелый человек.

Игривость может сообщать о вас очень важные вещи. Вы не нуждающийся, вы умный, вы остряк, всё это сообщает о вас, что вы являетесь социально адаптированным человеком. Вы сексуальны. Вы тот, с кем люди хотят находиться в одной компании.

И, наконец, самое главное, игривость это способ который позволяет вам и девушке оценить друг друга (screening), узнать что-то друг о друге (revealed to each other).

Давайте представим себе две компании парней, которые пошли знакомиться с девушками.

Первая группа парней ощущает постоянный дискомфорт, они нервничают в присутствии женщин. Их нервозность толкает их в логическое состояние рассудка, они надеются «просчитать» ситуацию с помощью логики, занимаются самокопанием.

Наконец они решают зайти в знакомое им «заведение», где они надеются кого-нибудь снять. Зайдя в заведение, они начинают совершать уйму глупых, но привычных им действий. Они решают для начала заказать себе «по пиву», прогуляться до туалета, обойти всё заведение с целью посмотреть нет ли здесь их знакомых, или начать слоняться вокруг для того что бы «поглазеть» на женщин.

Они «прочёсывают» окружающую обстановку нуждающимся взглядом и выглядят странно. В конце-концов они находят себе уютное местечко где можно присесть и держат бокалы с пивом перед своей грудью, как будто они защищаются ими от кого-то.

Вместо того что бы просто весело общаться друг с другом, они начинают вертеть шеями и пристально разглядывать окружающих их людей. Они ищут ценность и веселье в других людях, а так же обмениваются мнениями о том, как правильно снять девочку, рассказывая друг другу о своих прошлых «победах».

Их разговор приводит к тому, что они погружаются в логическое состояние рассудка, они начинают анализировать ситуацию, иногда даже спорить о том, что ещё им необходимо сделать что бы снять кого-нибудь.

В итоге они всё-таки подходят знакомиться, но в состоянии, которое говорит девочкам о том, что парни хотят что-то взять от них. Чего эти парни не осознают, так это то, что если они не способны создать привлекательную ауру общения разговаривая друг с другом, то вероятнее всего они не смогут так же создать ауру, общаясь с другими людьми.

Весь их подход к девочкам выглядит неестественным и скучным. Своим поведением они метасообщает окружающим о том, что они просто компания нищих духом мудаков, которые жалобно просят подать им ценности в виде хорошей компании и женского внимания.

Вторая группа парней чувствует себя комфортно и просто весело проводит время. Когда они идут в «заведение» то обмениваются шутками, находятся в игривом настроении, в общем они строят хорошую ауру прежде всего внутри своей группы.

Их аура заметна всем окружающим их людям. Когда они подходят к девушкам, то девушки с легкостью «открываются» Поскольку девушки уже заметили этих парней ранее, увидели их ауру, почувствовали энергетику. Просто парни устроили для себя праздник, а девушки очень хотят, что бы их пригласили на этот праздник.

Парни флиртуют с девчонками и одновременно обмениваются друг с другом, им одним понятными шутками. Девчонки смеются вместе с парнями, часто не вполне даже понимая смысл шуток. И чем больше девчонки смеются над шутками, тем больше их начинают заинтересовывать эти парни, они начинают втягиваться во фрейм парней.

Девушкам хочется продолжать оставаться в компании этих парней, потому что они весёлые и клёвые. Девушки знают, что они всегда смогут уйти от этих парней, если они этого захотят и парни не будут их преследовать. Парни не ищут реакции девушек на своё поведение и просто позволяют девушка оставаться на их празднике жизни. Позднее, когда парни предложат им поехать «посмотреть классный фильм», дома у одного из парней то девчонки скорее всего захотят, чтобы веселье продолжалось и согласятся.
\chapter{Доминирование}

Контроль фрейма

Когда человек с более высокой ценностью, начинает дразнить человека с низкой ценностью, то обычно последний не будет способен достойно ответить. Это происходит вследствие того, что человек с низкой ценностью будет постоянно задумываться над тем, достаточно ли хороший ответ он придумал для того, чтобы вернуть остроту «обидчику».

В его эмоциональном состоянии произойдут видимые перемены. Его глазной контакт, его голос и его язык тела вообще будут демонстрировать неспособность к доминированию.

Он начнёт судорожно искать способ, каким образом он сможет ответить не менее остроумно и окружающим будет это заметно. Он «уйдёт в себя» в поисках ответа\ldots Когда люди начинают себя вести подобным образом, то они метасообщают окружающим, что они менее доминантны, чем человек который их дразнит. Что бы они не сказали это не будет вписываться в ауру общения, потому что они начинают квалифицировать себя. Когда они отвечают, они отвечают способом, который говорит об их неспособности к доминированию.
Язык тела

На определённом бессознательном уровне любой парень знает, какое поведение является доминантным и как его продемонстрировать. Вместе с демонстрацией доминирующего поведения его самооценка возрастёт и он начнёт получать подтверждение своей ценности.

Его логика выключается и им начинают управлять его эмоции. Необходимость в постоянном сознательном контроле своего поведения теряет актуальность, поскольку он чувствует себя спокойно и уверенно в данной обстановке. Всё это делает его доминантным.

Он естественным образом займёт определённую ступень в социальной иерархии, и окружающие воспримут это нормально. Никто даже и не задумается, почему этот парень воспринимается как доминантный.

Это происходит потому, что своим поведением он метасообщает свой высокий статус и он конгруэнтен своему поведению. Если вы уверены в том, что вы делаете, это будет клёвым. Правильно? Если кто-то не совсем уверен в том, что то, что он делает это клёво, то он будет нерешителен в своих действиях. Правильно? Любой человек с правильным умонастроением и уверенностью будет клёвым.

Когда у вас есть это качество, то люди будут очень восприимчивы по отношению к вам. Они всегда будут рады уступить вам дорогу. Если вы захотите пойти перекусить, то они с радостью присоединятся к вам, даже если они не голодны.

Если вы одеты странно и необычно, то вы можете рассказывать окружающим как это клёво носить такую одежду и люди будут внимательно слушать. Окружающие будут поддакивать вам и соглашаться с вами, чтобы вы им не сказали. Люди будут смеяться над вашими шутками, независимо от того смешные это шутки или нет.

Когда вы входите в помещение, то девушки будут смотреть на вас. Временами даже вы сможете заметить как они перестанут обнимать своих парней и будут искать с вами глазного контакта. Высможете подойти к ним, подразнить их и даже вторгнуться в их личное социальное пространство. Они разрешат вам сделать это. В некоторых случаях вы можете даже начать игриво трогать девушку и потом зажечь с ней. (Читатели могут удивиться, а как же парень девушки? Он должен быть вам благодарен, как минимум, вы открываете ему глаза).

Девушек привлекают парни, излучающие доминантную энергию. Таким парням девушки позволяют то, чего они никогда не потерпят от менее доминантных парней. Девушка не будет возражать против подобного поведения со стороны доминантного парня, поскольку она знает, что он не ищет её реакции, и если даже она отвергнет подобное поведение, то это парня никак не заденет. А поскольку парню похуй на то как она реагирует то и она может делать, что захочет, не испытывая давления.

Поскольку парень не является нуждающимся, то и девушке нет смысла контролировать себя и беспокоится о том, как он оценит её реакцию. Она заражается позитивной энергией парня и может просто расслабиться и позволить взаимодействию продолжаться, не беспокоясь о том, что она должна отвечать парню определённым образом.

Она существует в одном моменте вместе с ним. Она чувствует лёгкость. Это не есть что-то важное. Поэтому для неё нет никаких причин для того, чтобы не позволить этому продолжаться. Она знает что этот парень не такой как остальные, кто будет оценивать и анализировать всё, что она скажет или сделает, и затем может осудить её за её поведение или начать толковать её поведение в стиле «Что значит то что она сделала? Что она хотела мне сказать?»

Такой парень так не похож на её друга, который её постоянно ревнует и заставляет выслушивать его проблемы. Такой парень не будет ревновать, считать своей собственностью или спорить с ней.

По этой причине, большинство «обычных» парней выглядят болванами. Когда девушка встречает парня, который позволяет ей расслабиться, она начинает чувствовать к нему сексуальное влечение. Это происходит на уровне бессознательного, строиться естественная аура между двумя людьми. Они оба в этой ауре.

Девушке понравится такой парень, и она уйдёт с ним куда угодно, если возникнет такая возможность. Потому что он клёвый. Потому что он беззаботный. Потому что он искренний. Потому что он весёлый. Всё это естественная, необходимая динамика. Девушка не сможет вести себя так же с парнем, у которого низкий статус. Скорее всего, она будет беспокоиться о том, что бы не ранить его чувства, отвергнув его.

Когда мужчина в прошлом уже испытывал боль, в его поведении могут проявляться разные отталкивающие моменты. И как следствие, женщина на автомате реагирует на эти его неприятные особенности поведения и избегает его. Он не находится «в моменте», он серьёзный, он не клёвый, она просто не хочет его общества.

Давайте рассмотрим признаки доминирующего поведения: глазной контакт, прикосновения, занятие окружающего пространства, расслабленный язык тела, положение в группе, голос, решительность, умение быть судьёй и заставить людей квалифицировать себя перед ним, белка вокруг дуба, видимые эмоциональные реакции и кто на кого больше реагирует, с ним трудно получить раппорт, люди уступают ему дорогу когда он идёт сквозь толпу, люди уважают его и реагируют на него, вместо того что бы искать друзей он сам первый проявляет инициативу.

Речь: не квалифицирует себя, заряжает энергией (интрига, юмор, сексуальность), контроль фрейма, не активировать ASD, настойчивый и уверенный в себе.

Невербалика: язык тела, тон голоса, способность громко выкрикнуть, способность к сопереживанию и установлению связи, энергичность, сексуальные намёки, игривость, не нуждаемость в одобрении, использование голоса для демонстрации высокой ценности, шепот, использование своего индивидуального неповторимого стиля: рукопожатие, выразительность, манера разговора, манера прикосновения к другим.

Вам необходимо понять, что получить привлекательную женщину не является чем-то трудным и сложным. Вы не должны рассматривать привлекательных женщин как каких-то волшебных созданий, они такие же как и остальные девчонки, которые просто хотят сильного мужчину:

--- быть непредсказуемым,

--- занимать нужное социальное пространство,

--- расслабленный язык тела, чувствовать себя комфортно,

--- громкий голос,

--- решительность,

--- быть арбитром или спровоцировать людей квалифицировать себя перед вами,

--- иметь в уме чёткие границы дозволенного, в нужный момент быть готовым разорвать раппорт,

--- белка, бегающая вокруг дуба,

--- ясные эмоциональные реакции,

--- прикасаться и правильно реагировать/провоцировать прикосновения.

Быть парнем, который не решает начать встречаться с первой девушкой, с которой он познакомился. Иметь в своём уме границы и стандарты. Расслабленный язык тела и голос; движения слегка замедлены; когда вы входите, люди оживляются; вы создаёте энергетику вокруг себя. Если в вас это есть, то это даст вам значительные преимущества. У вас так же должен быть определённый набор материала (истории, рутины) который в нужный момент поможет вам.

Проактивная социальная стратегия, вера что вы этого достойны, ожидание успеха, не быть «проблемным» и т. д.

На определённом личностном уровне у вас должна быть непоколебимая уверенность, что у вас есть на всё это, неотъемлемое право, право быть мужчиной с высокой ценностью. Вы должны верить в то, что в любом социальном взаимодействии, другой человек будет просто счастлив общаться с вами, что вы обладаете ценностью, которой можете делиться с другими.

Когда вы демонстрируете подобные личностные качества --- люди будут реагировать на вас. Однако, всегда найдётся небольшое количество людей, которые так же как и вы, обладают сильной личностью. Они не будут так явно, как другие реагировать на вас, тем не менее они «вычислят» вас, «примут за своего» и будут уважать вас. Такие люди социально адаптированы, им комфортно с самими собой, у них сильный фрейм, они так же чувствуют себя комфортно, общаясь с достойными людьми.

Так же вам будут встречаться ментально слабые люди, кто будет искать комфорт, пытаясь занимать низшие ступени социальной лестницы. Найдутся так же люди, которые будут пытаться демонстрировать вам сильную личность, каковой они в действительности не являются. Их доминирующее поведение имеет хрупкое основание, часто они занимаются самообманом, веря в то что являются доминантными. Такие люди почувствуют себя неуверенно в вашем присутствии и скорее всего, попытаются «опустить» вас.

Но подавляющее большинство людей будут принимать ваше поведение, они будут заинтересованы вами, смогут чему-то учиться у вас, а так же наслаждаться тем чувством безопасности, которое вы им даёте.

Они будут следовать за вами, как за лидером, и их мысли будут немного путаться. Они будут стараться любой ценой сохранить контакт с вами, отвечать на все ваши вопросы и использовать ваш тип юмора. Они будут стремиться шутить так же, как и вы, рассказывать истории на те же темы, что и вы, и соглашаться со всем что вы говорите, без критического осмысления ваших высказываний, поскольку они будут пытаться сохранить раппорт с вами.

Если вас воспринимают как доминирующего, то парни будут реагировать на вас, будут пытаться подружиться с вами, или даже подлизываться к вам. Девушки же будут воспринимать вас как более привлекательного.

Вы можете создать фрейм общения немедленно после начала разговора, или же вы можете делать это постепенно. Сможете ли вы удержать свой фрейм или вы потеряете его зависит от того достаточно ли силён ваш фрейм, через который вы демонстрируете вашу личность, сможете ли вы взаимодействовать с окружающими вас людьми таким образом, чтобы они реагировали на вас, стараясь сами измениться для того, чтобы приспособиться к вам.

Девушки же примут бессознательное решение подходите ли вы им в качестве возможного сексуального партнёра в течение первых нескольких минут знакомства с вами. Это бессознательное решение будет во многом основано на вашем языке тела и звучании голоса. Следуя правилу, что «Влечение не является выбором» мы можем утверждать некоторые вещи:

1) Если вы не знаете какой язык тела используют парни, которых мы называем «сексуальные животные», то, вероятнее всего, вы не сможете использовать те же вещи даже «случайно».

2) На самом деле слова, которые вы говорите, не имеют большого значения. Не имеет значения что вы говорите, важно как вы это говорите. Обратите внимание на звучание вашего голоса и ваш язык тела.

3) Вам необходимо научиться выражать свою сексуальность и уверенность используя язык тела и интонации голоса. Овладев этими базовыми вещами, вам намного легче будет привлекать женщин.

Я могу поспорить с вами, что если вы не знаете какой язык тела и тембр голоса заставляет возбуждаться женщин то вы и не применяете это, не так ли?

Итак, изучите это и используйте это. Когда я только начинал изучать искусство соблазнения, я думал, что мне всего лишь необходимо освоить какие-то шаблоны для знакомства и прочие трюки. Я понятия не имел, что весь этот материал бесполезен без правильного использования языка тела.

После множества проб и ошибок, я наконец начал понимать всю важность языка тела, кое-что исправил в своём поведении и практически сразу же я начал нравиться женщинам.

Как и большинство парней, вы вероятно хотите знать это и научиться этому. Что ж, если вы на самом деле хотите глубокого понимания этой темы, то я могу предложить вам мою аудио программу, которая посвящена этому вопросу.

Я потратил много времени, пытаясь разобраться в том, как это работает. Три следующие вещи моментально увеличат вашу привлекательность в глазах женщин:

1) Научитесь держать глазной контакт дольше, чем она. Если вы заметили привлекательную женщину и она посмотрела на вас, то ни в коем случае не отводите взгляд в сторону, не отворачивайтесь. Большинство парней могут смотреть на привлекательную женщину до тех пор, пока она не заметила, что за ней наблюдают. Как только она почувствовала, что на неё смотрят и поднимает взгляд на парня, то он немедленно отворачивается, тем самым он демонстрирует робость и застенчивость. Это огромная ошибка. Если вы хотите метасообщить о себе привлекательные качества вам необходимо продемонстрировать ей, что вы мужчина и не боитесь своих мужских желаний.

Есть хорошее упражнение для развития этого навыка. В течение нескольких часов гуляйте поторговому центру и смотрите прямо в глаза каждой женщине, которую вы видите. Если она в ответ посмотрит на вас, то не отводите взгляд первым. Вы окажете себе большую услугу, если во время выполнения этого упражнения вы не будете слишком широко открывать ваши глаза и улыбаться как наёмный убийца. Скорее всего, женщины не будут в восторге от этого. Просто научитесь держать глазной контакт до тех пор, пока женщина сама не отведёт взгляд. Это очень важно.

2) Уверенные позы.

Большинство мужчин ведут себя неуверенно и думают о себе примерно следующее «Я не чувствую уверенности в себе когда я что-либо делаю или говорю». Большинство же парней, которых я знаю как успешных с женщинами, ведут себя уверенно и думают про себя «Я самый крутой самец в этом месте, это моё место».

Втяните живот, поднимите вашу голову и слегка откиньте назад, расправьте плечи, выпрямитесь, и держите себя так, как держит себя самый сильный и доминантный мужчина которого вы видели или слышали. Конечно, я знаю, это звучит слишком просто, чтобы это работало, тем не менее сделайте это.

Вероятно после того, как вы всё это проделаете, вы испытаете неловкость и смущение, но пусть вас это не беспокоит. Если вы постоянно будете практиковаться в поддержании уверенной позы, то со временем вы будете чувствовать себя в ней комфортно. И что ещё более важно, вы начнёте привлекать внимание женщин.

Помните, женщине не нужен ещё один слабак. Женщин не привлекают тряпки. Если вы будете вести себя мужественно, то женщины это моментально заметят и оценят на бессознательном уровне.

3) Двигайтесь немного замедленно. Пусть ваши движения и жесты кажутся естественными и продуманными заранее. Посмотрите несколько фильмов о Джеймсе Бонде. Например фильм «Грязные негодяи». Вы можете заметить, что Джеймс Бонд никогда не выглядит так, как будто он не знает что дальше делать. Заметили ли вы, что Джеймс Бонд никогда не суетится и не нервничает? Всё что он делает, он делает как бы чуть-чуть медленнее, чем нужно. Это делает его таким клёвым. Попытайтесь научиться тому, как медленно поворачивать голову, как медленно моргать, как медленно менять выражение лица и как жестикулировать медленно. Это отразиться на том, как другие люди будут воспринимать вас. Такой тип поведения как бы метасообщает окружающим «Мне комфортно»

4) Говорите уверенным голосом.

Большинство людей говорят слабым, пискливым голосом, тем самым они метасообщают окружающим «У меня нет уверенности\ldots у меня низкая самооценка». Это отталкивает женщин.

Если вы хотите привлечь красивых женщин, то вам необходимо взять несколько уроков у Барри Уайта. Вы должны научиться говорить глубоким голосом. Звук должен идти из вашей груди и добавьте немного баса в ваш голос. Так же вам необходимо научиться говорить медленно. отчётливо произнося каждое слово. Делайте паузы, они создают предвкушение того, что за этим последует.

Большинство парней говорит слишком много и слишком быстро, они чувствуют необходимость говорить, потому что они нервничают. Не делайте этого! Просто откиньтесь назад, расслабьтесь и чувствуйте себя комфортно, даже если вы молчите и создаётся напряжение в беседе. Если вы научитесь правильно использовать язык тела и голос, передавая с помощью них уверенность и собственную сексуальность то те техники, которым я вас учу, начнут работать в несколько раз эффективнее.
Глазной контакт

Ещё до знакомства с человеком очень много можно узнать по его взгляду. Вы можете увидеть как он себя чувствует, как реагирует на окружающую среду и как взаимодействует с людьми. Вы можете уловить его настроение, почувствовать насколько он конгруэнтен.

Если парень не чувствует уверенности, то при встрече взглядами он будет отводить глаза от более доминантного человека. Когда же он чувствует в себе уверенность и доминантную энергию его глаза не будут бегать, боясь глазного контакта, и его взгляд будет полон уверенности в себе, а не покорности.

Если парень не конгруэнтен, то он будет пытаться избегать глазного контакта. Если парень старательно будет пытаться выглядеть доминирующим, то это будет заметно: он будет мигать немного быстрее обычного или наоборот будет слишком долго держать свои глаза открытыми, не мигая при этом.

Когда же парень просто спокоен и уверен в себе, его взгляд будет выглядеть естественно. Расширение и сужение зрачка будет синхронизировано с его мыслями.

Вы всегда можете узнать, насколько парень позволяет другому человеку вторгнуться в его кто-либо «зацепил» парня, то он «уходит в себя» и его глаза будут отражать его мысли. Если же парень не реагирует на поведение других, находясь «в моменте», его глаза не будут бегать, пытаясь установить раппорт с другими людьми, его взгляд будет не реагирующим и люди почувствуют это.

Если мужчина сможет посмотреть в глаза женщины пристальным взглядом, то она почувствует возбуждение. Часто, когда женщина флиртует с мужчиной, она бессознательно проверяя его, может посмотреть на него пристальным взглядом, чтобы увидеть смутит ли это его, уйдёт ли он в себя, даст ли он ей на себя глазную реакцию. Если его глаза не дрогнут, и он без напряжения выдержит её взгляд, то она найдёт его более привлекательным.

Позднее, когда она будет уверена в его ценности и влечение перерастёт в отношения, она так же будет смотреть в его глаза, ища его реакцию на себя, и пытаясь понять будет ли он заботиться о ней. Вам необходимо совершенствовать искусство глазного контакта.

Тем не менее, нет совершенного глазного контакта, а есть естественный глазной контакт. Вам нет необходимости избегать глазного контакта с кем бы то ни было, однако и не нужно таращиться на окружающих.
Голос доминантного мужчины

Голос человека великолепно передаёт кто он такой. В течение нескольких мгновений, после того как человек открыл рот, становится ясным его социальный статус, его эмоции, и его положение в группе.

Вы всегда можете почувствовать его настроение, его эмоции, через интонации его голоса. Вы можете почувствовать, является ли он конгруэнтным и уверенным в себе. Вы сможете понять его положение в группе, является ли он доминирующим или подчинённым. Вы можете почувствовать, расслаблен ли он или напряжён. Когда парень испытывает тревогу, его голос начинает дрожать и нервная энергия расходится по всему телу. Когда же парень спокоен и чувствует себя комфортно это так же можно почувствовать по интонациям его голоса.

Люди всегда будут обращать внимание на такого парня, поскольку кажется, что для него это привычно, что он даже не удивлён тому, как много людей обратили на него внимание. Люди будут реагировать на него, а женщины будут привлечены им.

Мы социально запрограммированы верить в то, что глубина голоса это одно из самых важных качеств, и действительно, глубокий голос звучит хорошо. Но несмотря на это, все же для женщины это лишь одно из многих качеств мужчины, метасообщающих его ценность.

Уверенный в себе, доминирующий мужчина, излучающий позитивные эмоции, всегда будет иметь преимущество перед парнем у которого из всех достоинств только один глубокий голос. Мужская энергетика передаётся через целый комплекс качеств, а не через какое-то единичное качество.

Часто, если вы увидите группу людей в помещении, то вы всегда сможете распознать голос наиболее доминантного парня в группе. Парень, который так использует свой голос, всегда ожидает что его будут слушать. Его голос звучит естественно, в нем нет страха, он мощно резонирует и исходит из его груди. В то же время его нельзя заподозрить в том, что он умышленно так говорит, это его естественный способ общения.

Он говорит чуть громче чем остальные, но не кричит при этом. Он чувствует себя комфортно, когда он говорит, то окружающие внимательно прислушиваются.

Он говорит выразительно, он может игриво соглашаться или отклонять предложения, кричать или говорить шёпотом. Он может быть возбуждённым и приковать к себе внимание всей группы или же он может говорить и двигаться медленно и расслабленно, говоря сексуальным и гипнотическим голосом.

Характеристики его голоса (тембр, громкость, скорость речи и т. д.) никак не меняются, при любой реакции окружающих на него, он не реагирует. Его голос устойчив, как и его внутренне состояние.
Язык тела

Язык тела так же является информативным в том, что касается статуса человека, его уверенности в себе и его эмоционального состояния. Когда парень заходит в помещение, то его язык тела демонстрирует окружающим его характеристики: доминирующий он или подчинённый, расслабленный или напряжённый.

Парень, который спокойно занимает необходимое ему для комфорта социальное пространство, тем самым как бы предполагает что другие позволят ему это сделать, отличается от парня, который стеснён в движениях, боится занимать социальное пространство и тем самым демонстрирует низкий социальный статус.

Иногда парень может чувствовать игривость и легкое нахальство, он может танцевать, активно двигая руками, выражая внутренне состояние, он тем самым демонстрирует что он находится в моменте.

Он двигается, подчиняясь внутренним инстинктам и без сознательного контроля, он просто весело проводит время. Это совершенный, безошибочный язык тела. Этот язык тела означает, что парень не реагирующий.

Он не пытается сознательно контролировать свои движения, он просто чувствует расслабленность и занимает необходимое ему социальное пространство. Он весело проводит время и никто не вправе осудить его за это. Или же он может сознательно принять уверенную расслабленную позу и настроится на положительную волну, тогда его разум воспримет сигналы тела и внешняя поза вызовет состояние спокойствия и расслабленности в уме парня. Делайте то, что работает для вас.
Прикосновения

Парню с высоким статусом будет абсолютно комфортно когда он прикасается к другим людям и когда прикасаются к нему. Его не беспокоит то, что человек к которому он прикасается, может не отреагировать позитивно на это, поскольку прикосновения это часть того, кто он есть, это качество доминантного человека.

Иногда он может даже шутливо продемонстрировать свою силу, показав акробатические этюды своим друзьям, например забросив девушку на плечо и гуляя с ней таким образом. Когда женщина прикасается к нему, то он не чувствует какой-то сильной эмоциональной реакции, поскольку женщины трогают его постоянно и он привык к этому.

И хотя ему полностью комфортно и он повинуясь своему импульсу прикасается к женщине это не выглядит так, как будто он нуждался в этом прикосновении или так неловко, как это делают большинство парней.

Как Правило, после прикосновения он разрывает физический контакт с девушкой, и это приводит к тому, что девушка хочет продолжения. Прикосновения естественны для него, он всегда ощущает себя комфортно при физическом контакте.

\textbf{РЕШИТЕЛЬНОСТЬ} (Пошли! Иди сюда! Не пытаться предугадать реакцию, доминировать).

\textbf{ЮМОР} (уничтожающий).

\textbf{ПОЗИЦИЯ} (непредсказуемость; Я --- приз; мир --- забавное место; изобилие --- женщин много, и меня не будет беспокоить если я одну потеряю; ожидание того, что люди будут отвечать на твои вопросы и пытаться тебя впечатлять).

\textbf{СПОСОБНОСТЬ СОЗДАТЬ ВОКРУГ СЕБЯ ХОРОШУЮ АУРУ} (отражает позитивный опыт прошлого).

\textbf{НАХОДИТЬСЯ В МОМЕНТЕ} (говорить что думаешь, схватить девушку и забросить к себе на плечо, покрутить её).

\textbf{БЫТЬ БЕЗРАЗЛИЧНЫМ К ТОМУ ЧТО ДРУГИЕ ДУМАЮТ О ТЕБЕ} (чувство комфорта, даже если ты вышел голым на улицу или бзднул при разговоре).

\textbf{ПЕРСОНАЛЬНЫЙ СТИЛЬ ПОВЕДЕНИЯ} (манеры, способ разговора).

\textbf{ИМЕТЬ ЧЁТКИЕ ГРАНИЦЫ, КАКОЕ ПОВЕДЕНИЕ ТЫ ПРИНИМАЕШЬ ОТ ОКРУЖАЮЩИХ, А КАКОЕ НЕТ. ИМЕТЬ ГРАНИЦЫ, НА КАКИЕ ВОПРОСЫ ТЫ ОТВЕЧАЕШЬ ЛОГИЧЕСКИ, А НА КАКИЕ НЕТ} (не принимать дерьмо по отношению к себе, быть готовым всегда уйти физически от неприемлемого поведения по отношению к себе, распознавать и не поддаваться на манипуляции и провокации).

\textbf{ИМЕТЬ СОБСТВЕННОЕ ПОНИМАНИЕ ТОГО, ЧТО ИМЕЕТ ДЛЯ ВАС ЗНАЧЕНИЕ В ЖИЗНИ. НЕ ИСКАТЬ ОДОБРЕНИЯ ВАШЕМУ ПОВЕДЕНИЮ СО СТОРОНЫ ОБЩЕСТВА. СООТВЕТСТВОВАТЬ СОБСТВЕННЫМ ТРЕБОВАНИЯМ. КОНТРОЛЬ ФРЕЙМА} (ожидание, что люди будут отвечать на ваши вопросы, что бы впечатлить вас, прыгать через обручи; быть арбитром; быть способным заставить людей квалифицировать себя перед вами; умение создать течение социальной энергии в нужном направлении).

\textbf{НЕ НУЖДАТЬСЯ В ОДОБРЕНИИ ДРУГИХ. НЕ НУЖДАТЬСЯ В ПРИНЯТИИ. ГОТОВНОСТЬ РАЗОРВАТЬ РАППОРТ В ЛЮБОЙ МОМЕНТ} (Заставлять реагировать на себя).

\textbf{НЕ ПОЗВОЛЯТЬ ОКРУЖАЮЩИМ ВТЯНУТЬ СЕБЯ В ИХ ФРЕЙМ} (Плохое настроение, превосходство и т. д.)
\chapter{Аутентичность (искренность)}

Значения:

\begin{enumerate}
\item Подлинность
\item Искренность
\item Отсутствие манипуляций, обмана, демонстрация настоящего лица, без попыток одевать «маски».
\end{enumerate}

Аутентичность это идеальное состояние. У нас всех могут быть разные мнения, и никто не может утверждать, что исключительно его понимание аутентичности является подлинным.

Все мы существа несовершенные и все стремимся к истине, одновременно признавая что никогда не сможем постичь всю истину. Вы можете бояться подойти и заговорить с человеком, потому что вы сомневаетесь, достаточно ли вас ценности, что бы сделать это. Вы не должны задумываться над тем, отвечаете ли вы их стандартам, много ли у вас классных друзей, или сможете ли вы развлечь человека и т. д. вся та чепуха, которая, как вы думаете, даёт вам право подойти к другому человеку и заговорить.

Знакомство с другим человеком должно быть для вас естественным и комфортным, потому что вы знаете, что вы цените их время. Ваш разум просто говорит «Я аутентичен».

Конечно же, если вы серая посредственность и конформист то вы не сможете прожить настоящую жизнь и не сможете развить свою личность.

Научиться аутентичности это в ваших интересах, поскольку это всегда даст вам больший статус. Вы должны осознавать это. Вы можете считать аутентичность той отправной точкой, пройдя которую, вам больше не нужно будет вести себя как конформист, соглашаясь во всём с окружающими и пытаясь им угождать.

Люди всегда лучше реагируют на аутентичных людей. Будучи аутентичным вы всегда будете притягивать к себе внимание окружающих, вы метасообщаете, что ведёте себя искренно, и люди ответят вам взаимностью.

Когда вы не делаете каких-то отрицательных вещей людям, тогда вы знаете что люди испытывают только положительные эмоции в вашем присутствии, они доверяют вам и это даёт вам собственную сексуальную уверенность. Вы знаете, что вам есть что предложить, и это делает вас более дерзким и намного более реальным.

Мои друзья, отличные, уверенные в себе парни, потому что они знают, что у них есть ценность, которую они могут предложить другим. Быть отличным парнем не означает что вам нужно подлизываться к людям, просить у них что-то, всегда соглашаться\ldots если мы посмотрим глубоко внутрь, на истоки подобного поведения, то окажется что такие люди просто хотят реакции на себя, вместо этого, вы должны знать, что вы не делаете «добро» кому-то, скорее вы позволяете людям войти в вашу реальность и испытать великолепный опыт пребывания в ней. и если некоторым людям понравится ваша реальность, то это даст вам уверенность действовать более дерзко, зная что вы этого достойны.

Многие люди делают гадости окружающим с целью самозащиты, окружающие отвечают им тем же, затем они снова делают гадости\ldots и т. д. Это замкнутый круг реагирования.

Вы же должны построить свою жизнь таким образом, что бы запускать замкнутые циклы в правильном направлении. Обман, неискренность, манипулирование на короткий период времени могут дать вам результат, но в конечном счёте, вы вредите способом, который до конца не осознаёте. Таким образом, уже в этой жизни каждый может устроить себе на земле ад или рай.

Аутентичность --- это ваши базовые убеждения. Люди всегда находят привлекательными вещи, которые другие делают искренне. Если вы уверены в том, что вы можете делиться собственным опытом, не принимая во внимание интересно ли то, что вы говорите с точки зрения социального программирования, то окружающие всегда с интересом будут слушать вас.

Моя бывшая подружка использовала множество глупых шуток, но так как она думала что это и вправду смешно, то я смеялся вместе с ней. Наверняка у каждого из вас есть друг, у которого очень странный юмор, но использование именно этого юмора делает его искренним, делает самим собой. Это делает его таким оригинальным, и вы смеётесь вместе с ним. Он аутентичный.

Но конечно, если вы прямо сейчас пойдёте и начнёте использовать эту информацию, то это может и не сработать. Почему? Потому что одного сознательного понимания данной концепции недостаточно. Вы должны чувствовать собственную индивидуальность, понимать откуда она исходит и на чём основывается и тогда всё что вы делаете --- вы будете делать искренне.

Вы должны копнуть глубоко внутрь себя, на тот скрытый уровень, где лежит ответ на вопрос «Кто Я?»

Прожить свою жизнь хорошо и достойно, не означает поддаваться воздействию социального программирования и думать что «Я клёвый, я классный только тогда, когда у меня есть отличная внешность и много денег».

Многие люди проводят свою жизнь в погоне за этими идеями. Вместо этого, вам необходимо найти свой путь, свои ценности, своё понимание «кто я есть» и быть гордым за это. Или просто даже гордиться тем, что вы способны создать отличную энергетику в компании, это то, что может дать вам уверенность.

Аутентичность и доминирование вместе дадут вам возможность перейти из разряда людей, которые развлекают окружающих, к тому типу людей, общества которых хотят и считают их всегда желанными.

Если вы аутентичны, то вы уже имеете высокую внутреннюю самооценку, вы не нуждаетесь в подтверждении собственной ценности со стороны окружающих, вы просто предлагаете ценность окружающим вас людям без всякой мысли что-то получить от них взамен.

Вы не ждёте их благодарности и признательности, вы просто действуете в рамках собственной реальности. Вы больше заботитесь о сохранении ценности в своих собственных глазах, чем о подтверждении вашей ценности со стороны окружающих.

Вы знаете что никто не сможет предложить вам ценность, поскольку вы можете получить всё что пожелаете (девушки, которые встречаются со «звёздами», говорят «Он мог бы иметь любую девушку, если бы захотел». «Звезда» живёт в своей реальности).

Ценность не придёт к вам, если вы пытаетесь взять её от других людей. Вы должны быть способны слушать каждого без осуждения, без попытки навязать ему свой взгляд на мир.

Не слова сами по себе делают парня клёвым. Скорее это его способность общаться с женщинами таким образом, где он полностью открыт, несмотря на поведение женщины. Женщин не интересуют его внешность, шмотки или деньги. в первую очередь их интересует его личность, а не его вещи.

Парни, которые слишком нуждаются в женщинах, ведут себя по-идиотски. Они пытаются взять ценность от успешных людей или же они их «обсирают», потому что они сами не считают себя успешными.

Поступая таким образом они лишь укрепляют свой собственный низкий статус для себя самих, потому что они не идентифицируют себя с успешными людьми.

Интерес девушки вызывает не что именно говорит парень, скорее как он это говорит, его способность легко выходить за пределы логики и общаться способом, интересным для него. Парень, который сможет общаться с девушкой аутентично, всегда будет казаться для неё очаровательным. Независимо о чём именно он говорит, сам его стиль изложения очень многое говорит о личности парня.

Когда парень способен общаться открыто несмотря на социальное давление, то это демонстрирует его высокую ценность.

Вам необходимо найти определённый баланс между вашей личностью, вашими мыслями, вашим пониманием «кто я такой» и суметь искренне продемонстрировать это. Это потребует для вас самоанализа, вы должны понять зачем и почему вы демонстрируете окружающим те личности, которыми вы не являетесь. Откуда они у вас взялись в прошлом?

Например, личность серьёзного и напряжённого парня, который не способен веселиться и демонстрировать сексуальность, потому что боится быть отвергнутым.

Аутентичность --- это признак доминантности, потому что если вы манипулируете другими людьми, надеваете маски, пытаясь демонстрировать личность, которой вы не являетесь --- то тем самым вы метасообщаете окружающим, что вам есть что прятать и поэтому вы боитесь демонстрировать настоящую личность.

Когда вы пытаетесь примерять на себя личность крутого парня, такого альфа-самца, то вы теряете ценность. Если вы по настоящему доминантны, то это должно быть вашим настоящим «я».

Часто, когда люди пытаются развивать свою личность, они запутываются и теряют из виду свои настоящие цели. Они начинают развивать то, что в первую очередь, позволяет им удерживать внимание окружающих и получать их положительную реакцию на себя.

Многие, вместо того что бы стать уверенными в себе, становятся просто наглыми. Это как рабочие на стройке, которые со второго этажа кричат что-то проходящим мимо девушкам. Это не уверенное поведение, это просто наглое поведение. Это такой способ самообмана, попытка демонстрировать качества, которых на самом деле нет. Подмена уверенности наглостью.

Никто не сможет сказать, что такого парня отвергли, потому что он демонстрировал чужую личность. Поступая таким образом, он как бы страхуется от того, что его отвергнут. Ведь если его поведение отвергнут, то как бы отвергли не его, а чужую личность, которую он демонстрировал.

Парни постоянно так поступают. Они создают другую личность для того что бы не чувствовать неприятных эмоций при негативной реакции девушки. «На самом деле я не такой, она отвергла не меня». Это то, что называется «надевать маску».

В сущности, если бы парни просто демонстрировали собственную личность окружающие бы намного лучше к ним относились.

Проблемы, которые возникают при демонстрации той личности, которой вы не являетесь:

--- Развлекать окружающих, вместо того что бы быть просто интересным парнем. Вам необходимо перейти от «развлекающего» парня, к парню интересному, желанному в любой компании.

--- Неумение игриво бросить вызов.

--- Быть наглым и дерзким пытаясь что-то сверхкомпенсировать в себе, вместо того, чтобы быть просто игривым и уверенным в себе.

--- Слишком стараться казаться равнодушным и отчуждённым. В определённый момент времени окружающие вас люди начнут считать вас высокомерным.

--- Стараться казаться слишком умным. Например использовать какую то лексику, термины, занимать определённую позицию, которые применяются в профессиональной области деятельности и не годятся для повседневного общения. Если вы будете это использовать в нормальном повседневном общении то скорее о вас подумают что вы понтуетесь, чем заметят в вас исключительного профессионала.

--- Демонстрировать поведение несносного нахального мачо, когда окружающим становится очевидно, что вы просто хотите их внимания.

--- Быть тихим, невыразительным, слишком милым. Несмотря на кажущуюся противоположность поведению мачо, это так же можно трактовать, что парень слишком старается понравиться. Если вы что-то кому-то предлагаете, и они не показывают вам своей благодарности за ваше предложение, то не навязывайте им это. Обычно это будет рассматриваться ими как ваша попытка подлизывания к ним, это будет ваша демонстрация низкой ценности.

Если кто-то пытается использовать вас, то вы можете просто не замечать этого человека, сменить тему разговора, или вообще промолчать, как будто вы ничего не слышали. Просто игнорируйте те темы разговора, которые вам не нравятся, как будто они вообще не существуют.

По той же самой причине девочки отшивают парней, которые слишком явно пытаются снять их. В них нет естественности и непринуждённости, то что они делают, они делают неискренне.

Вам не нужны оправдания. У вас высокие стандарты и вы должны следовать им. Подумайте об этом. Для того, что бы удерживать высокие стандарты, вам необходимо научиться нести ответственность.

Стремление к своему лучшему «я» это не то, что вы делаете, для того что бы получить одобрение со стороны общества. Вы делаете это для себя, потому что создание лучшей жизни для вас завязано на вашей внутренней ценности. Это не всегда означает, что вы постоянно должны испытывать эмоциональную безопасность и самоуспокоенность.

Это означает, что вы стремитесь к лучшему «я» и не нуждаетесь в признании собственной ценности со стороны других. Вы делаете это для себя.

Вы не должны всегда чувствовать эмоциональное спокойствие. Вы сделали свой выбор, и вы стремитесь стать лучше. Вам нет нужды управлять своим эмоциональным состоянием, потому что вам на это плевать. Вы знаете, что такое победа, и вы должны знать горечь поражения.

Современные заблуждения:

--- социальное программирование;

--- множество свиданий вместо включения переключателей влечения;

--- просто брать номера телефонов и не звонить по ним;

--- номер телефона это инструмент, а не конечная цель;

--- вам не нужно тратить много денег;

--- вечеринки ведут к сексу;

--- проводить как дневное, так и ночное время с одной и той же женщиной;

--- синдром Мадонны (демонстрация развратного поведения). На самом деле нет связи между демонстрируемым поведением женщины и её сексуальными потребностями. Что бы выяснить сексуальные предпочтения человека требуется много времени.

Проблема: Многие парни не способны вести себя естественно, быть «в моменте», быть весёлыми без попыток сначала продемонстрировать собственную ценность. Они знают что понтоваться (квалифицировать себя) это отстой, но в то же время они придумывают различные хитроумные способы как им продемонстрировать собственную ценность. (Те же понты только более скрытые и хитрые). Они не способны чувствовать себя комфортно до тех пор, пока не продемонстрируют свою ценность таким хитровыебанным способом.

Общие черты данной проблемы:

--- программировать себя на постоянный поиск реакции;

--- заменить понты и подлизывание через демонстрацию ценности «творческим» способом.

Недостатки «Игры»:

--- социальный робот (мысли о том, что вами постоянно пытаются манипулировать окружающие и вы, в свою очередь, то же должны пытаться обманывать и манипулировать окружающими);

--- предполагать что вас не любят, и плетут заговоры против вас (если вы ищите это в окружающих, то обязательно найдёте);

--- думать что все женщины это такие наркоманы, где наркотиком выступает их состояние. Думать что все женщины шлюхи (если вы будете так думать, то вы будете находить подтверждение своим мыслям в окружающем мире);

--- постоянно думать о том, как повысить свою ценность. Сделаться одержимым этой идеей;

--- быть неспособным слушать других людей;

--- основывать свою личность и самооценку на том, насколько хорошо женщины реагируют на тебя;

--- думать что нет незаменимых и исключительных женщин, что если вы прокололись с одной, вы всегда можете найти другую, третью и т. д. (это общепринятая точка зрения в съёме, однако некоторые парни слишком этим увлекаются, никак не могут остановиться и продолжают постоянно менять женщин).
Трудности, возникающие при «классическом свидании»

Когда вы приглашаете девушку на «свидание», она воспринимает вас как человека, который реагирует на неё. Это плохой способ для того, чтобы разжечь влечение в ней. Если ваша реальность достаточно сильна для того, чтобы другие люди реагировали на вас, то вы будете определять что в вашем окружении будет составлять высокую ценность и что другие должны делать чтобы подстроиться под вас. Вы становитесь опорой, которая поддерживает состояние окружающих вас людей, люди реагируют на вас, и изменяя своё поведение, пытаются приспособиться к вам.

Женщинам нравятся парни, которые живут в своей собственной реальности. Вот почему традиционный способ «ухаживания» за женщиной даёт очень медленные и неустойчивые результаты. Когда вы пытаетесь развивать отношения с женщиной следуя традиционным путём, вы хотите ей понравиться через приглашения её в какие-то общепринятые места для свиданий, говорить ей комплименты --- то тем самым вы помещаете себя в её реальность и она чувствует это.

Для того чтобы получить девушку, вы должны привести её в собственный мир. Устройте ей экскурсию по вашей реальности. Приводите её в те места, которые вы сами любите, разговаривайте с ней о том, о чём вы любите разговаривать, и задавайте ей вопросы, что бы удовлетворить своё любопытство.

Когда вы общаетесь с окружающими людьми, рассказываете истории, шутите, то в первую очередь вы развлекаетесь сами, это больше для вас, вы сами наслаждаетесь общением и создаёте позитивную ауру вокруг себя. Вы навязываете другим собственную реальность. Вы можете сделать клёвым всё, что вы говорите, просто поверив в то, что это клёво.

Нет ничего плохого в том, что бы чувствовать влечение к женщине, хотеть её, но ваше чувство реальности, ваше понимание «кто вы есть» не должно зависеть от реакции на вас женщины, или от чьей либо ещё реакции.

Вы должны быть сильным, тогда люди почувствуют это и потянутся к вам. Вы не должны подходить к женщине думая о том, как её впечатлить, как заставить её «правильно» отреагировать на вас, и ваше внутренне состояние не должно зависеть от её реакции на вас.

Если она почувствует, что вы ищете её реакции на себя, или вы «пыжитесь», пытаясь продемонстрировать ей личность, которая не является вашей, то вы не сможете привлечь её. До тех пор пока вы находитесь в собственной реальности, её внимание будет приковано к вам.

Влечение это не сознательный выбор. Влечение к мужчине, который находится в своей собственной реальности, для женщины не обязательно понимать и хотеть логически. Это просто то, на что она реагирует эмоционально.

У вас должно быть убеждение, что вы классный чувак, просто заставьте её реагировать и приспосабливать её поведение к вашей личности. Она должна хотеть подтверждения собственной ценности от вас, и ни в коем случае не наоборот.

\RULE  На определённом базовом личностном уровне вы определяете себя как парня, который привлекает многих женщин и поэтому способен выбирать между ними. Вы не должны быть парнем, который отчаянно нуждается в женщине и хочет, чтобы она сама выбрала его. Женщина всегда почувствует разницу, между тем как действуете вы, и тем как поступает большинство других парней.

Как парню, который привлекает многих женщин, вам не нужно будет беспокоиться о том, что бы получить реакцию на себя какой-то конкретной женщины, для вас в этом не будет смысла. Когда женщина почувствует что вы легко, без напряжения с ней общаетесь, что вы не хотите взять ценность от неё, что вы не ищете её реакции, одобрения или восхищения вами, или даже секса потому что всё это у вас уже есть в изобилии, потому что в вас есть ценность.

Как мы уже говорили, в любом социальном взаимодействии всегда есть один человек, реагирует на других меньше, чем они реагируют на него. Когда вы перестаёте реагировать на других людей, не ища их одобрения, то они начинают чувствовать, что вы обладаете более высокой ценностью и сами начинают реагировать на вас. Когда вы общаетесь с женщинами, через своё поведение и метасообщения вы просто демонстрируете им свою личность, свою уверенность, которая приходит к вам через понимание кто вы есть.
